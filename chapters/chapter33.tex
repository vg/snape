\documentclass[a4paper,11pt]{article}
\usepackage{ulem}
\usepackage{a4wide}
\usepackage[dvipsnames,svgnames]{xcolor}
\usepackage[pdftex]{graphicx}

\usepackage[utf8]{inputenc}\title{A Difference in the Family: The Snape Chronicles}
\usepackage{hyperref}
% commands generated by html2latex


\begin{document}

\subsection{Chapter 33: Chapter 33 – Quirinus Quirrell}

\textbf{Severus Snape: The Middle Years, Part II - Quirinus Quirrell}

\emph{Monday, August 1, 1988 (three days after the full moon)}

The greeting that fine August morning was a bit different than usual.

"You'd better get hopping up that hill quick-like. You're being supplanted."

Snape didn't even turn around, not the slightest bit surprised by the gruff voice. "Am I to understand that you're expecting me to be taking off for parts unknown in the very near future?"

"Unknown to you, maybe. Not to me. Not to a few dozen dementors. But I didn't mean in your job."

That was intriguing. "If I'm not being replaced in my job, Moody, how can I be supplanted?"

"You're being supplanted in preferential status. You're not the baby any more."

"A younger teacher!" The wry expression might have been what, on Snape's face, would pass for a smile, or it might have been merely a grimace of sarcasm. "You have no idea how ecstatically happy you've made me with that news. Now they can patronize someone else for a change."

"You wait a couple of days. You're going to miss the attention."

"About as much as I'll miss you for the rest of the school year."

The two men parted, Moody back to the Ministry and Snape up the hill. Despite his studied lack of interest while in Moody's presence, Snape was curious to see what this younger teacher looked like. He assumed it was in the Dark Arts position, and he was thinking of scathing things to say about it to Dumbledore.

It turned out, as Snape joined the assembling teachers for a late breakfast and the first staff meeting of the year, that there were two new teachers rather than one. The first was a woman, probably in her early thirties and therefore a few years older than the twenty-eight-year-old Snape. She was introduced to the staff as Charity Burbage, who would be taking over Sapientia Dawson's job in Muggle Studies.

The second newcomer was a very young-looking man, though Snape suspected he was in his mid-twenties. He was pale and mousy looking, and his name was Quirinus Quirrell.

"Well," said Flitwick proudly, "I know you're not teaching Care of Magical Creatures because Max is still here, so I assume it's Defense against the Dark Arts." He looked around the table. "Quirinus took an Outstanding in his NEWTs in both subjects, you know, not to mention Charms." He smiled at Quirrell. "I thought you'd gotten a job at the Ministry. Regulation and Control of Magical Creatures, wasn't it?"

"Yes, sir," Quirrell said politely, and Snape was reminded of how he, too, had spoken to the others when he was first hired, as if he were still a student. "But after a while I thought I'd prefer something less strenuous. I fear I'm too much of an academic for creature wrestling."

"What sort of creatures were you wrestling?" Snape asked blandly.

"Trolls."

It was unfortunate that Snape had chosen that moment to take a bite of sausage. When he could breathe again, he apologized. "I'm not usually that inept with my food."

"That's all right," Quirrell replied. "A lot of people have had that reaction. Apparently it's an unusual combination."

"Do you remember Quirinus from school, Severus?" asked McGonagall. "He was about three years behind you as I recall, and in Ravenclaw."

"Not really," Snape admitted. "Slytherin tended to stick to its own."

"I recall you," said Quirrell. "Not that I knew you personally, of course. I was in fourth year when you were approached about the possibility of tutoring Sigfried Thurifer. He was a year ahead of me. We all remember your reply."

"Oh really," said Flitwick. "And what was that? I don't think I ever heard of a Ravenclaw going to a Slytherin for tutoring."

"It was in Potions," Quirrell explained. "Nobody could touch Snape in Potions, and Sigfried wanted the best."

"What was Severus's reply?" McGonagall asked.

Snape stepped hastily in. "He didn't need tutoring," he said quickly. "He just needed to reorder his priorities and use his resources more efficiently. I sent him back to Ravenclaw."

Across the table, Snape's eyes met Quirrell's. Quirrell was clearly sizing him up as well.

There was no way to escape Quirrell after the meeting because Dumbledore, predictably, asked Snape to show Quirrell around and get him settled into the Dark Arts position.

"You resent my being here, don't you," Quirrell said as the two ascended to the first floor Dark Arts classroom.

"Not at all," Snape replied. "I find it highly amusing when a brand new colleague tries to put me on detention. What were you going to tell them, anyway?"

"I didn't have anything planned," Quirrell admitted. "I was more interested in what you'd do. You got out of it nicely. They always said you had a quick brain."

Snape waited until they were inside the classroom, away from all possibility of McGonagall coming across them on her way to her own office, and then spun on Quirrell. "You didn't have anything planned, but you did your best to embarrass me anyway. You're going explain, and you're going to do it now."

Quirrell didn't back down, even though he was an inch or two shorter than Snape. "I wanted you to know that I was ready for you. You're not going to push me around. The others may not know you were the kind to string people up by their heels for no reason, but I do. You won't take me by surprise."

"No reason? Thurifer played innocent, then? Why didn't he go to Flitwick if he was so innocent?"

"He was afraid of you. I'm not, though." Quirrell thought for a moment. "What was he doing if it wasn't just asking for tutoring?"

"He was taking money from a couple of Gryffindors to lead me into a trap. I trusted him and went, too. I was lucky someone was watching my back that day. We figured hanging Thurifer upside down in a girls' lavatory was a lot nicer than what they planned on doing to me."

Quirrell tilted his head to one side. "Taking money for stabbing someone in the back does sound like something Sigfried might do," he admitted. "Still, you must agree I had some cause for antagonism."

"No," said Snape, "I don't. And if you always make it a habit to walk blindly into things you only half understand, you're going to get yourself into serious trouble one of these days." That being said, Snape looked around the office. It was full of boxes and crates, all opened but none unpacked. The boxes were full of books. "What did you do? Raid a library?"

"Yes," Quirrell replied. "My own. These were just the ones I thought would be useful here. For the class, of course."

Snape picked up a couple of the books. \emph{One was Seven Steps to the Mountain of Darkness}, and the other \emph{When Light Fails in the Depths of the Mind}. Both were books of the darkest philosophy. A glance at the other boxes showed Snape they were full of tomes on dark creatures, dark objects, and dark spells. "You believe in a thorough grounding in your subject," he commented.

"How can you instruct others how to fight it if you don't understand it yourself?"

\emph{I understand it}, Snape thought. \emph{I lived with it for three years, and I don't see one book here that talks about the lust for power that sends people down the dark path. There may be ivory-tower types who are fascinated by the Dark Arts just because they're there, but the dangerous ones are after power, and the Dark Arts are just a means to an end}.

"You certainly have a good theoretical background," he said to Quirrell, and headed for the door.

"Were you supposed to give me some kind of orientation?" Quirrell asked.

"Do you know what you're supposed to teach?" Snape asked.

"Yes, of course. I took these classes for seven years."

"Do you know where your supplies are?"

"Yes."

"Then you already know everything I could possibly tell you." Snape nodded once in farewell and walked out the door.

At first he headed for the great marble staircase down to the entrance hall and thence to his own office, but after a moment's hesitation, Snape changed his mind and went up instead. Up to Dumbledore's office.

"Why did you hire him?" Snape demanded the moment he was inside.

"Good afternoon, Severus. I trust you are well. I see we are about to have our usual annual discussion." Dumbledore waved Snape into one of the chairs and went to pour a glass of mead.

The conversation between Snape and Dumbledore went quickly, mostly because they'd had the same conversation at the beginning of the previous five years.

"And in this case," Dumbledore pointed out, "he is exceptionally well qualified. He is a brilliant scholar, very well-grounded in his subject on a basic level, and with a depth of knowledge that few much older people ever achieve. I might venture that he could teach you something about the Dark Arts."

Snape opened his mouth, then closed it again. He couldn't refute Dumbledore's statement since he had no information on the extent of Quirrell's studies. The only thing he could offer was a technicality. "But he's not supposed to teach the Dark Arts, he's supposed to teach Defense against them. What practical experience does he have?"

It was not a strong argument since the school would never engage teaching aids such as real ghouls and vampires, so the course was of necessity mostly academic anyway. Snape returned to the lower levels of Hogwarts defeated.

Quirrell soon began to grate on Snape's nerves in more ways than one. The first was that Quirrell did, in fact, know more about the Dark Arts than Snape did. Snape's familiarity with the subject was on a highly practical level regarding the magic, with the actual creation and use of spells, with real combat against them in controlled but nonetheless physical situations, with poisons and their antidotes, but his knowledge of dark creatures and enchanted objects was book learning and more limited.

Quirrell, on the other hand, had amazingly detailed knowledge of all aspects of Dark Magic - spells, creatures, objects, poisons - the only thing he was short of was the physical experience. Snape quickly became wary of even bringing up the topic of Dark Arts in any casual situation, since Quirrell would begin to expound based on the vast range of his studies, and it would soon become apparent to anyone listening that Snape was out of his league. Snape started spending more time in the library.

To be honest, Snape was certain he could best Quirrell in a duel, but dueling on school grounds was frowned on, and it certainly wasn't part of the curriculum.

Another source of irritation, though Snape would have been incensed if anyone else mentioned it, was that Snape had gotten used to the position of being the youngest teacher in the school. It meant he was asked to do things that required greater speed or agility, but it also meant a certain amount of leeway was given him in terms of behavior, and a certain amount of coddling. Now Quirrell was usurping Snape's place, and the specter of jealousy raised its ugly head.

And then there was the matter of Quirrell having been a lower level student aware of Snape and his activities as a fellow student - things the other professors could never know.

"How did you manage to stay free?" Quirrell asked one evening at dinner, shortly before Halloween. Snape and Quirrell sat next to each other at the staff table, and it was impossible to completely avoid conversation, though Snape tried.

"Free? Free from what?"

"Come now. You were hand in glove with the Blacks, the Lestranges, Avery, Mulciber, Rosier, Wilkes... The reports of deaths, roundups by the Ministry, and convictions read like a class reunion for your friends in Slytherin house. How did you manage to slip out of the noose?"

"What makes you think there was a noose to slip out of?"

"Let me see," Quirrell cupped his right elbow in his left hand and laid his right index finger on his jaw in a parody of thought. "When I left Hogwarts, I got an internship in the Ministry - Department of Magical Creatures. Then I worked my way up the career ladder for a few years before I applied for this job. You graduated in what? Seventy-eight?"

"Is this leading to a point?"

"I was just wondering what you did during those three years before you joined the teaching staff at Hogwarts. No one seems to know."

"Is that an admission that you've been nosing around trying to find information on me?"

"No. Not at all. It's come up once or twice in chats, nothing specific. I was just curious."

"Tutoring and potion brewing. All for private clients. Satisfied?"

"I suppose I shall have to be. I was hoping for something more colorful and exciting."

"Sorry to disappoint."

On Halloween Snape kept his annual day of quiet reflection to remember Lily. He never mentioned it to anyone, and yet he had the feeling that Quirrell noticed even this. All in all, Quirrell was becoming an insufferable busybody.

It was shortly before the Christmas break that Charity Burbage did what Snape had been expecting her to do ever since August. She settled herself into the seat next to him one snowy day in December. "We haven't really had a chance to get to know each other," she said.

"Barclay's," Snape replied, "and yes I do have a bank card."

Professor Burbage was not put out in the slightest. "Would you mind if I, if I touched it?"

Relieved that she was at least direct about it, Snape reached into a pocket in his robes and drew out an honest-to-goodness muggle wallet. Burbage went into ecstasies. Snape was ready to stand up and leave the Great Hall right there, but restrained himself, pulled the little plastic card out, and handed it to her. She examined every detail with great eagerness.

"How does it work?" she asked, adding a few seconds later, "If you don't mind my taking up your valuable time."

"Not at all. Do you see this dark stripe on the back? It's a magnetic strip that can be read by some kind of computer. I can insert the card into a computer terminal in a wall, enter a secret code, the computer checks whether or not I have money in my account, and if I do, it will give me some of that money, deducting the sum from my account. Instant cash, twenty-four hours a day. The whole process takes less than two minutes. Much nicer than a nasty rough ride in a nasty old cart through dark underground passages."

Burbage studied the card with greater respect. "What's the secret code?" she asked.

"A secret," Snape replied, retrieving his card and replacing it in his wallet.

She blushed crimson, a rather attractive color on her otherwise plain features. "That was rude of me," she said. "Do you think I might ever watch you use it?"

"You'd have to visit London with me. There aren't any bank teller machines in Hogsmeade."

"They say you... frequent restaurants. And go to... movies."

"From time to time. Plays and bookstores, too. If you joined me one evening, you could probably claim the whole excursion as work-related and get overtime for it."

"What's overtime?" Burbage asked, then giggled at the expression on his face. "Seriously though, I may take you up on the trip to London some time."

Christmas with its trees and snow came and went. Snape finally managed to take Professor Burbage to London in March 1989 to see not only how a bank teller machine worked, but also to explore the intricacies of cash registers, elevators, and the peculiar etiquette of black taxicabs.

"But the fare was only five pounds twenty, and you gave him..."

"It's called a tip. It's expected that you pay a certain percentage above the fare..."

"But why don't they just raise the fares?"

"It's supposed to ensure better service if it's voluntary. In America, I hear, it's fifteen percent for waiters in restaurants, while in certain eastern European countries you tip with cigarettes."

"You can't be serious!"

"Marlboros in Hungary, and Kents in Romania."

"Now I know you're joking. Not even muggles would do that!"

Burbage was enchanted by the movie. It dealt with a young muggle man who felt himself cheated of an inheritance only to find that the money had gone to a brother he never knew he had, a brother with a curious mental deficiency. After kidnapping this brother to claim the inheritance, the young man found himself growing fond of the strange, handicapped sibling, finally realizing the brother's needs were greater than his own.

"That was so touching," Burbage exclaimed as they left the theater. "Do muggles really have ailments like that?"

"Yeah," said Snape, imitating Raymond Babbit. "Yeah." Burbage giggled, but Snape had just noticed on one of the posters the name of the actor who played Raymond - the same one who had played Tootsie so many years before. The wonder of the acting profession filled him again. \emph{Merlin, I wish I could do that, he thought}.

Then it was June, and Snape was once again in Dumbledore's office. "He's staying isn't he? This one is staying, and it proves the curse you've been fobbing me off with is a fraud!"

"Well, no, not exactly," Dumbledore replied. "He has asked for a year's leave to do field research. Even if he does return, it will not be two consecutive years, and it will therefore prove nothing."

It was just like the old charlatan to clutch at a technicality as evidence of what Snape had now long regarded as an unsubstantiated hypothesis. What did they have to show that the Dark Lord was not utterly destroyed? The Dark Lord's appearance? Even Dumbledore admitted that he did not know what kind of Dark Magic would have caused that transformation. The fanatic actions of Bella Lestrange and a few friends? It was the sort of thing that Bella would do. The continued sensitivity of the dark mark on his arm? It could be inherent in the mark itself and unrelated to the presence or absence of the man. The only other thing was the pattern in the tenure of the Defense against the Dark Arts position, and Snape had long ago pointed out that it was a pattern artificially extended by Dumbledore's selection of teachers he knew were temporary.

"They all decide to leave at the end of one year for some reason or other." Snape was continually insisting. "Not one opts to stay on. Now it might be proof of something if one of them was intending to come back for a second year and mysteriously came down with dragon pox, or was crushed by a crowd at a late-August performance of the Weird Sisters. Then I might agree that there was something in the idea. But maybe the problem lies with the Headmaster. Maybe you just hire unstable people."

Now, with Quirrell, the matter was about to come to a head. They were about to discover once for all if there was a curse. Quirrell would go on his year of field research, would return for a second, nonconsecutive year, and then would want to stay for a third. That would be the moment of truth! Even before the students boarded the train to London at the end of June, Snape was obsessing on the whole idea of Quirinus Quirrell.

Then, over the summer, it finally hit Snape how diabolical Dumbledore was. There would be no proof of the nonexistence of the curse until Quirrell started his third year. Only then could Snape go to Dumbledore and say, "See how wrong you've been!" but by then Quirrell would be so firmly ensconced in the position that Dumbledore would have no excuse to dismiss him. He would stay on, the permanent Dark Arts instructor. In the moment of victory, Snape would face total defeat.

Snape was really beginning to hate Quirinus Quirrell, and the man wasn't even going to be there for Snape to lash out at. He was going to be in Africa, or South America, or Eastern Europe. Snape would have hoped that Quirrell would have an unfortunate encounter with a dragon in Romania, except that that would prove that Dumbledore was right about the curse.

It was thus in this mood of sublime discontent that Severus Snape returned to Hogwarts in August 1989. The presence of Alastor Moody was of no more significance than a gnat that Snape could swat aside. The ordering of supplies was routine. The new Dark Arts professor was an intentional stopgap, and Snape, more than ever before, was dreading the resumption of his teaching duties and the classes from hell.

Then, on the morning of the first Friday in September, Snape found out what hell was really like.

\emph{Friday, September 8, 1989 (the first quarter)}

It did not register at once on that day when Snape walked into the classroom to meet the brand new first years from Gryffindor house. Everything seemed perfectly normal. There was that little trick of vision that had Snape wondering for a moment if he should ask Madam Pomfrey to check his eyesight, that sudden impression that maybe he was seeing double. He glanced down the list of students. The Slytherins were already known to him. It was only McGonagall's sweet charges who were new. There was no sense of foreboding as he went down the list, not even when he got to -

"Jordan, Lee."

"Here, sir."

The name at the bottom seemed at first to have been copied twice, but it was in fact -

"Weasley, Fred."

"Here, sir."

"Weasley, George."

"Here, sir."

There was no mistaking them. It had been no trick of vision or seeing double. There before him were two Weasleys as like each other as peas in a pod, if anything so flamingly red could ever be compared to a pea.

Even then, Professor Snape continued blithely with his lesson. He was unconcerned, unwary, totally and serenely unprepared for what was about to be unleashed on an unsuspecting world. After all, he'd already taught Weasley children. Bill was steady and dependable. Charlie was fiery, but focused. Percy was studious, polite, and a joy to teach.

Fred and George were going to be just like their older brothers, of that Professor Snape was absolutely certain.

Nothing happened during that first lesson to disabuse Snape of this pleasant dream. The Friday morning Potions class was usually a little hectic, sometimes even tense, because it was the only class the first-year Slytherins had with the first-year Gryffindors, and the Potions class was the first time the two groups met together since the Sorting. There was always posturing, the trading of rude gestures, and \emph{sub rosa} insulting between the boys, and frequently between the girls as well. Snape's big job was to keep them focused on the task and away from each other.

The first lesson went reasonably well. Only one Gryffindor sprayed wart remover at a Slytherin counterpart, and only three out of twenty potions were utterly useless. At lunch, when McGonagall asked Snape how the morning had gone, he told her it was one of the better classes of first years, and that he hoped for a fairly uneventful year. She raised her eyebrows, but otherwise said nothing.

The following Friday, Snape realized that he had no trouble telling Fred and George apart. True that in form and feature they were identical, and that their studied uniformity of hair, gait, and expression operated to keep them confused in the eyes of their fellow man, but there was an indefinable difference in the sparkle in the eyes of each that Snape caught at once. It took him half the class to realize he was reading them, and the rest of the class to decide that he wasn't going to tell them about it. It isn't every day that the ordinary mortal has the advantage of identical twins, and Snape wasn't about to give it up. He did use the right names, and Fred and George were clearly impressed.

On the third Friday, the skunk made his appearance. As Snape opened the door of the classroom for the morning double Potions lesson, his olfactory senses were assaulted by something that seemed no more nor no less than a combination of garlic, rotten eggs, and burning tires. The students held their noses and fled to the entrance hall. Snape, being in a position to have to deal with the problem, entered the room.

There was no mistaking the plump black and white-striped animal's intent. It hissed, it stamped, it turned its back and lifted it hindquarters - and Snape was out of the classroom and on the other side of a good, thick door in a shot. He dismissed his class. He needed backup.

"It's a what?" Max Kettleburn was practically rolling on the floor with mad laughter.

"A skunk," replied Snape with as much dignity as he could muster.

"That's a North American animal!"

"It is, nevertheless, in the Potions classroom. What are you going to do about it?"

"Why me?" Kettleburn chuckled, then burst out laughing again.

"You're the Care of Magical Creatures instructor."

"Severus," Kettleburn howled with glee, "a skunk isn't a magical creature!"

Then there was Hagrid.

"Well, I don't know as that's my business, seein's it's not a native creature. More of an import, like. Shouldn't you be checking with the Ministry? Control and Regulation."

But Control and Regulation was also for magical creatures, not garden variety skunks. Snape detected very little sympathy among the officials that he contacted. By this time the afternoon classes had been canceled as well.

"Tomato juice," was Professor Burbage's contribution. "The muggle literature I've researched says that bathing in tomato juice will help take away the smell if you, eh, get sprayed."

"Thank you ever so for that contribution," Snape told her. "It happens to be an eventuality I'm trying to avoid."

By this time the skunk had been in the Potions classroom for more than four hours. Its presence was evident in the passage outside the room, and was beginning to permeate the upper areas of the dungeons. Slytherin students heading for the common room and the dormitories held their noses, but the time was fast approaching when even that would not be enough.

Snape went to Dumbledore for help.

"Well clearly," Dumbledore said, "someone must go in and immobilize the skunk. Then we can transport it... to the proper authorities."

"Will you go in, sir?"

"It is not my classroom."

Many gallons of tomato juice were brought in to the teachers' bathroom. The students were arbitrarily restricted to their common rooms and dorms. The teachers assembled at a respectful and safe distance. Snape managed to restrain and encage the horrid beast, giving it to Hagrid in a sealed cage. Snape then hurried up to the teachers' bath to immerse himself in tomato juice. He clothed himself in new robes after that, having burned the old ones.

They had to fumigate the teachers' bathroom.

The next day, Saturday, brought the revelation that the perpetrator's of Friday's dastardly deed were possessed with neither the sophistication nor the sense of self-preservation necessary for a life of successful crime. At breakfast, a significant number of Gryffindors, mostly boys, sported little glowing badges that read "Slytherin Stinks." A very brief application of Professor McGonagall's well-honed interrogatory skills uncovered the guilty, and Fred and George Weasley found themselves doing a string of detentions in the dungeons with the result that by the weekend before Halloween, the Potions classroom and the stairway down to within one level of the Slytherin common room were spotlessly clean.

During the enforced servitude of the Weasley boys, Snape learned a couple of things about them from bits and pieces of conversation they made no attempt to hide, almost as if it never occurred to them that he might overhear. The first, more disturbing thing was that they harbored a deep, almost passionate sense that Slytherin house was their enemy. Not any individual Slytherin, but Slytherin as a cosmic entity, an overarching concept.

There was, if Snape went hunting through the compartments of his mind looking for it, a vivid recollection of a Quidditch game, and of Molly Weasley's fanatic partisanship not only for Gryffindor but against Slytherin. Snape suspected then that Fred and George may have been named for their uncles, Fabian and Gideon Prewett. They would have been about two years old when the Prewett brothers died. Fred and George would have grown up on stories about the dark times, and the role Slytherin house played in the Dark Lord's rise to power.

The second thing was that, anti-Slytherin prejudice aside, there were scarcely any boys at Hogwarts who were as open, cheerful, and optimistic as the Weasley twins. The world was a game, a challenge of their wits and creativity, and they expended a great deal of intellectual energy looking for ways to exercise their talents. But through all the whispers and eleven-year-old giggles, Snape caught no hint of malice. Even the crusade against Slytherin was more of upholding family pride than actual personal dislike.

Fred and George Weasley were as unlike James Potter and Sirius Black as any pair of pranksters could be, and for that Snape was immensely grateful. He went so far as to quiz them about their deed on the last day of their detention.

"By the way, Weasleys, where did you get the skunk? It isn't the sort of creature one keeps in the family as a pet."

George glanced at Fred, and two stifled giggles turned into a pair of snorts. "Do you know about \emph{The Quibbler}, sir?" George asked.

"I am acquainted with the name. I do not read it."

"Well, old man Lovegood lives near us, and he thinks skunks are related to jarveys. So he brought a couple in to breed. He expected them to keep the gnomes out of his garden. They did. They kept everyone else out of his garden, too. Now he's sending them all back, so we borrowed one."

"I see. A pretty prank, but if you ever do anything like that to me again, I shall nail your shoes to your feet with your own toenails. Do you understand?"

It may have been the wrong thing to say, for the eyes of both boys lit up like Christmas trees.

"Would you, professor?" gasped Fred.

"Could you, professor?" whispered George.

"Will you teach us how?" both boys chorused.

Now Snape had the measure of the twins and knew exactly where he stood. "If you never bother me again," he said sternly, "I'll consider it."

"Yes, sir!" The boys were out the door and up into the Great Hall like two bolts of lightning, for Saturday lunch had just been served, and food was a major motivating force in their lives.

The following Saturday was November 4, and the first Quidditch match of the season - Slytherin against Gryffindor. Snape kept a close eye on the Weasley twins during the whole match, but other than wear their "Slytherin Stinks" badges, they did nothing. Nothing that affected the game at any rate. Snape had his doubts about the sudden fit of sneezing that affected the Ravenclaw student who was the commentator, but there was nothing he could prove.

Fred and George Weasley confined their pranks to Gryffindor house except for one or two excursions against students of the other houses, and Snape lapsed back into his normal patterns of life. By Christmas break, he'd almost forgotten the skunk incident. All the Weasleys, Charlie, Percy, Fred, and George, went home for the holidays, and Hogwarts was at peace.

There must have been something about the Weasley home that fired Weasley children in unique ways. Bill had always arrived back from his breaks with renewed determination, Charlie with increased fire, and Percy more dedicated than before. It was no different with Fred and George, who seemed to have imbibed nothing of a practical nature from their brief sojourn at the family hearth.

"Smell? No, I don't detect any..." Snape began in response to Sprout's outraged question as they met in the entrance hall one Friday morning in early February on their way to breakfast. Then it hit him. It was either some unfortunate soul suffering from a gruesome gastrointestinal disorder, or it was a stink bomb.

"Oh, that is foul," said Snape, backing away from the entrance to Hufflepuff house. "And here I thought your house was always so neat and tidy."

"This had better not be Slytherin!" Sprout hissed at him.

"Hardly," Snape assured her. "Slytherin tends to be up close and personal. If it was a student who was targeted, I'd worry, but not the whole house." He considered the question for a moment. "You remember my September house guest, don't you?"

Sprout nodded.

"Well, aren't you playing Gryffindor tomorrow? I don't think they'd be dumb enough to sport 'Hufflepuff Stinks' badges, but you never know with Gryffindor."

Together the heads of Slytherin and Hufflepuff approached the head of Gryffindor house.

"What makes you think it's the Weasley boys?" McGonagall chided them. "It could have been anyone."

"I've been here eight and a half years," Snape said, "and Pomona longer, and we've never had a stink bomb set off in the entrance hall or dungeons before. Now, suddenly, we have it twice. It points to a newcomer, and that points to the first years. Not only that, Hufflepuff gets attacked the day before their Quidditch match with Gryffindor. Coincidence piles on coincidence."

"I still believe you're jumping too quickly to con..." - McGonagall paused - "...clusions." She was looking over Snape's shoulder.

Snape turned. Sure enough, 'Hufflepuff Stinks' badges flashed on Gryffindor robes. McGonagall rose majestically, sailed across the Hall, seized Fred and George each by an ear, and hustled them both up to her office. That evening the twins began a month of detentions scrubbing down the corridor outside the kitchens, and were barred from viewing the next day's Quidditch game, neither of which seemed to depress the two in the slightest.

It was an easy thing to accost the pair as they started up the marble staircase to their dorms in Gryffindor tower. Crossing his arms on his chest, Snape regarded them with something like a sneer. "Rank amateurs," he said. "You don't deserve my toenail hex. Imagine pulling the same trick twice."

"No," Fred insisted, "it wasn't the same. Last time it was a skunk. This time..."

"It was directed against a group and it involved a stench. It was the same trick. Face it. All that children your age can think of is falling down, bad smells, and embarrassing noises. Pathetic."

"We can do better next time," George pleaded. "Give us another chance."

"All right." Snape had intended to concede if they asked. "But something creative this time. And don't hurt anybody." He watched coldly as the two promised results then scampered upstairs. He was rather hoping they would come up with something good, because he had plans.

It was a matter of priorities. Snape wanted Quirrell to come back to his job to counter the curse theory, but he didn't want Quirrell to establish tenure in the position. A young professor entrenched in the Dark Arts position was more of a threat to Snape's own dreams than battling Dumbledore over the curse year after year. At least in the latter situation, Snape could keep up his hopes from year to year. But if Quirrell could hold on, Snape's dream of getting out of Potions was doomed for decades to come.

If they could live up to their potential, Fred and George were the best weapon Snape was ever likely to find. It was just a matter of proper prior planning.

All that spring, the Hogwarts students were subjected to a series of bizarre occurrences. Several Ravenclaw students developed severe cases of warts, a condition finally traced to a particular stone in the wall of their tower staircase. Almost immediately afterwards, some Gryffindor students contracted a painless purple rash that was discovered to come from brushing the railing on their own tower's stairs. Snape commented rather publicly that it looked like a pattern, and the hexes stopped. It took three weeks for the next one to come, but it was a beauty.

One March morning at mail call, a nearly imperceptible shimmer in the air over the Great Hall caused every single mail owl to lose control of its bowels and defecate onto the tables at practically the same time. Students were leaping up everywhere, and Snape was pleased to note that Fred and George were also splattered by the incontinent owls. It was a sure way to deflect suspicion from themselves.

As the end of the year approached, Fred and George confronted Snape. "What about the toenail hex, Professor," said Fred. "You promised."

"You have not demonstrated yourselves to be worthy," Snape replied. "You'll have to wait until next year - if you've improved."

"I don't think there is a toenail hex," George told his brother. "He's been conning us."

It was a challenge not to be ignored, especially since Snape had been casting nonverbal spells since he was a child. George had already turned and was walking away when he stopped, a triumphant smile spreading across his face as he stared at his feet. "You gotta teach me that one, Professor. You just have to."

"Come back with something creative in September, and we'll talk," Snape said. The boys agreed.

That summer flew by in a burst of creative energy such as Snape had not felt since the days when he battled Black and Potter. He pulled out his old Advanced Potions book and reviewed all the curses he'd invented all those years ago, then began the long, careful process of refining and tailoring them to fit one specific target. In the process, he came up with several new ones as well. A major priority was the reworking of the toenail hex, since Snape didn't feel he should give the Weasley twins the full-powered one. He didn't trust them to use it with any discretion.

There were moments, odd reflective times, usually just before falling asleep, when Snape realized that what he was doing was childish and petty. Worse, if Dumbledore ever found out that one of his teachers was planning a hex campaign against another, he might withdraw his protection and give Moody what he wanted. But at this moment, Snape didn't really care. The only thing that had kept him sane over the last nine years was the hope that Dumbledore would finally relent, take him out of the Potions position, and give him Dark Arts. Now that hope was about to be dashed, and it was Quirrell's fault. Nasty, stuck-up, snide, opinionated Quirrell.

For years Snape had felt as if he were going to explode from pure frustration. This year he had the satisfying feeling that it was finally going to happen.

\emph{Wednesday, August 1, 1990 (three days after the first quarter)}

"There he is, the second-youngest teacher at Hogwarts! Looks like Albus plans on keeping this one."

"Good morning, Mr. Moody. No invitation to Azkaban today? Don't tell me there are no vacancies this year." Snape managed to meet Moody's normal eye, but still couldn't bring himself to look into the spinning blue one.

"I can be patient. My sources tell me this Dark Arts teacher gets under your skin like a case of hives. I'm just going to sit back and watch you square off against each other. If Albus gets tired of you, I may get my wish after all. Enjoy your year."

\emph{How does he know about me and Quirrell?} Snape thought as he walked up the hill to the castle. \emph{Is Quirrell here already? Did he say something to Moody?}

Quirrell was indeed already there, seated at the table in the center of the Great Hall where the teachers usually met for breakfast on the first day. Snape walked quietly up behind him and said, "How was your sabbatical?" To Snape's surprise, Quirrell jumped at the sound of his voice, as if startled. "Sorry," Snape said. "Didn't mean to surprise you like that." \emph{If I make a point of being nice to him now, no one will suspect me when the fun starts.}

"You d... didn't surprise me. I was just concentrating on something else."

"So, how was your sabbatical? Did you manage to get everything done that you wanted to?"

"Most of it. It was... Have you ever done field work before?"

It was an odd question, especially coming from Quirrell who'd claimed to have worked with trolls for the Ministry. "Yes," Snape replied cautiously. "after a fashion. But only in Britain. I've never had the chance to go abroad. Why?"

"Just wondering. I was in A... Albania last month. Before that in the mountains in Transylvania. Brasov, Castle Bran. V...ampires, you know." Quirrell stopped.

By this time the rest of the teachers had gathered, and Snape joined the other heads of house. Dumbledore welcomed them all, and they began eating and discussing summer vacations and the routine of starting school again. No one particularly remarked on Quirrell's presence except that Flitwick commented that it was nice to start the year with all familiar faces for once. Quirrell, deep in a discussion with Sprout about mandrakes, barely noticed.

"By the way, Severus," Quirrell said as they rose from breakfast, "I brought back a few souvenirs. Would you like to see them?"

Snape was surprised, but agreed, and followed Quirrell up to the Dark Arts office on the second floor. There were some crates on one side of the room, but the thing that caught Snape's attention at once was a small cage on Quirrell's desk. It contained a snake, grayish brown with a thick, dark, zigzag marking along the length of its back. It was about twenty inches long.

The snake raised its head as the two men walked in, and seemed to be watching them.

Snape examined the snake in its cage as Quirrell rummaged through his crates. Quirrell seemed to be taking an inordinate amount of time, but since Snape found the snake fascinating, he didn't really mind. The little reptile was quite active, rolling in liquid coils and darting its tongue. It made no attempt to strike at the sides of the cage, contenting itself with moving its head up and down as it regarded the man.

"Ursini's Viper, isn't it?" Snape asked. "They're native to that part of Europe."

Quirrell lifted his head from a crate, a crate from which he had yet to extract a single item. "Yes, intelligent little thing. Poisonous, too, so don't touch."

"Hemotoxin," Snape retorted, just for the pleasure of showing Quirrell he wasn't totally ignorant about snakes. "Rarely fatal, though there have been cases... What are you looking for?"

"Here," Quirrell gasped, pulling several wrapped objects that proved on unwrapping to be gris-gris bags from Haiti, voodoo dolls from New Orleans, Santeria drums from Cuba, and minkisi from the Congo. It looked for all the world like the collection of Snape's great-grandfather Wensley. "Do you know what these are?"

"I have my own," Snape said. "I've had them since I was a child. What are these?"

"This," said Quirrell, holding up a twelve-inch fang, "is the tooth of a Kulshedra. It starts as a Bolla, a great snake, then after twelve years undergoes a metamorphosis into a dragon with nine tongues. I brought some of the tongue for you - dried and pickled - for the potions store." He fished in the crate and brought out six jars, three of each kind of preserved tongue.

"Why, thank you," said Snape, suddenly ashamed of his own meanness regarding Quirrell, for the tongue was rare and expensive. "I hope you didn't spend..."

"N... not to worry. Since it's technically for school stores, I'll submit a voucher to Dumbledore." He took a little box and opened it, holding up a fine golden chain with a bit of cloth pendant in the middle. "There's a sickle inside that I soaked in the blood vomited by a Shtriga. It protects you from Shtrigas permanently."

Another crate held African magical items, mostly fetishes, including a very beautiful monkey paw, and claws from various beasts of prey.

Snape took several of the articles over to one of the windows to study them more carefully in better light. He looked up to find Quirrell staring at him intently. "What?" Snape asked.

"I still find it hard to imagine that someone so close to the Blacks, and the Lestranges, and all those others was never interested in joining You-Know-Who. Especially one with your knowledge of the Dark Arts."

"We've had this discussion before. I have never been accused..."

"That's not true. I graduated the June before You-Know-Who fell. I remember there was a lot in \emph{The Prophet} about rounding up Death Eaters, and I know I saw your name."

"That was only because several of the students in my house had parents who were being arrested, and I defended the students. Naturally \emph{The Prophet} would jump to the wrong conclusions. You never saw any Ministry confirmation of the charge, did you?"

"No... How long have you been here, anyway?"

"I was hired at the beginning of the autumn term in 1981, and if I hadn't been squeaky clean, Dumbledore would never have taken me on."

Quirrell protested, "But how could you have continued when so many people you'd known, been friends with, were being hunted, arrested, imprisoned?"

It was beyond belief. It was as if the man knew he'd been a Death Eater, even though that information was still considered secret by the Department of Magical Law. Snape was furious, and now thoroughly convinced that any action he took against Quirrell was more than justified.

"Why," Snape demanded, "would the arrest of someone I'd gone to school with years before, someone I no longer had any connection with, induce me to abandon my position at Hogwarts? The illogic of the action aside, you have a pretty poor opinion of my sense of duty if you think I would run just because former colleagues of mine were being arrested. Excuse me, Quirrell. I have work to do in my own office."

Snape left. Halfway to the stairs he clenched his fists in anger and was rewarded with a stab of pain. Looking at his hands, he saw he was still carrying one of the claw fetishes. Returning to the Dark Arts office, Snape put his hand on the knob, but paused as he heard Quirrell inside, talking to himself.

"Yes, yes, he stayed. Stayed while everyone was being rounded up and sent to Azkaban. But does that really mean he was on the right side? Does it really mean I can trust him?"

Snape kept the fetish and went down to his office. More than ever he was convinced that he had to get rid of Quirrell. Not just for himself, but for the school. The man was a looney tune.

\emph{Thursday, September 6, 1990 (the day after the full moon)}

August turned out to be a much bigger trial than Snape could have imagined because Quirrell was always watching him. And there was nothing Snape could do about it. Action had to wait until the students returned. With three hundred people in the castle, and with spells coming at times when it couldn't possibly be the Weasley twins, and at other times when it couldn't possibly be Snape, Quirrell would crack. Snape knew Quirrell would crack, and then he'd be sacked and the Dark Arts job would be up for grabs again.

Snape made no contact with the Weasley twins until after the first second-year Slytherin-Gryffindor Potions lesson on the first Thursday morning of September. He signaled to them to wait until the rest had gone to lunch. When there was no one else in the dungeon corridors, he led them down a passage and into an empty storeroom.

"All right," Snape said when the door was closed and an illumination spell lit, "what've you come up with?"

George ticked things off on his fingers. "We've got dungbombs, and firecrackers, and..."

"I don't believe this!" Snape cried to the ceiling, then rounded on the boys. "I ask you for creativity, and you bring me Zonko's and Gambol and Japes. I might just as well ask Percy. He at least can follow instructions."

The sudden fire this brought to the twins' behavior was noted and logged for future reference.

"No, sir! No! We can do it!"

"Loads better than that prat Percy can!"

"Just give us a chance, sir!"

"Fine. Another chance. What do you have in the way of locomotor impediment spells?"

"Loco who, sir?" George looked at Fred, who shrugged.

"Things that get in the way of movement. Tripping spells are among them, but they're too crude. Things that make you feel you've accidentally hit your elbow against a door jamb, or got the hem of your robe caught on a twig. Or as if there was a flagstone sticking up a quarter of an inch right in front of your toe. Things that'll make the victim feel awkward and ungainly, and maybe not immediately suspect it was a spell at all."

The twins were speechless, as if a whole new and marvelous world had just opened in front of them. "You're wicked, sir," Fred breathed, and George nodded agreement.

"Who do you want us to use these spells against, sir?" George asked.

"Professor Quirrell," Snape replied.

"Yes!" Fred exulted, his fists pumping air. "Charlie said it! He said you fancied Quirrell's job! He said you'd be better at it, too!"

"I thank Charlie for his sterling support. Do not, however, breathe a word of this to him. There are other body parts I can affect besides toenails. Now, about these impeding spells..."

He showed the twins several, and then the Weasleys went first to lunch with Snape three minutes behind. There was nothing remotely unusual in their aggregate behavior except that Fred and George seemed more interested than usual in the teachers' table, where Snape was now approaching Quirrell.

Quirrell was reading a book and eating soup. Snape said a perfunctory, "Good afternoon," as he sat in the chair on Quirrell's right and reached for a piece of bread. Quirrell murmured, "'Noon," without taking his nose from his book, misjudged his distance, hit the side of the soup bowl with his spoon, and sent soup flowing across the table. Neither the book nor his robes were soiled in any way.

"Drat!" Quirrell exclaimed, rising quickly and grabbing a table napkin to stem the tide of soup. "Cleanup please," he called, and the house-elves below cleared the mess away. Quirrell brushed the front of his robes with the napkin and turned to Snape, who was regarding him with some concern. "Sorry about that Severus. Terribly clumsy of me."

"Not at all," Snape replied as the two calmly continued their meal. It was another minute before Snape glanced at the Gryffindor table, but there was no mistaking the admiration that gleamed from two identical pairs of eyes.

From that moment, Quirrell became a klutz. He caught his sleeve on chair arms and stepped on the hem of his robes going up the marble staircase. He hit his shoulder against the oaken front doors, snagged his books on the corners of desks, and fumbled with his quills, pointer, and wand. Sometimes it happened when Fred and George were in their common room or the library. Sometimes it happened when Snape was in Dumbledore's office or in conference with McGonagall. No one person could ever be said to have been present on each occasion, and indeed, Quirrell seemed not to suspect that it was spell induced.

That was the part Snape found hard to explain. Quirrell was nervous, more nervous than could be accounted for by the pranks, and he even seemed to feel that the awkwardness was caused by his nerves rather than the other way around.

The absolute truth of the matter was that Quirrell was getting downright twitchy, which only made him more irritating. He began rubbing his hands in an odd, twisting motion, and the slight hesitation in his speech that Snape noticed at the beginning of the school year was developing into a pronounced stammer. Snape had no direct personal experience of what was happening in Quirrell's classes, but the Slytherin students said he alternated between one moment telling them how wonderful and fascinating the Dark Arts were, and the next moment jumping, starting, and emitting strange squeaking sounds at the slightest noise.

It was Michael Bole, the latest in the line of Slytherin Beaters, who gave Snape the news that after Halloween Quirrell had started bringing the viper to class and placing its cage so that the snake could see what was going on.

"He fair talks to that blooming reptile, sir, like it was his mum or something," Bole was obviously disappointed in his Dark Arts teacher. Not that Bole wasn't disappointed in most of his teachers, Bole being far from academically inclined.

"What does he say to it?" Snape asked, intrigued.

"Weird stuff. Little things like 'yes, yes, of course' or 'it's harder than you think.' Then sometimes it's like he's talking about a person - 'he's suspicious whenever I bring it up,' and once he said, 'I don't think you can count on him anymore.' It was like he was planning something with that blooming snake."

"Thank you, Bole," said Snape and went straight to Dumbledore.

The interview went about as well as could be expected, which meant that from Snape's point of view it didn't go well at all.

"I understand your concerns, Severus, but the fact remains that Professor Quirrell is a highly qualified Dark Arts teacher. Both his OWL and his NEWT candidates did very well in his first year with us, and I expect them to do equally well or better this coming June. I do not think that the acquisition of a nervous tic or two is grounds for dismissing a teacher. You do know what would have happened if I had taken every complaint about a teacher's behavior seriously, don't you?" And Dumbledore peered at Snape over the rim of his glasses.

McGonagall was more on Snape's side. "I don't know where that boy went last year, or what he did, but he's gotten as jumpy as a cat in a room full of rocking chairs. It makes me twitch just watching him. Could you brew him something to make him calm down, Severus? Immobilize him, maybe?"

"Do you think it's affecting his ability to handle his classes?" Snape suggested, wondering if he could influence Dumbledore via McGonagall.

"I know it's affecting my digestion," McGonagall replied. "I marvel you can stand to sit next to him at dinner. Your appetite was never good at the best of times."

Shortly before the Christmas break, Quirrell brought the viper to dinner, setting the cage on the table in front of him.

"Whatever is that thing doing here?" Snape asked, not only irritated, but somehow no longer hungry with the cold reptile eyes regarding him unblinkingly.

"I j...just wanted to g...give it a change of sc...scenery," said Quirrell. "It seems more a...ctive and in...telligent than your average sn...ake."

Hagrid passed by them and noticed the newcomer. "That's Ursini's ain't it?" he said, stopping to watching the coiling and uncoiling viper. "Pretty little thing." He laid a huge hand on Snape's shoulder. "Now you don't go letting this fellow put ya off yer feed, Professor. Ain't no snake 'd ever take away my appetite, 'n if you let it get t' ya, lad, I'll come down this end 'n feed ya meself."

Others came to look as well, and to exchange the normal pleasantries and conversation. Pomfrey asked about restocking the clinic, Kettleburn wanted to know about the Slytherin Seeker and whether an injury in November's game would affect the one in January, McGonagall had a question about an order of supplies for the spring term, and even Dumbledore came over to look at the snake, which flicked its tongue and hissed at him.

Then it was Christmas, and most of the staff and students went home, including Quirrell. When they returned in January, the snake was gone. When Snape asked about it, Quirrell merely shrugged. "It was getting t...iresome. I s...old it."

Snape did notice, however, that Quirrell had developed a sudden interest in the Forbidden Forest, and could be seen at least twice a day, and sometimes more frequently, hovering about the fringes of the trees, or actually disappearing into the forest's shade for a half hour at a time. He had by now acquired a tic in the muscle next to his left eye that made it difficult to look at him for longer than thirty seconds at a time.

The Weasley twins were given permission to step up their campaign, and everywhere Quirrell went, things had a tendency to fall, or break, or slam. Quirrell's repertoire of little squeals and shrieks whenever this happened was quite amusing. Dumbledore still refused to consider a change of staff until after the results of the OWLs and NEWTs showed what kind of job Quirrell was doing with the students.

The key, Snape had learned during his days as an infiltrating spy, was to have your cover story ready in advance. It was thus that when Hagrid caught him sneaking into the Forbidden Forest on Quirrell's trail around mid February, Snape knew exactly what to tell him.

"Snowbells."

"Come again?"

"Alpine snowbells. They're beginning to stick their little heads out of the snow about now, and its the perfect time to gather and dry them for Frostbite Salve."

"Ain't never heard of 'em around here."

"Alpine snowbells and evergreen lichen. An unbeatable frostbite combination."

"Lichen ain't evergreen. Not in winter, anyways."

"Of course, it isn't the fungus I'm after. It's the cyanobacteria. That can only be isolated under laboratory conditions. But then, you know that."

Hagrid was one of those rare people whose attitude toward science was that if he didn't understand it, it must be true. He didn't argue the point. Instead he redirected it. "If ya got a minute, come inside and have a cuppa. I want t' ask ya about Quirrell."

It was impossible to tell if that boded good or ill, but as at that moment Snape couldn't think of a good reason why not, he accompanied Hagrid to the cabin. "What about Quirrell?" he asked once they were inside.

"He's fidgety. Ya got somewhat t' do with that?"

"Why ever would you think..." Snape started to protest, when Hagrid cut him off.

"First I got to ask meself what the head o' Slytherin house's got t' do with a pair o' redheaded rascals who'd curl up 'n die before they'd put on a green 'n silver badge, 'n then what the same two rascals was doing stalking the Dark Arts professor up t' the fourth floor, 'n then when Professor Trelawney was coming down, what coulda made her cards jump outta her hands 'n hit Professor Quirrell in the side of the head, when I seen exactly the same kinda trick performed in the entrance hall 'gainst Sirius Black by a scapegrace Slytherin some seven years ago."

Snape stared, opened his mouth, closed it again, narrowed his eyes, and said, "And the miracle is you got that all out without taking a breath. Are you accusing me of something?"

"You know," Hagrid said quietly, though there was no mistaking his ire, "you ain't the first runt I ever mollycoddled, 'n you ain't going t' be the last, 'n in between I get t' know a lot of the students. He weren't a bad lot, Quirrell, 'n I ain't going t' see him bullied just 'cause you ain't satisfied with the job you got."

Snape drew himself up in offended dignity. "You presume, Hagrid. I am offended." He turned and left, marching with straight back and squared shoulders up the hill to the castle.

Once there, however, Snape sought out the Weasley twins. "Excellent news," he told them. "Your mission has been accomplished. All we have to do now is sit back and watch the target deteriorate on his own. It's much more subtle that way."

The boys didn't want to give up, having, it seemed, enjoyed the escapade thoroughly. Snape had to resort to threats of grievously embarrassing curses to induce them to concur. That, and he gave them the promised toenail hex.

The fortunate part was that Snape turned out to be right. Quirrell was by now so jumpy and jittery that he became accident prone and a danger to be around. Hagrid accused Snape, Snape protested his innocence, Hagrid maintained surveillance, and eventually Hagrid was forced to admit that neither Snape nor the Weasleys were hexing Quirrell. Hagrid even apologized for his earlier suspicions.

In mid June, with exams over and the year coming to an end, Dumbledore called Snape into his office.

"I wanted to tell you first, Severus. Better straight from me than on the rumor mill."

Snape could guess. "The OWL and NEWT results were excellent and you've engaged Quirrell for next year."

"You always did have a quick mind."

"You know," Snape said bitterly, "this means I was right all along. If there ever was a curse, it's gone. The Dark Lord isn't coming back."

"I prefer to watch for a while longer," Dumbledore responded, "though even I must admit it is a reassuring piece of evidence. I did also wish to remind you to prepare yourself over the summer break."

"What for?"

"It is 1991. Eleven years. Next September, if all goes well, begins the wizard education of Lily's son Harry."

Snape stared at Dumbledore, pain forming behind his eyes. "What's that to me? One more Gryffindor brat among so many others."

"You promised to help me protect him."

"That was when it looked like he might need protection. Look around you, Headmaster. The Dark Lord hasn't been seen or heard from in nearly ten years. His followers are incarcerated or trying their best to forget they ever knew him. Every shred of evidence we had that he wasn't truly gone is being proven false. Bella didn't know anything, she was just crazy. The mark is nothing more than a brand designed to respond to certain syllables. If there ever was a curse on the Dark Arts position, it's gone. What is there to protect Potter Junior from?"

"Humor me. Pretend I might be right, and be ready to jump in if it turns out I am."

Snape paused, thought, and then said, "I'm willing to do that as long as you remember your promise."

"And that was?" Dumbledore cocked his head to one side.

"Never tell him. Never tell anyone. It's bad enough being reduced to nursemaiding James Potter's son, but to have it noised abroad..."

"I shall renew my vow, Severus. I shall never..."

"Thank you, sir," Snape said, and turned, suddenly overwhelmed by images he'd thought forgotten, and strode from the office.

"...reveal the best of you." Dumbledore finished, watching the disappearing back as it fled down the spiral staircase. The look in his eyes was one of tenderness and concern, though he had to be careful never to show it to Snape, who would have been mortified even knowing it existed.

Lancashire was not the haven it had once been. Freed from school and once again in his own home, in his own bed, Snape dreamt of emerald green eyes. Waking, he saw nothing but reminders. There was the bridge that separated business town from laboring town, the mill whence his father had brought back venom reserved for managerial scum like Lily's father, the local where Tobias Snape had proven himself, and by extension his family, no more than common lowlife. Worse, there was the tree on the opposite bank of the river from the mill where he and Lily had met so often in those now-magical years before they went to Hogwarts.

Life was a lesson in might have been, a long trail of 'if-onlies.' \emph{If only Mercury hadn't been retrograde, if only she'd been sorted into Slytherin, if only I hadn't fought back, if only Potter'd been content with the girls he had, if only there hadn't been an Invisibility Cloak... if only the Dark Lord had realized that the prophecy referred to... the other one, the auror's child.}

Restless, Snape walked past the house that had once been Lily's almost every time he left his own. He haunted the school yard, empty for the summer. He sat under the tree trying to remember every scrap of conversation...

And then, mercifully, he met Mrs. Hanson in the market, Mrs. Hanson who'd been visiting her sister in Manchester. He chatted about rheumatism and arthritis and the best way to cook asparagus. He carried packages, and offered his arm across the bridge, and remembered Dr. Who and the Avengers. He went to tea and learned all over again why the Conservative Party was the worst thing that had ever happened to Britain.

Thanks to Mrs. Hanson, Snape was feeling his usual self - not normal, but usual - as July trickled its way through the great glass of time. When July merged with August, Snape had even forgotten to remember its import in the cosmic scheme of things. He only remembered it was time to return again to Hogwarts.

\end{document}
