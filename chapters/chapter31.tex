% A Difference in the Family: The Snape Chronicles (Rannaro)
% The Middle Years (1982-83, Carmichael)

\chapter{Counter Charge}

\subsection{Monday, April 18, 1983}

On April 17, Snape apparated back to Hogwarts feeling more relaxed than when he'd left, though the sight of the great castle on the hill did cause some of that good feeling to evaporate. \emph{One more term to go, and I'll be free of this place for a month. I can hardly wait.}

Monday classes were as routine as they could be for students who'd just woken up to the fact that they had less than two months before their OWLs and their NEWTs. Snape announced that special tutoring sessions would be arranged in the Great Hall for those who wanted the extra help.

During his last class of the day, Snape received a message asking him to go to Dumbledore's office before supper. Gawain Robards had come up from London.

Snape met McGonagall on his way up the stairs. They were both clearly going in the same direction. When they walked together into Dumbledore's office, they found not only the Headmaster and Mr. Robards, but also Madam Pomfrey, Dr. Carmichael, and a plump little man with wispy brown hair and watery eyes who was introduced to them as Tristan Platt, Carmichael's publishing agent. From the moment Snape entered, Carmichael glared at him with undisguised loathing.

Dumbledore began, "Mr. Robards has just been explaining to us the nature of a certain condition called an allergy. He believes Dr. Carmichael may have one. His says you brought this to his attention, Severus. What have you to tell us?"

Surprised by the directness of the question, Snape hesitated, then considered there was nothing wrong about speaking in the present company since either they already knew of his muggle blood, or they had no reason to hold it against him.

"I went home for the Easter break, and while shopping one day I chanced to speak to a neighbor lady about the seafood for sale. She said she couldn't eat it because of an allergy, and described her reactions. They seemed identical to Dr. Carmichael's, and since no trace of poison was ever found, I thought the information opened up another line of inquiry for the Ministry."

Carmichael jumped in immediately. "You already knew about this allergy business, and you deliberately selected an undetectable poison that mimics the symptoms. Admit it, Death Eater. I'm a witch! I do not have muggle ailments! You're trying to get away with poisoning me!"

"Madam Pomfrey?" Robards said.

The nurse cleared her throat. "It is true that muggle-born witches and wizards do not generally suffer from the common diseases of the muggle world, so that we have a very low incidence of things such as measles, chicken pox, et cetera. But it is also true that from time to time a case will crop up. Even half-bloods and purebloods occasionally catch colds. My understanding is that this allergy is not a disease, however. It is something that originates in the body itself as a malfunctioning of the immune system. St. Mungo's has records of a few cases..."

"I am not allergic to oysters!" Carmichael shrieked. "I'm being poisoned!"

"There is a way to find out," said Snape. "Bring in a bowl of oysters, and we all eat some. Then we wait five hours. If they're poisoned we all get sick. If you have an allergy, only you get sick."

"I refuse to engage in a barbaric and humiliating test as if I were a guinea pig or something. The two of you are working together, from vindictiveness and a loyalty to the dark forces, and I'm going to let the whole world know."

"Then," said McGonagall very primly, "I shall sue you for slander."

"Wait a minute," said plump little Mr. Platt, his cheeks growing pale, "I'm sure we can arrive at a more amicable solution."

"I think not," said McGonagall. "If she slanders me, I shall sue."

"You've hated me ever since your son fell in love with me. From the moment I came here, you've been trying to thwart me, to hurt me. You even tried to stop me from being friendly with him just for spite." Carmichael gestured wildly at Snape.

"If I was trying to hurt you, why would I want you to stay away from someone who wanted to poison you? Wouldn't I have encouraged it? And Marcellus did not fall in love with you. You went out of your way to lure him."

"He saw me in Hogsmeade and came to my table."

"You waited for him in Hogsmeade and called him to your table."

"How would you know?"

"He told me. After it was all over, he told me. You waited until he was of age so that you couldn't be accused of misleading a minor, and then you pounced."

"Why would I do something like that? Answer me! Why?"

Light that had already been glimmering on the horizon suddenly dawned. "Your NEWT," Snape said. "You considered Professor McGonagall responsible for your failing the exam."

Dr. Carmichael stared at Snape for a long moment, then stormed to the door. Her hand on the latch, she snapped at Platt, "Come on! We're leaving!" The little man followed her down the stairs for all the world like a pet dog at heel.

The others stared at Snape. "What do you know about Dr. Carmichael's NEWTs?" Dumbledore asked quietly.

"She told me that she got low marks on an important one because of the incompetence of a new teacher. She didn't say, but I assumed she meant Professor McGonagall. But when I checked the archives I found she'd been disqualified in Transfiguration for cheating."

"Oh, really?" Robards said. "I think I should hear about this."

Snape briefly told them Dr. Carmichael's version of the NEWTs, how her chances for success with the Ministry had been destroyed by lack of proper preparation for the exams, how she had married her first husband and accidentally met Marcellus McGonagall in Hogsmeade during the period of the divorce. Professor McGonagall remained silent the whole time.

"Well, Minerva," said Dumbledore when Snape had finished. "What is the other version of this? The part Dr. Carmichael did not tell Severus."

McGonagall sat stiffly on Dumbledore's sofa, her hands folded in her lap. "Valeria," she said, "was not a good student at Transfiguration. The only subject I think she really excelled at was Dark Arts. I didna know her well, as I was there only during her last year, but she had that reputation with the instructors. Almost as soon as we arrived at Hogwarts, she took a fancy to Marcellus - he was in first year - and adopted him. She'd help him with his studies and his assignments - she was very motherly. She professed great surprise when she found out that I was his mother. Her favorite teacher and her favorite little kid, she said. It wasna until the end of the year that I found out why.

"Just before the examiners came, she asked me if I could give her a hint of what the questions and tasks would be. I told her that I'd given as much information as I could in the class, and I couldna give her any more. She said she thought we had a special relationship, and I said that what she asked went beyond what was proper. Then she came 'round again saying that she'd done me a favor, taking care of Marcellus, and it had cost her study time. It was my responsibility to make that up to her by helping her out with the questions. At about the same time Marcellus began acting up, and I found she was feeding him a story about how wicked I was.

"When the examiners came, she tried once more, and once more I turned her down. I went to the examiners and reported what had happened. One of them, d' ye remember Dr. Prudhoe, asked me to give her wrong information, but I refused. He then asked me to put a folder into my desk where only a student searching for information to cheat would find it. I did, and Valeria gave the planted answers on the test. She never even bothered to check if they were right or not. She accused me of telling her the false answers on purpose to make her fail, but the examiners bought none of it."

"Why did I know none of this, Minerva?" Dumbledore asked.

"Och, Albus! That was the June that Professor Dippet decided he was a canary bird and kept trying to fly. Ye were a wee bit busy."

"Ah, yes. I remember it now. But I do recall that Marcellus's dislike of your plans for him postdated Dr. Carmichael's time at Hogwarts."

"I fear ye may be right there, Albus. I did so want him to have a good career, but he's happier now. But that woman, when she came back, she was trying to lead him out of Hogwarts and out of a Ministry career to spite me."

"Are you sure of that?" Robards asked.

"Aye. Marcellus told me, after it was all over. It was something she harped on. That and her conquests. The way she could twist young men around her finger, lead them down the primrose path, and play with them like a fish on a hook. So when she started casting flies at our Severus here... Well, hasna he had enough troubles this past year?"

"And yet, Minerva, your interest may only have made matters worse."

McGonagall looked up at Dumbledore. "I see that now. I didna see it then."

"Who can witness to this besides you?" Robards asked.

"Marcellus can. Dr. Prudhoe has retired, but I'm sure he's still alive. Professor Tofty did the OWLs that year, but Dr. Marchbanks was administering NEWTs. 'T is all in the records."

"Excellent," said Robards. "I have a feeling this is one case we can solve to almost everyone's satisfaction."

"Mr. Moody's?" Snape asked.

"I said almost everyone's. Don't worry about Alastor. We can keep him in line. He's even mellowed a bit since last year."

"Funny," said Snape, "I hadn't noticed."

"That's because you don't have to live with him on a day-to-day basis. Believe me, he obsesses on you a lot less than he used to. When he finds he put his money on a losing horse, he'll back off from the pure shame of it."

"I suppose that should make me feel better."

"It should. I actually once witnessed Alastor admit that he'd been wrong. It was several years ago, and the circumstances were admittedly traumatic, but miracles do happen. Albus, I need to return to London now, but I'll keep in touch. Professor McGonagall, a pleasure as always. Madam Pomfrey. Professor Snape." Then Robards was gone, heading out to Hogsmeade where he could apparate back to the Ministry.

Two big contests were looming, Slytherin against Hufflepuff in Quidditch, and Snape against Carmichael in public relations. Carmichael struck first.

\emph{The Daily Prophet} article contained a subtitle - "Prejudice at Hogwarts?"

\emph{Reporter: And so, Dr. Carmichael, in spite of your fame, your credentials, you have found yourself the subject of anti-muggle-born prejudice even at an institution as venerable as Hogwarts School?}

\emph{Dr. Carmichael: It is so difficult to accept. Of all places where I thought the accidents of my birth and ancestry would not be held against me, Hogwarts rated the highest, especially under the guidance of Professor Dumbledore. But even the best of us may be swayed by clever subterfuge, and the still-loyal agents of You-Know-Who are working their insidious mischief even there.}

\emph{Reporter: Can you give us any particulars?}

\emph{Dr. Carmichael: Do you realize that I have been threatened with legal action, with public hounding in the courts, just for speaking the truth? And yet there is a teacher there, a teacher whose connections to You-Know-Who and his minions have already been documented, who is allowed to slander me with impunity and prejudice even the Department of Magical Law Enforcement against me. And I am not allowed to defend myself. He is joined by another professor, one who hopes to hide behind a mask of propriety, who has hated me ever since she began teaching because I had the temerity to fall in love, something her prudish narrow-mindedness could not accept. And this vindictive harridan and her Death Eater partner have marked me for death, hiding their machinations behind the facade that it can all be blamed on the impurity of my muggle blood.}

\emph{Reporter: That is monstrous, Dr. Carmichael. And you can't identify these miscreants for us?}

\emph{Dr. Carmichael: If I do, I shall be forced to face the full might and power of both Hogwarts and the Ministry on my own, without any other defense but my poor protestations of innocence. And we all know how much good that does. No, it is the great, fair-minded public of the wizarding world that is my only defense. If they can demand the resignation of these two villains, then justice will have been done.}

"Well?" McGonagall demanded, shoving the newspaper in Snape's face. "What are you going to do about it?"

Snape glanced through the interview, his face and mind closed. "It seems to me," he said, "that since you are the vindictive harridan, and I am only the Death Eater, you should have the honors."

"Don't get cheeky with me, boy. I can still deduct points from Slytherin."

"Not for a teacher, you can't." Snape turned to Flitwick. "She can't, can she?"

"Just because I've never seen it done doesn't mean it can't be," replied Flitwick, burying his face in his breakfast plate and fizzing slightly. Beyond him Sprout was also hiding her expression.

"I think," Snape said determinedly, "that we should file a grievance and have her summoned to respond to a libel suit."

"Since when did you become a legal expert?" McGonagall huffed. "And wouldn't that put our names on the front page?"

"All the better," replied Snape. "How many witches and wizards have you taught in the last, what, twenty-six years? They can't all have hated you. I didn't hate you. Do you realize that once you're identified as the teacher Carmichael is referring to, she may lose support rather than gain it?"

"Well..." said McGonagall.

"Let me at least contact Robards. Maybe he can talk to that publishing agent. They have to see reason at some point."

"All right. You go ahead."

\subsection{Friday, April 22, 1983}

The day before the Slytherin-Hufflepuff Quidditch match, Robards returned to Hogwarts. With him was plump little Tristan Platt. Dr. Carmichael was the last in, having followed Snape and McGonagall up the stairs.

"I have here," said Robards, "a summons issued in answer to a petition filed by Minerva McGonagall and Severus Snape asking that Valeria Messalina Carmichael, née Aurifosser, be called to answer for certain slanderous statements publicly made in an instrument of the media known as \emph{The Daily Prophet}. Before I serve this summons, I'd like to know if you, Dr. Carmichael, have anything to say that might stop this action."

"Serve away," said Carmichael.

Platt coughed slightly. "You know, Valeria, that might not be wise."

The temperature in the room dropped by about fifteen degrees. "Are you implying, Tristan, that I am not capable of defending myself in this matter?"

"But, Valeria dear... Will you excuse us for a moment?" Mr. Platt asked Dumbledore, and when the headmaster nodded, he led Carmichael to one side and whispered in her ear. Carmichael went livid, muttered something about finding out what they wanted, and left the office.

Platt was apologetic. "I am sorry. She gets terribly single-minded sometimes. I think we can agree, however, that this is best settled amicably, with a minimum of publicity."

"I thought she wanted publicity," said McGonagall. "I thought she was calling on the power of public opinion."

"Dear Valeria believes in the basic goodness of the public, but doesn't always realize that once awakened it isn't easy to control. Things could get out of hand, emotions being what they are, and we don't necessarily want that."

"So she'll retract her words and stop attacking us in the news?"

"Well, Professor McGonagall, it might be hard to accomplish the retracting part, especially since she never mentioned anyone by name. I hope the stop attacking part is easier. I think she can be brought around to that."

"What do you think?' McGonagall asked Snape.

"It's true, no names were mentioned. I doubt that it's occurred to anyone yet that she was even talking about you. The description didn't fit. If it's never mentioned again, it'll die away and little harm done."

"Very well." McGonagall turned back to Platt. "We'll be content if she never refers to it again."

"Thank you, dear lady, gracious gentleman! You won't regret this generosity." And then Platt was gone, too, leaving the others puzzled.

"I wonder what they know that we don't," said Robards, "that makes him so anxious to avoid a libel suit."

The next day was the last Saturday in April, and the Quidditch match between Slytherin and Hufflepuff. Snape sat with the Slytherin Quidditch team at breakfast that morning and discussed overall strategy. Or rather, he listened while they discussed, adding only an occasional comment.

"It's Gryffindor in the best position," said Lionel, the Seeker. "They scored enough against Hufflepuff in their last game that even if they score nothing before catching the Snitch, we'll need 40 Quaffle points before the Snitch just to tie them. And Hufflepuff needs a hundred Quaffle points for a tie."

"Which means," said the seventh-year Chaser Rhonda, "that Ravenclaw has to beat Gryffindor. Ravenclaw is weakest in points and can't hope to beat either us or Hufflepuff. But they can come in second with a simple Snitch victory. So if we or Hufflepuff score high, Gryffindor will be looking for Quaffle points while Ravenclaw will be Snitch hunting. But if we both score low, Gryffindor will be Snitch hunting, too."

"So it's a Quaffle game, and pray Ravenclaw finds the Snitch in May."

Hufflepuff had the same strategy, and the Quidditch game quickly became a battle around the hoops. Both teams excelled at defense, and the score stayed low. After an hour of play, Slytherin had thirty and Hufflepuff forty.

It was at that point that the Gryffindor stands began to heckle the Hufflepuff players.

"What's the matter, Badger-boys? Scared of the Snitch?"

"Hufflepuff, not fast enough! Hufflepuff, not fast enough!"

"Snitch! Snitch! Snitch!"

Flitwick, sitting between Snape and McGonagall, looked at Gryffindor's head of house. "What are they doing that for?"

Snape leaned over, his voice raised against the noise. "If either Slytherin or Hufflepuff gets the Snitch now, Gryffindor won't need to score before it goes for the Snitch. They could win outright in minutes. But if we score one more goal and then win, Gryffindor will have to play the Quaffle to get first place. They want Hufflepuff to be content with second place and catch the Snitch before we score again."

But it seemed that neither Hufflepuff nor Slytherin was to be controlled by Gryffindor. Both teams kept doggedly to their game plan, and the score stood 60 for Hufflepuff when Slytherin scored its fifth goal and began to hunt the Snitch.

No longer willing to chance it, Hufflepuff placed its hopes in Ravenclaw and went Snitch hunting, too. The gods favored Hufflepuff, and after furious feints by both Seekers, Hufflepuff claimed the Snitch and won. Slytherin would have no Quidditch Cup that year.

Slytherin faced defeat again in the media competition on the following Monday. \emph{The Daily Prophet} ran another interview.

\emph{Reporter: But Dr. Carmichael, you were so strong last week. You came out a fighter.}

\emph{Dr. Carmichael: Even the strongest of us have our limits. I'm facing the full weight of Hogwarts and the Ministry. I can't fight anymore.}

\emph{Reporter: How can you let this happen?}

\emph{Dr. Carmichael: They're hitting me from several sides at once. First, since You-Know-Who targeted me for death, I haven't been able to do as much research and writing. So now they're threatening my publisher with exorbitant legal fees in a civil suit whose primary purpose is to drain my funds. Moreover, they're trying to blacken my character by taking a misunderstanding with my seventh year Transfiguration teacher and expanding it into a character issue. Then there's the other one, who everyone knows used to work for You-Know-Who, but they won't protect me from him. I'm so beleaguered, I don't know where to turn anymore.}

\emph{Reporter: Is there anything our readers can do?}

\emph{Dr. Carmichael: Yes, though I hesitate to say it for fear of reprisals.}

\emph{Reporter: Please, Dr. Carmichael, confide in your supporters.}

\emph{Dr. Carmichael: If they could just let Headmaster Dumbledore and the Ministry know that I'm not alone, that if something happens to me, it will be noticed in the wizarding world, and that evil deeds will not go unpunished. Then I think I could find the strength to go on.}

McGonagall slapped the paper in front of Snape. "Look at that!" she snapped.

"Do I have to? I'm eating. You know if you destroy my appetite now, I'm lost for the rest of the day."

"Even if you don't, others will. Albus has gotten three owls already this morning. If parents start calling for resignations again, we'll have to take it to court."

"I notice," Snape said, glancing at the interview, "that she still hasn't mentioned any names."

"I presume that will be her defense, that we can't actually prove that she was talking about us."

"And she has an eye to public opinion." Snape looked at the paper more carefully. "She hasn't said that the Transfiguration teacher and the harridan are the same person. Maybe she felt that not everyone would be on her side if they knew it was you she was complaining about."

"Well," McGonagall said, "there is some comfort in that."

Snape seldom even saw Dr. Carmichael at the school anymore, since she commuted daily to London and no longer took meals with the rest of the staff. He continued to monitor the progress of his house in Dark Arts, but everything there seemed to be going well.

Dumbledore came to Snape's office Wednesday afternoon. He was carrying a large packet of letters. "Most of them assume it is you," he told Snape. "They are more reluctant to name Professor McGonagall. Do you intend to respond to her attack."

"I wish she would just let it drop. The Ministry knows no one tried to poison her, even if she won't believe it. Everyone else will forget about it in a few weeks. But if it goes to court, it'll be all over Britain for months."

It didn't stop. Dr. Carmichael arranged a series of talks about her books, and the subject of her near-escapes was brought up during the first two by her audiences. The incidents made it into articles in \emph{The Daily Prophet}. On Monday, May 16, Minerva McGonagall and Severus Snape filed a request for an injunction against Valeria Carmichael and submitted papers to start a civil suit.

The first hearing on the injunction was set for the following Monday, May 23. Dumbledore accompanied Snape and McGonagall to the Ministry, where Judge Bones had taken the case. Both Robards and Moody were in the courtroom, and a few minutes later McGonagall was joined by a tall man with gray eyes and dark brown hair, probably in his late thirties.

The introduction was whispered and brief. "Severus, I'd like you to meet my son, Marcellus." The two men shook hands, and then waited quietly.

Dr. Carmichael's entrance was more showy. She was preceded by a photographer, accompanied by her agent, who looked decidedly uncomfortable, and followed by two reporters.

Robards stopped them before the little procession had gotten halfway into the court. "You shouldn't have the press here. This is just a hearing to review the facts of the case."

"There!" Carmichael announce to the reporters. "They don't want you to know the facts!"

Robards sighed. "We'll let the judge decide." He sent word to Judge Bones that all were assembled.

Judge Bones went first to the bench, then bade them all be seated. She glanced around, then crooked a finger at the clerk. "There are people here unconnected with the case," she said.

"Yes, your Honor. The Defendant brought them."

The judge looked through her papers. "Valeria Carmichael, step forward." Carmichael approached the bench. "For what purpose, Dr. Carmichael, have you brought representatives of the media to this hearing?"

"I understand the action has to do with remarks I made in \emph{The Daily Prophet}. This lady and gentleman are from that paper, and are here as witnesses for me."

"I see. And the photographer?"

"To take pictures."

"The photographer will wait outside the courtroom. And you will take no pictures without obtaining prior consent. I see any pictures of this proceeding, and you go to jail for contempt."

"Yes, your Honor," said the photographer, and left.

"Now, Gawain," the judge continued, "do you speak for the plaintiff, the defendant, or the court?"

"For the court, your Honor."

"This is for an injunction of cease and desist, and so far no damages are involved. What is the material in question?"

"Four interviews in \emph{The Daily Prophet}," Robards laid the papers in front of the judge, "and statements made at two book-reading events."

Judge Bones scanned the evidence. "No names are mentioned. Why do you think the public will connect the statements to you?"

McGonagall answered first. "She speaks of a female teacher who started at Hogwarts while she was a student. That can only be me. She confirms it later by mentioning a problem with her seventh-year professor in Transfiguration. That again can only be me."

"Thank you. Professor Snape?"

Snape didn't look at Carmichael. "She talks of a male professor who's known to have links to You-Know-Who. I'm the first male professor to be hired in fifteen years, and there have been previous questions about my connections to the Death Eaters. There is no other professor that it could be."

Handing a paper to Robards, the judge said, "There has been a previous complaint about poisoning and attempted murder. What were the results of the investigation?"

"Your Honor, no trace of poison was found either in the glass, the bottle that supplied its contents, any other bottle or glass, or in the expelled body fluids."

"That's because it was an undetectable poison," interrupted Carmichael.

The judge peered over her glasses at Carmichael. "Who is our foremost authority on undetectable poisons?" she asked.

"I am."

"Where did you study these poisons?"

"Mostly in the jungles of South America... Brazil, Venezuela."

"Can you identify for the court a poison that causes these symptoms?"

"No, but..."

"Professor Snape, have you ever been outside Britain?"

"No, your Honor."

"How do you acquire poisonous material for your classes?"

"I order it through school channels."

"Has an inventory been made of your supplies?"

"Yes, your Honor."

"Alastor Moody," the judge smiled at the auror, "was Professor Snape anywhere, during the time before the alleged attacks, where he might have purchased suspect materials?"

"No, your Honor," Moody was forced to admit.

"Has anything else surfaced in your investigation?" Judge Bones asked Robards.

"The defendant's symptoms match certain symptoms experienced by muggles who have a condition called an allergy to certain types of food. We have asked the defendant to cooperate in tests to determine whether or not she is suffering from such an allergy. She has refused."

"Why would you refuse?" the judge asked Carmichael.

"I don't have an allergy," Carmichael replied.

"How do you know?"

"I would know if I had an allergy."

"What is this allegedly an allergy to?"

"Oysters, your Honor," Robards responded.

"Dr. Carmichael, how many times in your life have you eaten oysters?"

Carmichael hesitated. "Twice," she said at last.

"Were each of these times also a time when you claim to have been poisoned?"

"Yes, your Honor."

"Have you ever been poisoned on a day when you haven't eaten oysters or eaten oysters on a day when you haven't been poisoned?"

Carmichael jumped on the question. "Yes!" she exclaimed. "When You-Know-Who tried to poison me the first times."

From behind Carmichael came what sounded like a small moan. It seemed to come from Mr. Tristan Platt. The judge turned gently to the self-effacing little man. "Mr. Platt," she said, "are you the agent mentioned by Dr. Carmichael in her books who helped save her from the attacks by Lord Voldemort?"

"Yes, your Honor."

"Mr. Platt. You are aware, of course, that testimony in front of a judge is under compulsion of perjury?"

"Yes, your Honor." Behind Mr. Platt, Carmichael had grown suddenly pale.

"Please describe Dr. Carmichael's symptoms after she was poisoned by Lord Voldemort."

There was a long period of silence.

"Mr. Platt, do you understand my instructions?"

"Be quiet!" snapped Carmichael to Platt. Then she turned to the judge. "My agent isn't here as a witness. I just asked him to come for moral support."

"He nevertheless appears to have been the witness of something," the judge replied, "and I shall decide whether or not it is relevant. Now, Mr. Platt, will you describe the symptoms?"

"I can't," Platt replied.

"Why not? Didn't you see them?"

"Uh, no. I... uh... didn't."

"Who did?"

Carmichael hissed, Platt sighed, and Judge Bones drummed her fingers on her desk. "I am waiting, Mr. Platt. We do have one or two nice cells where you would have leisure time to consider your answer if you feel under too much pressure here and now."

"You wouldn't!" Carmichael exclaimed. "You couldn't!"

"I would, and I could, and you will be silent. Mr. Platt, who witnessed the symptoms?"

Platt glanced woefully at Carmichael, but his choices were few. "No one did," he answered. "There were no symptoms. The poisoning by You-Know-Who never took place. It was a publicity stunt to cover her move to America and was made because of slumping book sales."

"You're sacked as of this moment!" Carmichael shrieked, but no one was paying attention to her anymore.

The two reporters looked at each other, and one rose and headed for the door. "Where are you going, Madame?" Judge Bones said, and the reporter stopped.

"I need to send an owl," she answered.

"No you don't, not about business that's before my court." The reporter sat down again, and the judge turned her attention back to Carmichael. "Dr. Carmichael, you have alleged that someone has been trying to poison you, and you have accused Professor Snape in this attempt both formally to the Ministry of Magic and by implication to the media, claiming as his motive that he was trying to fulfill Lord Voldemort's sentence of death against you, thus leading the public to the supposition that he was at some time a Death Eater. We now find that the designs against your life were nonexistent, that Voldemort never marked you for death. How could Professor Snape be trying to carry out an order that was never given?"

"He was really acting out of a spirit of revenge, since I had rebuffed his romantic advances. I didn't want to embarrass him by telling everyone that he was turned down by someone of my experience."

Judge Bones peered at Carmichael over her glasses. "You felt that being pilloried as a former Death Eater was somehow preferable to being ridiculed for being attracted to older women? Do you have a shred of evidence that Professor Snape was ever a Death Eater?"

Carmichael glanced back at Moody, but Moody was very subtly shaking his head. Looking back at the judge, Carmichael replied, "No, your Honor."

"And what of your allegations against Professor McGonagall?"

"She has resented me for years because her son and I fell in love, and he was ready to leave Hogwarts and his family for my sake."

The judge's gaze swept the front bench. "It would appear, Mr. McGonagall, that your testimony has suddenly become pertinent." Marcellus rose, and as he did Carmichael gasped. She had clearly not recognized him after the passage of twenty years.

"What do you need from me, your Honor?" Marcellus asked.

"Simple narrative would suffice, I think."

"I was already in my first year at Hogwarts when my mother was hired at the end of the autumn term to take over the Transfiguration classes. Almost at once, Mrs. Carmichael - she was Miss Aurifosser then - began to single me out for treats, help with my assignments, general attention. I thought this was great, and she told me it was because I was so smart, good-looking, and personable. The only time I ever heard anything unpleasant from her was at the end of the year, in June, when she seemed angry and told me my mother was an ingrate. I didn't find out the truth until more than six years later."

"Under what circumstances did you discover this `truth'?"

"In my seventh year, after I'd turned eighteen in fact, I encountered Mrs. Carmichael in Hogsmeade. One thing led to another, and we became involved. So involved that I left Hogwarts to be with her all the time. Our relationship turned sour not long after that, and in the course of the breakup she threw several things in my face. One of them was that I was an obnoxious little brat and the only reason she'd befriended me was to get the NEWT exam questions from my mother. When mother refused, she decided to get even by ruining my chances the same way hers had been ruined."

"How much of this do you know from your own experience, Professor McGonagall?"

"This is outrageous!" Carmichael shrieked.

"And you will be silent or I shall call a guard."

"Well," said McGonagall, "she did ask me for the questions, and I reported it to the examiners. They investigated and disqualified her."

"What have you to say in this matter?" Judge Bones asked Carmichael.

"They're lying! They're both lying!"

Snape diffidently raised his hand.

"Yes, Professor Snape?"

"Wouldn't it be in the school archives?" Snape asked innocently. When the judge nodded, he added, "I wouldn't want to speak for Professor McGonagall, but I for one would be willing to drop the petition for an injunction." McGonagall stared at him, then shrugged and agreed.

"You're a fast learner," said the judge. "The case is dropped," she announced, "and you are all free to go. Since there is no longer a case before this court, no restrictions can be imposed."

The two reporters exchanged glances again, and raced for the door.

It hit \emph{The Daily Prophet} headlines the next day - `Author Hoodwinks Public!'

"Look at this," McGonagall said, tossing the paper in front of Snape at breakfast. "I guarantee it will improve your appetite."

Snape skimmed down the page. It was primarily an account of the publishing agent's admission that the You-Know-Who poisoning claim was false, together with commentary about the impact of this on Carmichael's more recent allegations. Of particular interest was the section that spoke of Carmichael's `less than admirable school career' and her confession to the judge that she had no solid evidence about any former Death Eaters at Hogwarts.

"Well," Snape said, "that certainly seems to cover everything." McGonagall was right. His appetite did improve.

The rest of the day was remarkably pleasant. All of Snape's students were well-behaved, even studious, and the Slytherin students positively glowed. Many of them congratulated him throughout the day, and copies of \emph{The Prophet} were being read quite prominently at the Slytherin table during lunch and dinner.

Dr. Carmichael did not return to Hogwarts. Luckily, the term and the year were nearly over - she had covered her entire curriculum and was reviewing in all her classes. The students appeared very well prepared for their exams.

"In fact," Dumbledore confided to Snape and McGonagall, "were it not for the little personal problems that surfaced, she was quite the best Dark Arts professor we have had in many years. In that respect I am sorry to see her go."

"I'm not," said Snape, and McGonagall sided with him.

\subsection{Friday, May 27, 1983}

On Friday, the last Friday in May, there was another interview in \emph{The Daily Prophet}. Carmichael was still fighting.

\emph{Reporter: How do you explain that your own publishing agent stated in court that the so-called attempts by You-Know-Who to kill you were fabrications?}

\emph{Carmichael: They were not fabrications. My life was in danger before we released the story about the poison. Mr. Platt, who is no longer my agent, never denied that. He merely commented on the poisoning story that we used as a cover to conceal the true information we had about other attempts. And there are Death Eaters running around free today who know this to be true.}

\emph{Reporter: You implied that a teacher at Hogwarts was one of these Death Eaters, and yet you said in court that you had no evidence to support that.}

\emph{Carmichael: My information came from the Ministry itself, from the aurors. But they've been instructed not to back me up.}

\emph{Reporter: Why might that be, since the aurors more than anyone want to incarcerate as many of `his' former servants as possible?}

\emph{Carmichael: Because the Ministry wants to discredit me. They are out to get me, too.}

\emph{Reporter: To the extent of allowing Death Eaters to go free just to embarrass you? Come now, Dr. Carmichael. Isn't that a little far-fetched? Isn't it more likely that you have a common muggle condition called an allergy...}

\emph{Carmichael: There! You're working for them! You're part of the plot to discredit me! Which master do you serve? The dark one, or the Ministry?}

\emph{Reporter: It seems, Dr. Carmichael, that you have a habit of throwing these accusations at anyone who contradicts you. Your readers might start to think that your earlier accusations are just as baseless as your present ones.}

"Just as baseless," said McGonagall, allowing Snape to read over her shoulder. "That has a good sound to it."

"A most excellent sound. Now maybe all of this will go away."

"Don't hold your breath, Severus. Nothing ever completely goes away."

McGonagall's words came true at the Quidditch match between Gryffindor and Ravenclaw the next day. Once again the visitor stands were packed, Weasleys prominent in the crowd. In fact, there were more people than in November, and it was not just the good weather.

The outcome of this match affected all the houses. If Gryffindor won, Hufflepuff would be second, Slytherin third, and Ravenclaw once again on the bottom. If Gryffindor scored before Ravenclaw won, it would be Hufflepuff, Ravenclaw, Gryffindor, and Slytherin. If Ravenclaw caught the Snitch before Gryffindor scored, Gryffindor would be last.

It was not often that all three of the other houses were united against Gryffindor, and the mood in the stands and on the pitch was raucous, to say the least. Even reporters from \emph{The Prophet} were there, though as it turned out, they were more interested in McGonagall and Snape than in the game.

"Excuse me," a reporter said as Snape crossed the field on his way to the teachers' stands. "Could I trouble you with a few questions?"

At a loss for what to do or say, Snape took the coward's way out and postponed the encounter. "I'm sorry, I have to be on the other side in two minutes."

"Perhaps after the game?"

"Perhaps."

In the teachers' stand there was another awkward moment, for while it was a given that McGonagall and Flitwick would sit farthest apart, it was not clear who would sit next to McGonagall, all three of the others being in the embarrassing situation of hoping her house would lose.

"I think I'm strongest for Flitwick," said Sprout, "seeing that if he wins the game, I win the Cup."

So Snape sat next to McGonagall. That having been decided, he mentioned the reporter. "I haven't a clue what to say to him," he admitted.

"Simple," McGonagall snorted. "Tell him it's about time that horrid woman got what she deserves. And you no longer trust \emph{The Daily Prophet} for believing her in the first place."

"I don't know," Sprout interjected. "You don't want to make enemies, you know. If you're nasty to the newspaper, they'll be looking for another chance to strike at you. If you're nice and pleasant, they'll leave you alone."

"I doubt that," said McGonagall.

"Did I mention boring?" added Sprout. "If you're boring, a newspaper always leaves you alone."

"You do make a valid point," Snape said. "How do I manage being both polite and boring?"

Flitwick laughed. "Usually the two go together anyway. Just be polite and let boring take care of itself."

Don't make enemies. It was an old lesson, one of the first his mother had taught him, but Snape had somehow always managed to annoy someone. \emph{It's just that I always seem to say or do the wrong thing.} In any case, he had the whole Quidditch game to think about it. \emph{Assuming they don't catch the Snitch in the first minute, that is.}

They didn't catch the Snitch in the first minute, or the first five minutes, or the first fifteen minutes. By then Ravenclaw's strategy was clear. Gryffindor wanted to score twice before Snitch hunting, so if Ravenclaw could keep the Quaffle away from the hoops, they could delay a Gryffindor bid for the prize. Since no amount of scoring could help Ravenclaw, they didn't even worry about it, dedicating all their players except the Seeker and one Beater to the task of keeping the Quaffle out of Gryffindor hands and protecting their own hoops. If Ravenclaw never scored, Gryffindor would never be given the Quaffle.

It turned into a running comedy for everyone except McGonagall and the Gryffindor stands, both students and visitors. For the first time, Ravenclaw revealed a stunning series of lateral passes that they'd been practicing in secret. Every time a Gryffindor Chaser came close to seizing the Quaffle, it was gone. The red and gold stands were chanting `Fly high, Gryffindor!' but the answering chant from the other houses was `Fly blind, Gryffindor!' Tempers were getting short in one quarter of the field.

Snape had a good half hour to think of what to say to the reporter that was polite and dull. Then the Snitch appeared, and Ravenclaw went into action. Two Ravenclaw Chasers stayed just outside their scoring area to keep the Quaffle busy and deflect scoring attempts by Gryffindor while the third Chaser and the second Beater moved in to distract Gryffindor's Seeker. Two minutes later, Ravenclaw had the Snitch.

Hufflepuff went wild at their first undisputed Cup in years. Ravenclaw exploded at the first time in two decades that they were one of the top two. Slytherin was just glad they weren't last. The only house that nursed unmitigated disappointment was proud Gryffindor. Vows of vengeance the next year were already circulating in their stands.

"Professor Snape!" the reporter called, falling in next to Snape as he left the field. "What do you think about the recent developments in the Carmichael business?"

"We're sorry she's leaving us. She was an excellent Dark Arts teacher."

"But what about her poisoning charges?"

"I sympathize with her. It must be frightening to be suddenly so ill for no apparent reason."

"What is your opinion of the boycott of her books that some readers are threatening?"

"I think it's unwise. She is still one of the foremost authorities on the Dark Arts in the wizarding world today. I have read her books; they are excellent. Now, if you will excuse me."

"Thank you, Professor Snape," the reporter sighed, and went off in search of more promising game.

Then, as if without warning, the exams were on them. One day everyone was studying and reviewing, and the next the examiners, first for the OWLs and then for the NEWTs, had taken over the Great Hall, while the lower years were diligently writing foot after foot of parchment in their classrooms.

No more was heard from Dr. Carmichael, at least not during that term, and as June itself reached its peak and began sliding towards its end, the students started packing to return home for the summer. Snape visited the common room to say goodbye to the seventh years.

"Don't mind me," he said as the students in the common room rose at his entrance. "Just keep on talking." Snape crossed over to the fireplace where the two Chasers, Lionel and David, were mourning the loss of Rhonda. Settling into a chair near them, he asked Rhonda, "Have you anointed a successor yet?"

She laughed. "Sergey would be furious if I did. He takes his job seriously."

"Of course I do, my dear," chimed in Sergey from a nearby conversation. "How else are we going to clobber Gryffindor next year?"

"This time next year," Rhonda said dramatically, "everyone in the wizarding world will be wearing a Rhonda Cordonnier designed robe."

"Who's Rhonda Cordonnier?" David asked.

"May I hazard a guess that it's French for Shoemaker," said Snape.

"Naturally," said Rhonda. "Who'd pay top galleon for a robe designed by someone named Shoemaker? Cordonnier just sounds better."

"It must be nice knowing what you're going to do after you leave here," sighed David. "I haven't worked it out yet."

Lionel swatted him. "You've only just finished fourth year. Nobody knows in fourth year. Me now, I know. I've finally worked it out and started making contacts. Well, my father has anyway. I'm planning on working for Golden Cauldron. They need travelers."

"What's that?" David asked.

"They provide exotic potions ingredients, mostly to the apothecaries..."

"And to Hogwarts," Snape added.

"Right, a lot to Hogwarts. They need people to travel to places like Papua New Guinea or Suriname to inspect shipments and keep in touch with suppliers. It should be fun. All I have to do is get good marks in Herbology and Potions on my NEWTs."

After a few minutes, Snape left the common room, having said something to each of the departing seventh years. The experience on the whole left him terribly depressed.

\emph{They all know where they're going and what they're going to be doing. Their parents have contacts with wizard companies, and they've planned their studies to match. At the end of my seventh year, I didn't know that Golden Cauldron even existed. All I knew was that I was going to join the Dark Lord. I don't even remember making the decision. Even if Dad and Mum were still alive and there were no Dark Lord, I probably wouldn't have been looking further than coal mines and factories. And now that I know about other possibilities, I'm forbidden to try for them.}

Dumbledore sent for him, and Snape climbed the long staircases up to the seventh floor. He was greeted with a warm smile and a goblet of mead.

"I just wanted to tell you," Dumbledore said, "that the examiners were very impressed with your students' performances on the OWLs and NEWTs. A high number of Outstandings and Exceeds Expectations. Nothing lower than an Acceptable on the NEWTs. That's two years in a row that the scores have been impressive. Congratulations."

"Thank you, sir," said Snape, his heart not quite in the words.

"You do not sound pleased. Here I am telling you that you are an excellent teacher with a good future here at Hogwarts, and you act as if I was about to sack you."

"It's nothing, sir. I've just come from the common room, and the seventh years are talking about the jobs they have lined up, and... I suppose we just have to play the cards we're dealt, but sometimes..."

"I understand. Our lives are shaped by accidents of birth and by events outside our control. Some are born rich, others poor, some into a time of openness and freedom, others into a time of fettered choices." Dumbledore placed a hand on Snape's shoulder. Snape looked away, not meeting Dumbledore's eyes. "It is not good to be bitter, Severus. I know that happiness here at Hogwarts may be asking too much, but do you not think you might be content?"

"I'll try, Headmaster," Snape said, and returned to his rooms.

\subsection{Saturday, June 25, 1983 (the full moon)}

The students entered and left the Great Hall for breakfast that morning like migrating flocks of different species of birds, swooping through in tightly huddled groups to rest and feed, and totally ignoring the other flocks. Within each group they laughed, hugged, exchanged small tokens of friendship and even promised to write. There was almost no intergroup contact, for this was the last day they would be together until the next September, and there were both priorities and proprieties to observe.

The oldest students apparated home. Some parents arrived to apparate side-by-side with their children. Snape joined the other teachers in escorting the rest to the thestral carriages, after which they became the responsibility of Hagrid and the Hogsmeade station attendants. Snape watched them down the hill, then went to his own rooms to pack.

There wasn't much. There was never much. Snape generally told himself that it was because there wasn't any room, but he had occasional lucid moments when he opened those mental doors usually closed even to himself and acknowledged that he was afraid of spending the money.

Books were permitted, for they fed the mind. The theater was permitted for the same reason, and dinner of course was part of the theater experience. Beyond that and food when he was at home, it was a long time since Snape had bought anything for himself. Certainly nothing like clothing, household items, or things to personalize his rooms.

That was something else to think kindly of Dr. Carmichael for. She hadn't pinched pennies, but had graciously picked up the tab when Snape confessed his relative poverty. And since December he'd bought nothing at all, for he hadn't left Hogwarts.

It occurred to Snape that with his pay deposited in Gringotts, and his long term muggle account earning interest, he might be substantially better off financially than he'd been in August. It was a sudden, unexpected, pleasant feeling.

It was in this vulnerable frame of mind that he uncovered the lararium that had remained unnoticed since the investigative team had pushed it into the nook by the fireplace during their search in November. It was in this vulnerable frame of mind that he came face to face with the fact that he had remembered neither his parents' deaths nor Lily's more recent death all that year.

Holding the picture of his mum and dad at Blackpool in one hand, and the framed note from Lily in the other, Snape sank into the chair behind his desk. \emph{What a disgustingly self-indulgent person I am! Here are the three most influential people in my life, and I couldn't spare a moment for their memories? And that was before the first `attack' on Halloween. Even before the trouble, I stopped thinking about them. What a toad! What an absolute toad!}

There was nothing he could do. It was like remembering that you had an appointment five hours after you'd missed it. You might be able to apologize and pave the way for the future, but you could never go back and remove the original damage. Three of the most important persons in his life, and Snape had forgotten them, not just for a day, but for months.

\emph{You are such a basket case, Severus. It's not bad enough you have a blighted past and a nonexistent future, you have to mess up the bit in between, too.}

At lunch the teachers wished each other a pleasant holiday. McGonagall especially cornered Snape and thanked him for the little discoveries that had finally exploded Carmichael's case against them. "It was only a matter of deciding which lady I would prefer to spend time with over the next few years," said Snape. "After that it was easy." McGonagall's mouth got all prim and pursed as she swatted him with her wand, but both of them understood that it was only because there was nothing else she could say.

After lunch, Snape got his Gladstone bag with his muggle clothes - he left the wizard robes in the wardrobe for the next school year - and a small bundle of books that included the Shakespeare he'd bought the year before, and apparated to the moors of the Pendle district in eastern Lancashire. He had five weeks of freedom, and he needed to make the most of it.

The first person Snape met after arriving home, stashing his things, and going out for a walk and some fresh air was, of course, Mrs. Hanson.

"Russ, child! Are you back for the summer, then?"

"I am, Mrs. Hanson, and glad to be here. How have you been these several weeks?"

"Love ya, dearie, what with the arthritis and the back and the knees, not to say, mind you, that it might not all be the same problem, I've been a bit under the weather, and one of them colds to keep me company, too. But I ought not complain. The alternative is worse, if you get my meaning. How is your friend, that Mr. Robards I think it was?"

"He is well, and I'll tell him you inquired. You've no idea, in fact, how pleased I am to see you, for your information helped us solve a most baffling medical mystery. I was going to pop round and tell you, but since we're here, may I invite you to tea at Mrs. Lewes's shop? It's a most interesting story."

"A medical mystery? Do tell! I am all agog, Russ Snape, all agog. And you will tell me all the particulars?" Together the two went to Mrs. Lewes's for tea.


