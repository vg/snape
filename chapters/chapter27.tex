\documentclass[a4paper,11pt]{article}
\usepackage{ulem}
\usepackage{a4wide}
\usepackage[dvipsnames,svgnames]{xcolor}
\usepackage[pdftex]{graphicx}

\usepackage[utf8]{inputenc}\title{A Difference in the Family: The Snape Chronicles}
\usepackage{hyperref}
% commands generated by html2latex


\begin{document}

\subsection{Chapter 27: Chapter 27 – Of God and Quidditch}

\textbf{Severus Snape: The Middle Years - Of God and Quidditch}

\subsection{Friday, April 9, 1982 (the day after the full moon)}

Snape went to breakfast as usual the next day, and no one spoke to him of the night before because, as he found out, Dumbledore had told none of them of it. That suited Snape well, for he didn't wish to speak of it either. Instead he exchanged pleasantries, then took his breakfast to his office with the excuse that he was working on something.

Which was only partly true. The fact of the matter was that he was working out something and needed quiet and undisturbed calm.

\emph{Why didn't I resist them? They were wrong, they were trying to trap me. Why did I go along so passively, like a sheep?} The answer, when it came, was deceptively simple. \emph{I obeyed them because I knew I'd broken the law. I wasn't supposed to speak to Death Eaters, I did, and I knew I was being punished for my own error. If it hadn't been for Judge Bones, I'd be kicking myself in Azkaban now for what a fool I'd been to talk to Bodkin. But if I'd fought back, maybe their plot would have been uncovered even without Judge Bones.}

That thought brought iron resolve for all of twenty minutes, then gave way to one more sobering. \emph{They took me because I didn't follow the rules. They were able to take me because I met them halfway. So predictable. Just like Sirius Black, the candy, and the green hair. Sirius got green hair because he stole and ate another student's candy. I got arrested because I broke the rule about talking to Death Eaters. The plot wouldn't have worked if I hadn't stepped into it.}

Rules, he realized, weren't there to control you. They were there to protect you. \emph{If I hadn't broken the rule, if I'd been arrested on a one hundred percent fabrication, I'd have had the moral strength to fight them from the beginning. When you break the rules, you forfeit the protection.} He thought of the New Year's resolutions that had lain unnoticed since January. This fell into the `Be Prepared' category. Know what the rules are, then never break them. There was a corollary: Unless you are prepared to accept the consequences.

Another problem was that of justice. \emph{Was I being justly punished for having peached on my mates} - some how at this point Severus always heard his father's voice - \emph{by having someone peach on me?}

Punished by whom? Probably not by Bodkin, who almost certainly did not know that Snape had given twenty-seven names to the Ministry. Not by Moody either, for Moody had wanted the names and wouldn't punish Snape for having given them. The question of just punishment thus presupposed an overarching source of justice - the existence of God. Snape wasn't sure he wanted to go this far, certainly not yet.

There was always the possibility of the universe expressing its sense of irony, which in its turn assumed that the universe was a conscious entity.

Luckily, Snape was able to sidestep the question entirely by remembering that the two instances were not analogous, for he had given names of people for something they had actually done - been members of the Death Eaters - whereas Bodkin had accused him of something he had not done - plan the return of the Dark Lord.

The problem of justice was temporarily shelved.

Lunch loomed, and the question of what to do. Break no rules. Snape searched for the areas in his mind that contained dangerous spontaneity and locked them behind a brand new door. Every word, every action, should be weighed. Snape went to the Great Hall.

"Ah, Severus!" Flitwick greeted him. "Up for a game or two?"

A heartbeat, the tiniest of pauses. "I think so. I'm in the middle of something, but a game or two might clear the brain."

Sprout joined them. "As long as I have a moment to go over the late spring plantings. There's a lot we could grow that you wouldn't have to order."

Another heartbeat as timetables clicked and resolved. "Would later this afternoon be all right? It would give me time to dig out my lists of what's needed for each class, and when it comes in the year."

"Severus, have you managed to look at that book on Greek myths? I have encountered a question or two..."

Black eyes met blue ones, and Dumbledore recoiled, not exactly in shock, but in consternation, the gentle smile still on his face. "Not yet, Headmaster, but I hope to soon. I shall let you know when I have," Snape said, the normal-sounding words not relieving Dumbledore's worry at all.

Dumbledore went to London to attend the wizards' council, and returned with news of the sentencing of the Lestranges and Barty Crouch. He discussed it only briefly with Snape.

"It is disturbing that Bellatrix is so positive that Moriarty has not been destroyed. Her faith in his return is unshakable. Can you think of any reason why that might be so?"

\emph{Does he want me to reveal that I know more about it than I've told them? But I don't know anything}. "No, Headmaster. I have no idea why she would think that."

"I felt particularly sorry for Barty. The dementors have frightened him badly, and I believe he would have done or said anything to be taken out of their hands. His father was adamant, though. Well, that is all."

\emph{Another warning about dementors. A reminder of what's waiting outside. Does he want to keep me frightened as well, bound to him?} "Thank you, sir. I appreciate your sharing this with me." Severus returned to his own office and Dumbledore went to talk to Hagrid.

Hagrid poured tea while Dumbledore nibbled on something that was supposed to resemble a scone. "The problem is, I am not certain if this is ultimately a good thing or a bad thing," Dumbledore mused. "If he is exploring new ways to protect himself that he will eventually have more control over and can fine tune to adapt to circumstances, then all may be well. If he has merely found a new way to hide, it could be disastrous. I wish I could get inside his mind."

"All I can say is, if he's goin' back t' where he was when he were thirteen, I'll be a mite disappointed. One thing t' hold 'em when they're small and kicking, but he's growed now and his tantrums are more focused."

"Exactly the point," said Dumbledore. "The tighter he closes himself off, the greater the explosion when it breaks through. It is not healthy."

"What're you doing about it?"

"Nothing at the moment. Interference may be counterproductive. If he were a thestral, Hagrid, what would you do?"

"Get him used t' what makes him skittish. Start with it far away and not a threat, then slowly move it closer 'til he sees it as a normal part of the landscape. Let him examine it, sniff it, play with it if he wants."

"That would work very well if he were skittish about an object, but I doubt Alastor or Rufus would let themselves be a normal part of any landscape, much less be sniffed."

"Sometimes it helps if they've a chance just t' run free for a bit, not penned-up like."

"He won't leave the grounds. I have to work him back up to that."

"Begging your pardon, Professor, but it ain't really his body that's penned up. It's his brain. He needs to open them doors and let fresh air in."

"The students return this weekend. Are you planning anything for your last few days of rest?" McGonagall's question was a general one, addressed to the resident staff as a whole. The response was restrained. Sprout was reorganizing greenhouse three, and Flitwick just wanted to rest. "What about you?" McGonagall asked Snape.

\emph{To whom might you pass information about my comings and goings?} "I still haven't finished the books I bought last week. I'll probably spend the time quietly reading."

As the teachers finished breakfast and separated, Dumbledore joined Snape. "It might be a good idea to take advantage of the opportunity. Is there nothing that needs doing outside? No place you would like to go?"

\emph{Why do you want me off the grounds? To be set up for arrest like last week?} "No, sir. I really would prefer to stay here." Snape headed for the security of the dungeons, still pondering Dumbledore's motives. \emph{Every time something happens, it makes me more dependent on him. Is that what he wants, for me to be tied to Hogwarts forever? Or is he honestly concerned about me?}

Stepping into the office, however, he noticed the unfinished lararium. Was it worth continuing the experiment to see if it revealed anything about death and the existence of an afterworld? Suddenly Snape remembered that there was something he wanted outside of Hogwarts. He hurried back up to the entrance hall. Dumbledore was still there, talking to Filch.

When the headmaster was free, and Filch gone upstairs, Snape made his request. "There is a place I want to go, but... do you think I might take Hagrid with me?"

"That would be up to Hagrid. I have no objection. I hope you have a pleasant day."

Snape waited to see if Dumbledore would ask where he was going, but the headmaster simply went upstairs. It was a good sign. Snape left the castle and went down to Hagrid's hut.

Snape was still nervous about being tailed, so he and Hagrid apparated first to London, then side by side to Pendle Hill. From the top of the great tor they surveyed the rolling countryside. "Moor country," Hagrid said. "I didn't know ya came from moor country. Always thought of ya as kind of a city boy somehow."

"Factory town," said Snape. "Mill and mine." He took Hagrid to the east side of the hill and pointed out a spot in the distance. "We're going there. Shouldn't be anyone around at this time of the morning. They'll mostly be working or at market."

The two approached the quiet, deserted street and the small house at the end. A couple of other houses still had boarded-up windows, but others that had been empty were now inhabited again. Snape released the locking spells and opened the door. Hagrid had some trouble squeezing through, but managed it. The house seemed tiny with Hagrid in it. The groundskeeper filled the whole living room.

"Ya got a lot of books," Hagrid said as Snape moved furniture to give him space to pass.

"I hope to have more. I plan to line this other wall with bookcases, too. Can I get you some tea? There's nothing to eat, I'm afraid."

"Tea 'd be nice." Hagrid followed Snape to the kitchen watching with interest the whole business of lighting coal in a grate. "Whyn't ya just magic it?"

It was a logical question, and Hagrid already knew about his parents. "Dad never liked the magic part of it, and Mum wouldn't use magic in the house. It wouldn't feel right if I did." It took a moment for the water to run clear after sitting so long in the pipes, but soon a kettle was on the grate and shortly thereafter they had tea. Snape did use magic to make one of the chairs sturdy enough for Hagrid to sit on, then excused himself to go upstairs.

He returned a few minutes later with a slim album of photographs. "We didn't have many. These were the last Dad ever had taken. We went to Blackpool when I was nine."

Hagrid examined the pictures, comparing Tobias's and Eileen's faces with Snape's own. "Ya do favor both of them, don't ya? Yer dad had a craggier face, though, and thicker hair. Are all muggle pictures, ya know, quiet?"

"Muggles don't have moving photographs. Not yet, anyway. I'll be upstairs for a bit. There's one or two other things I want to find, but the photos were the most important."

Most of the things Snape wanted were in boxes in the storeroom. He had a photo each of Gra and Nana in small frames. For his great-grandfather, he took a little voodoo doll. One thing that troubled him was that he'd never had a picture of Lily, but while looking for Wensley Snape's dark magic artifacts, he found his old schoolbooks. Flipping through them quickly he came across a scrap of paper with "Lake. Tonight." scrawled across it. It was a piece of Lily he hadn't realized he possessed.

Feeling much better now, Snape returned to Hagrid, who was fascinated by the voodoo doll. "It was the muggle one had this?" he kept asking, as if the idea that a muggle might know something about magic was totally alien to him.

They left the little house, and Snape reset all the locks. This time they didn't worry about being followed, apparating directly to the outskirts of Hogsmeade. "If ya don't mind," Hagrid said, "I'll stop by the Hog's Head for a nip. Ya don't need me anymore, do ya?"

"No, I'm all right from here. Thank you for coming. It was good of you." Snape left Hagrid to his socializing and went up the hill to the castle to assemble his lararium.

Dumbledore was waiting in the Hog's Head. "I trust all went well," he said as Hagrid ordered a firewhisky.

"Not hide nor hair of a problem," Hagrid replied. "In and out smooth as can be. He got all he went for, too."

"I am pleased to hear that. It is time something went right for him, even something this small."

"Did ya know he lived in a place like that? Fair surprised me, it did. I was expecting something a bit, well, bigger."

"The first time I saw it was after his parents died. I, too, had not realized the extent of the poverty he grew up with. It helped explain some of his discomfort around the other students."

"Ya'd think he'd move, now that he's on his own."

"You know, Hagrid, I think he is afraid of spending the money. He was very careful in the bank to find a way to conserve his resources. It is as if it is his guarantee that he will not be poor again. I do not think he could find it in himself to give up that much."

"I can see that. I'd be scared t' go back to that kind of life m' self. Fair makes my hut look like a palace."

Dumbledore did not comment on the hut. He and Hagrid chatted a bit more, and then Dumbledore, too, returned to the castle.

The little table by the fireplace was covered with a green cloth. On it, back from the center, Snape placed a photo of his parents in a standing frame and, on either side, the pictures of his grandmothers, Gra next to his father and Nana next to his mother. In front of them stood the voodoo doll and Lily's note, also encased in a frame. For the geniuses of the place Snape had rocks - smooth polished stones from the lake shore and a larger, rougher rock veined with green from the cliff. The forest had given him a pine cone. At the front of the table was a votive candle, an incense burner, and dish of clean gravel for offerings and libations.

It was, as near as Severus could tell, just as described in his book for later lararia. That was when he encountered the next problem. \emph{How do you use a lararium?} He went back to his books to try to find some description of the rites involved. What he found was very sketchy. All he could figure out was that it was twice daily, and lasted about two or three minutes, but what words you said, or what exactly you did was a mystery.

\emph{What if I do it wrong? Will that negate the experiment?} That, naturally, brought up the larger question of `Is there a God?' The first thing Severus did was revisit his conclusions about belief. \emph{Belief does not affect God. If God does not exist, my belief will not create God. If God exists, my non-belief will not cause God to disappear. I cannot use belief as proof.}

The next step was rather simple. \emph{If God does not exist, nothing I experiment with will hurt me in any way. It can't get me into any trouble. The nonexistence of God brings no new problems, but no new comfort either.}

The possibility of the existence of God did bring problems. \emph{It really depends on the personality of God. If God is gentle and kindly, any attempt to find God will meet with approval. If God is rigid and vindictive, any wrong action will bring punishment. Is wrong action worse or better than no action?}

The question could not be resolved empirically. Religion was not science. In the end, it was a leap in the dark. You might land in paradise, or you might plunge into the abyss, or nothing might happen at all, but you would not know which until you jumped.

Being by upbringing and inclination an agnostic, Snape was not overly concerned about the punishments of a rigid and vindictive God. \emph{If I find out that what I am doing is not exactly right, I can change it. The point is to honor God and the spirits. All religions believe in spirits. Sometimes they call those spirits by other names, like angels, but they have them.}

By then it was supper time, and Snape went to the Great Hall, but he didn't stay long. Instead he took a morsel of bread and a small vial of - well it was going to be pumpkin juice, but the whole problem of combustibility made him take firewhisky instead - and returned to his office. There he agonized over the question of sacrifice and finally gave up a five-pound note. The note, folded small, and the morsel of bread, he placed on the dish of gravel, the vial of firewhisky next to it. Then he lit both the candle and the incense.

It felt awkward, speaking out loud in an empty room. Standing in front of the table, Severus said, "I dedicate this place to God, whoever and whatever God may be, and offer these symbols of food and drink to represent my continued life, and money to represent my labor, in honor of God." With that he poured the firewhisky over the bread and money, and ignited it, watching the resultant fire with pleasurable interest. Then he continued.

"I honor the spirits of cliff, lake, and forest that shelter and protect Hogwarts. I honor the spirits of my parents and my grandparents, of my great-grandfather, and of Lily Evans. If they are still able to see and hear me, I ask their continued interest in my life. May good fortune attend all in this place. Let it be so."

That was it. It was somehow, except for the fire, unsatisfying, being far too short and simple. \emph{Maybe the repetition makes it more meaningful. I have to remember to do this twice a day. Not the five pound note, though.}

Severus went back to the Great Hall for supper. He discussed plantings with Sprout and played cribbage with Flitwick, and to all appearances his flirtation with religion changed nothing.

The students returned for the summer term, and immediately the world of Death Eaters and the Ministry intruded itself into Hogwarts. Anna Prendergast and Richie Gamp came to Snape together to tell him that both their fathers had been sentenced to Azkaban. Snape let them mourn in the privacy of his office, then informed the other teachers, so that any teasing might be nipped in the bud.

The biggest, most pressing focus of the school, however, was far more normal than anything the outside world could force on them. The fifth and seventh year students were in the last stages of preparing for OWLs and NEWTs, and the library and study rooms were packed with serious, and occasionally panicky, scholars. Snape even held review sessions that reminded him of his tutoring days when he was a student.

And, of course, there was the upcoming Quidditch match between Ravenclaw and Hufflepuff on the twenty-fourth of April, the last Saturday of the month.

"Who do you favor?" Sprout demanded of Snape two days before the match.

"Ravenclaw, of course. If they win, then the worst we can do is second place. But if you win we might end up in third place."

"Humpf," said Sprout. "You were a much nicer person when you didn't understand Quidditch."

Snape had been saying his private ritual in front of the lararium for about two weeks, minus the burnt offering, with indifferent results. At the beginning it had been hard to remember, and Snape had to force himself to stop, recollect, and say the words. At the end of a week it became more automatic, an almost habitual part of his routine. He still had no reaction from it until the night before the Quidditch match.

\emph{He was in a greenhouse checking the replanting schedules and stopped in front of a Lilium bulbiferum, Sprout's pride and joy. "Where am I going next," the flower asked.}

\emph{"One of the gardens," he answered. "I'm sure you'll like it there."}

\emph{"Not the Magician's garden. He has roses, and I don't like them."}

\emph{"I'm not sure which one. I don't know much about the gardens."}

\emph{"Do you remember the first one we ever saw?" The flower was smiling.}

\emph{"Of course. We had to sneak in."}

\emph{"Whose stands were we in?"}

\emph{"Hufflepuff's."}

\emph{"Who lost?"}

\emph{"Ravenclaw."}

"\emph{Remember me."}

Snape started awake. It was three in the morning, and he was shaking like a leaf in a storm. He lay for a while in bed, trying to recall who and where he was, the flower's voice as real and present as the touch of his sheets and pillow.

After about fifteen minutes, he got up and went into the office, where the embers of the fire still cast the tiniest of glows. The lararium was veiled in darkness, but Snape stood in front of it, looking towards the spot where Lily's shadowed note lay. \emph{I remember}, he thought.

He sat for an hour and a half, until dawn began to soften the dark, staring at the embers. He knew little of the gardens, but he remembered the stealth of feigning to be Hufflepuff and his first joy of bludgers. It was a message he would heed. \emph{I need a control. Every experiment needs a control.}

The control came at lunch in the form of Professor Kettleburn. Snape cornered him. "Make a bet with me."

"Who do you favor?"

"I want Ravenclaw to win, but I want to bet on Hufflepuff. Make a bet with me."

Kettleburn stepped back and eyed Snape suspiciously. "Why?"

"I had a dream. I just have to bet something on Hufflepuff."

"You got it. You want a galleon, a sickle?"

"Just a knut. All I need is a bet."

Kettleburn agreed with no hesitation, then went and placed all his other bets on Hufflepuff.

It was a hard-fought game, for Ravenclaw needed to win, but Hufflepuff needed to score. The yellow and black kept the blue and bronze away from the Snitch for over two hours, until the score favored Ravenclaw 150 to 120. Then, in a breathtaking bit of flying, the Hufflepuff Seeker found the Snitch, and the score ended at 270 to 150. The Hufflepuff stands went crazy.

Ravenclaw was in last place; they had won no games, so their total score was irrelevant. Hufflepuff had two victories and a total of 450 points. Slytherin had two victories and a total of 420 points, but with a game still to play. Gryffindor had one victory, 300 points, and a game still to play.

If Slytherin won their last game, then Slytherin would win the Quidditch Cup. Hufflepuff would be second unless Gryffindor managed at least 160 Quaffle points - not likely. But if Gryffindor won without Quaffle scoring, then they tied Hufflepuff for the Cup. On the other hand, a Slytherin Quaffle score of at least forty would beat out Hufflepuff... There was no way to explain it without a diagram. What it meant was that in the last game Slytherin had to win outright or, barring that, to outscore Gryffindor by at least forty points with the Quaffle.

Kettleburn cared nothing for the long-term planning. He'd bet Hufflepuff, and he'd won from everyone except Snape to whom he cheerfully paid the knut. "You tell me about any other dreams you have," he said as he left.

Did the Hufflepuff win mean what Snape wanted it to mean? Was that really a communication with Lily, or was it a subconscious desire of his own mind expressed in a dream? Snape had no way to tell.

True, the Hufflepuff victory had been unexpected, not so much because they won, but because of the way they won. To hold Ravenclaw from victory for so long while at the same time scoring so many points was not Hufflepuff's usual form of play. They now had a chance at the Quidditch Cup, something no one had anticipated in the fall.

\emph{Why would I dream that? Why would I dream anything about Quidditch at all? If Lily and my parents can speak to me, why don't I ever dream about Nana and Gra?}

Another thing Snape now wanted to know was whether or not the dead learned the truth about people. \emph{Does Lily know that Sirius betrayed her? Has she learned what a hypocrite her husband was, pretending to hate dark arts and using curses, hexes, and jinxes on a daily basis, as pranks or as a means to intimidate other students? Has James found out what a fool he was, making a traitor his secret keeper and causing his wife's death?}

The idea that the dead discovered the truth was a little eerie. \emph{What is there about me that I would not want Lily to know? In all honesty, he could think of nothing. She already knew about his demon. She would learn he'd been a Death Eater, but she would also learn that he'd turned against the Dark Lord and risked his life for her. In fact, the idea that Lily would know the truth after she died was, on the whole, a comforting thought.}

One thing Snape was sure of was that he would continue to use the lararium. The dream by itself wasn't proof that it worked, but it certainly wasn't proof that it didn't work. He was rather hoping for more dreams.

The next weekend was a Hogsmeade excursion, and Snape had supervisory duties in the town. He wandered the streets for a while, then went into the Three Broomstick for a snack and something to drink. He'd just settled in a corner with tea and biscuits, when a figure rose from another table and slipped into a chair next to him. It was Gawain Robards.

"I thought you were supposed to leave me alone," said Snape.

"Moody is supposed to leave you alone. There was no injunction written against me."

"Are you going to try arresting me again?"

"That wasn't me. That was Moody and Scrimgeour. I did my job, got my conviction, and sent you to prison."

"I'm not in prison."

"No? That's not the impression I had. It doesn't matter. This is more in the way of a friendly conversation. I just wanted you to know that Moody's getting help adjusting - he doesn't want the help, but he's getting it. Crouch is out, and Scrimgeour is looking to move up, so he won't be after you any more. He's angling for a lot bigger fish."

"It's kind of you to tell me this."

"I don't do anything out of pure kindness. I want to know what you can tell me about him." Robards lay a piece of paper in front of Snape with the note `L. Malfoy' written in one corner.

\emph{Can this be another trap? At least he's asking about someone specific, and not just for names in general.} "His father hated... `him', and he wouldn't become a... one of us until his father died mysteriously of dragon pox. I think it was his father's death that finally drove him to it."

"Mysteriously? What do you mean?"

"There hadn't been any outbreaks. It was an isolated case and..." Snape paused. It was the first time he had ever voiced the suspicion. "I know that shortly before it happened, the Dark Lord was interested in protection from dragon pox. The infection may not have been accidental."

"Did you know him well?"

"A little, from before, when none of us had yet joined. Afterwards, I hardly saw him at all."

"Do you know if... `his' agents ever used Imperius curses on people."

"Oh, yes. They did. I didn't know it at first, but I overheard some of them talking. That was a tactic I know was used more than once."

"So this person might be telling us the truth?"

"I don't know for certain, but it's possible."

"Did he ever go out on raids?'

"I don't know. I never saw him in any of my classes."

Robards seemed pleased with the information. "That'll help with the case. It gives me a better idea what to look for. You haven't been in contact with him recently, have you?"

"No, sir."

"Be sure you don't. You escaped the consequences for talking to one of your old colleagues. You wouldn't be able to wiggle out of this one." Robards rose to leave. "Well, thank you for the conversation. Enjoy your day with the students."

Snape watched as Robards left the Three Broomsticks and headed for the outskirts of the town, where he could apparate beck to London. \emph{Am I ever going to be free of the Ministry?}

When Snape was a student at Hogwarts, the mutability of time had always intrigued him. At the beginning of the school year, a month seemed to last an eternity. At the end of the year, a month sped by in the blink of an eye. It was the same as a teacher. As May rushed to its closing, the school was concerned with two things - exams and Quidditch, and Quidditch predominated. For the first time in years, it was not certain at all who would win the Quidditch Cup, or even who would be second.

\subsection{Saturday, May 29, 1982 (the first quarter)}

"Who's going to win?" was Kettleburn's first question. "Any dreams?"

"I haven't a clue," was Snape's response. "And none at all."

There were no house alliances at this game. Slytherin needed to win, or outscore Gryffindor with the Quaffle by at least forty points. Gryffindor needed to score and then win. Hufflepuff wanted a quick Gryffindor win with no prior scoring just to tie for first place. Only Ravenclaw went simply to watch a game. Whatever happened, they were in fourth place.

As the students packed the stands, the teachers, too, took their places. This time Snape and McGonagall were separated, with Sprout and Flitwick between them, Sprout next to Snape. They all knew it would be a short game. Once Gryffindor scored, both Seekers would be hunting the Snitch.

Madam Hooch released the four Quidditch balls, and the game began. Gryffindor got the Quaffle and made a run at the hoops. Slytherin blocked and took the Quaffle. Then it happened.

The Slytherin chasers rushed the Gryffindor hoops, and the Quaffle carrier entered the scoring area. Just as she made her pass, the Gryffindor Keeper suddenly looked beyond her and pointed. She paid no attention, taking advantage of his distraction to score, but there was no resounding cheer from the Slytherin stands. Instead, the entire school rose to its feet with a gasp of dismay.

Out in the center of the pitch, both Gryffindor Beaters had hit the same Bludger at the same time. Impelled by the double force, the Bludger rocketed forward with murderous speed and struck the Slytherin Seeker in the back. He tumbled from his broom and lay motionless in the grass.

Snape was out of his seat at once, running across the pitch, Madam Pomfrey right behind him. Madam Hooch called a halt to the game. As the silent stands watched, Snape and Pomfrey knelt by the unconscious Seeker. Dumbledore and the other teachers gathered around them.

Pomfrey checked the boy's back while Snape gently held his eyelids open and looked into his eyes. "I need to get him up to the hospital wing at once," Pomfrey murmured, "to check for internal injuries."

Snape shook his head. "Cracked ribs," he whispered. "He's had the breath knocked out of him, but I don't see anything else."

Pomfrey regarded Snape for a few seconds. "Maybe there's more of Constantina Rossendale in you than I thought." She rose to her feet and conjured a stretcher. "It looks like nothing serious. We'll move him to the sidelines and I'll treat him for the injuries. I can't say yet if he'll be able to resume the game."

That was it. The game had to continue, with Slytherin minus their Seeker. This meant that Slytherin had no chance of winning the game, because only the Seeker could catch the Snitch. They would have to outscore Gryffindor by more than 150 points, a clear impossibility. Gryffindor could score its goal and hunt the Snitch at leisure.

But Slytherin wasn't going to let them do it. Fired and focused with anger, the Slytherin team now had one goal - to keep Gryffindor from scoring. The Beaters began aiming for the Quaffle, forcing Chasers to swerve away from the scoring area and even knocking the Quaffle itself away from the hoops twice. McGonagall protested, but Hooch pointed out that there was no rule about hitting the Quaffle with a Bludger. The Chasers, meanwhile, swooped and attacked each other like dog-fighting airplanes. The miraculous happened. Slytherin scored again, and it was 20-0.

Then, on the sidelines, Slytherin's Seeker appeared, talking calmly to Snape, Pomfrey, and Dumbledore, and it became clear that he would reenter the game. Gryffindor's choices were narrowing. If they couldn't score, a Slytherin win would put them in third place and endanger their chances of winning the House Cup. If Slytherin scored twice more, even a Gryffindor win in this game would put Slytherin in first place for the Quidditch Cup. A Gryffindor win now would tie them for first place with Hufflepuff. It was a tense, spit-second decision for the Gryffindor team, and their Seeker went for the Snitch.

Half of Gryffindor was screaming for him to stop, sure that they could keep fighting for a definitive win, but it was too late. In a breathtaking dive, the Seeker grabbed the Snitch. The game was over. Gryffindor won the game to share the Quidditch Cup with Hufflepuff. The Hufflepuff stands went wild.

"How strong is it?" Madam Pomfrey asked Snape as they climbed the hill back to the castle.

"Not very. Most of the time I can't see anything, but when someone's knocked out like that, there's no interference. Nothing blocking it, if you know what I mean. My grandmother had me do it once for an injured townsman."

"What did you see that time?"

"A ruptured spleen."

"That specific? It's good to know if there should ever come a time..."

Kettleburn caught up with them. "What a game! Who'd 've thought? I figured it was all over when your Seeker went down! Sprout owes you one!"

"How did you do?" Snape asked politely as Pomfrey left them and hurried forward to get back to the hospital wing.

"Not well. Lost some and tied others. I should have listened when you said you didn't have a clue. Well, there's always next year."

"Ah, but next year we won't have the same Beaters. They're seventh years."

"More's the pity. If you need help with the selection, let me know."

The mood in the Slytherin common room was almost like a victory party.

"Did you see it, Professor? Did you see!" Algie was as happy as it was possible for a losing player to be. "They put their tail between their legs and ran! They couldn't face us team to team! They took second best because they couldn't beat us!"

It was good that even defeat was a kind of victory, and Snape wondered if Gryffindor saw their victory as a kind of defeat. Probably not. That viewpoint usually belonged to the losers. Still, Gryffindor had clearly not wanted to take any chances against a full Slytherin team, and that was something that hadn't happened in a long time.

"I'm sorry you won't be here to help us win next year," Snape told Algie

Chris joined them then. "I've been telling him he has to fail his NEWTs so he can keep playing. Then you really would win next year."

"I think we'll muddle along without the two of you," Snape said, and left the common room to its partying.

The next day, Snape asked to speak with Dumbledore in his office. He was requesting to be allowed to stay at Hogwarts during the summer break.

"It would give me a chance to do some curriculum planning, to rework lesson plans, to do research in the library..."

"Alastor has been ordered not to bother you." Dumbledore sat at his desk as Snape paced the office.

"The injunction expires in mid July. What if he does something then?"

"There are more people watching out for you, Severus, than you realize. Not just me, Hagrid, and Judge Bones, you know. Alastor's friends don't want to see him in more trouble. They will be keeping an eye on him, to be sure he does not transgress again. They will keep an eye on you, too, to be sure you are safe."

"Spy on me?"

"I agree that there is a fine line between protection and control, but I assure you that no one wishes any control beyond what the court order requires, and that is that you remain under my authority. I would like you to go home for the summer. I would like you to live a more normal life than has been possible these last few months. The `spying' is intended to help you do that. Is it too much to ask?"

"No, sir," Snape replied.

June was well under way, and suddenly the Sunday when the OWL examiners arrived was upon them. Snape had proctoring duties for written exams in subjects other than his own, and partial supervision of the astronomy nighttime practical session. In addition, he had his own exams to give and grade.

And then it was over. Students, results in hand, were packing and saying their goodbyes for the summer, teachers were storing equipment and sealing their rooms, and the whole school was shutting down.

Professor Sprout arrived at the farewell feast beaming from ear to ear. When the other teachers finally got her to tell them why she was so pleased, they found she was spending July in the Amazon rain forest on an educational and plant specimen collecting trip.

"Anacondas," said Hagrid, having overheard the conversation.

"I beg your pardon," Sprout said, bewildered.

"Ya might bring back a breeding pair of anacondas. I'd like a snake or two t' keep the vermin down in the gardens."

"Don't they get rather big?" asked McGonagall. "I, for one, don't want any hundred foot long snakes eating the students."

"Maybe she could bring back snake eggs instead of snakes," Snape suggested.

"That shows how much you know," Hagrid informed them. "Longest confirmed measurement of a South American anaconda was twenty-eight feet, and they don't lay eggs. Ya been listening to too many stories."

"I stand corrected," McGonagall retorted, "but I don't want any twenty-eight foot long snakes eating the students either."

"You couldn't have anacondas here anyway," Kettleburn jumped in, this being his specialty. "They're water snakes. Can't support that bulk on the ground or in trees, just in the water. Think what the merpeople would do if we put giant predators into their lake. And if you brought a breeding pair you wouldn't have a couple of baby snakes, more like six dozen."

"No anacondas," said Sprout.

"What are you doing for the summer?" McGonagall asked Snape.

"Staying at home mostly. I have a couple of projects I'd like to finish. What about you?"

"My son and his wife are going to Thailand, so I'll be riding herd on the grandchildren for a few weeks. I'll need the time to whip them back into proper shape. Generally they are badly spoiled. These young folk don't know how to raise children."

"I am off to the upper Nile and the pyramids of Nubia," Flitwick announced. "It's something I've always wanted to do, and this year I am doing it."

\emph{Everyone else has interesting plans, and here I am thinking it's something just to have the nerve to go home. How pathetic can you get?} Still, it had been several years since Snape was able to spend an extended time in his own town, and there was something to be said for the quiet life.

\subsection{Saturday, June 26, 1982 (2 days before the first quarter)}

Hogwarts was closing. Only Dumbledore, Hagrid, Filch and the house-elves remained to watch over the school. The graduating seventh years for the most part apparated out. Many other students left from Hogsmeade with their parents, while the rest took the train down to London. After they were gone, the teachers bade each other a pleasant summer and left as well.

Snape arrived back in Lancashire with his old Gladstone bag and a large parcel of books. He left the lararium in Hogwarts, having concluded that experiment for the time being. After setting his things down in the sitting room, he went back into his little kitchen to rediscover the fact that he had no food in the house.

\emph{Idiot! You have to go shopping! What do you get?} He remembered his cookbook then, and went to the sitting room to open the parcel. Most of the recipes seemed very complicated, but then he settled on one for chicken paprikas with noodles that looked relatively simple. He started to make a list of things he would need, and included the eggs, milk, sausages, coffee and other things he could think of for breakfasts and lunches.

That was when Snape remembered that he didn't have any electricity for the icebox. \emph{I could put a spell on the icebox to keep it cold inside. But the magic is traceable. On the other hand, I already used magic here for Hagrid's chair...} In the end Snape decided that one small spell on the icebox was worth the trouble. After that he went shopping.

The market on the other side of the river was bigger than he remembered, and then Snape realized that it was, in fact, new - a different place from the market of his childhood. It also carried a wider variety of goods. He took a basket and started to look for the things he needed, overwhelmed by the variety and the lack of experience that made it difficult for him to choose.

"You," said a matronly voice behind him, "aren't you Eileen Snape's boy?"

Snape turned to find himself confronting a familiar face. "Good afternoon, Mrs. Hanson," he said, "you have an excellent memory." They chatted for a few minutes, then made their purchases and left to go their own ways. It was a pleasant encounter.

The chicken paprikas took twice as long to make as Snape anticipated, but it tasted quite good.

\end{document}
