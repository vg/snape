\documentclass[a4paper,11pt]{article}
\usepackage{ulem}
\usepackage{a4wide}
\usepackage[dvipsnames,svgnames]{xcolor}
\usepackage[pdftex]{graphicx}

\usepackage[utf8]{inputenc}\title{A Difference in the Family: The Snape Chronicles}
\usepackage{hyperref}
% commands generated by html2latex


\begin{document}

\subsection{Chapter 29: Chapter 29 – Blackmail}

\textbf{Severus Snape: The Middle Years 1982-1983 - Blackmail}

\emph{Sunday, November 7, 1982 (the day before the last quarter)}

Dr. Carmichael made her next move on Sunday.

"You have quite a fan club in Alastor Moody," she cooed as she settled into the seat next to Snape at brunch and helped herself to an omelet. "We were comparing notes."

"That must have been dull. It was lucky you had the Quidditch to liven things up." Snape was selecting his usual kipper, toast, and coffee.

"On the contrary, it was fascinating. A young, unattached male Death Eater with an interest in sex and murder. It sounds like material for a popular novel. How many parents would feel comfortable with someone like that in daily close contact with their teenage daughters? Think of the scandal if I should happen to slip and mention it in an interview with \emph{The Daily Prophet}."

"Are you trying to blackmail me?"

"You do cut right to the chase, don't you."

"I see no reason to waste time with someone as direct as you."

"Good boy. Now, what would you do if your position here was suddenly attacked by outraged mothers?"

"You're behind the times. It already has been, and Professor Dumbledore stood behind me all the way."

"Ah. That was the Death Eater business. I'm talking about sixteen- and seventeen-year-old girls in your NEWT level classes. There's not that much difference in your ages. Don't tell me you've never been tempted." Carmichael stretched out her hand as she spoke and turned Snape's jaw so that he faced her, meeting her eyes. "Oh," she said quietly, and looked away again. "There's more to this than I thought."

It was a shock to Snape as well to discover that Val Carmichael was something of a legilimens. His mind raced quickly over the past couple of months, trying to recall any incident when looking at her had brought unexplained random thoughts to the surface, but there were none. This was the first time she'd tried legilimency on him.

"Does it bother you?" he asked.

"Not at all, but I can see where it might bother you, dear. Intimacy does require a disturbing degree of... openness, doesn't it? Not something you're good at - or comfortable with - I imagine. I am now on the horns of a dilemma." Carmichael toyed with her food, then continued. "I'm beginning to see why someone like you might reject the advances of someone like me. Or of anyone for that matter. That little shell of yours is going to be very painful to crack, and I'm not into pain - quite the contrary. On the other hand, I have a reputation to protect. Men don't reject me, at least not publicly."

"What would you suggest?"

She considered him speculatively. "Whatever happened to the sweet, innocent boy of last month who blushed at everything I said? I rather miss him. At least there was some warmth there. This new one has gone all cold on me."

"I'll try to conjure up a blush."

"You'll conjure up more than that. Thursday's your night off. You're taking me on a date."

Snape's fork stopped in mid air as he glanced sideways at her. "Why?" he said.

"I told you. I have a reputation to protect. I'll leave you alone here at Hogwarts if we give the impression something's happening outside."

"Why don't you just choose someone else?"

"Who? Binns is a ghost, Flitwick and Dumbledore are too old. The others are married."

"Hagrid? Filch?"

Carmichael glared at him. "I'm trying to let you off the hook, and you're insulting me. I can get nasty, too, you know."

"What kind of date were you contemplating?"

"They tell me you're a muggle-raised half-blood. I'm muggle-born. How about dinner and a movie?"

Snape thought for a moment. It didn't seem like such a bad proposition, as long as they kept it platonic. "Anything in particularly that you want to see?" he asked.

"Surprise me."

And that was it. Snape had a date for Thursday evening.

The next problem was deciding where to go for dinner and what movie to see. Fortunately, Snape knew something of Dr. Carmichael's taste in food, and there was more than one restaurant, nice restaurants with good atmosphere (at least that was how they appeared through the windows), that catered to the plain, British, roast beef taste. On a Thursday evening in November there should be no problem getting a table.

It would, however, be more expensive than Snape was used to, which meant he would have to take a relatively large sum out of his bank account, necessitating the use of his cash card. He didn't really want Dr. Carmichael to know he had a muggle bank account, but there was no help for it.

That left the choice of a movie. Snape had restarted his subscription to the Guardian with the beginning of the new school year, and he pored over the listings trying to find something that he could watch with her that wouldn't be suggestive of a closer relationship.

That proved hard to do. It was November, and the movie listings were mostly films that had come out in the spring and the summer. Snape had saved old newspapers and was able to check the reviews, and was appalled at how many of them had either a romantic theme or were blatantly risqué. One or two looked promising on the surface, until he read the review. One, which he'd hoped to be a scholarly work, turned out to be very graphic. \emph{It's a good thing I have the newspaper. I'd hate to think what would happen if I took her to that one.}

In the end, the only one that seemed safe was a film about a man who got himself trapped inside something called a computer. Since much of the movie itself was actually made on one of these computer things, Snape thought it might be interesting to watch.

\emph{Thursday, November 11, 1982 (3 days after the last quarter)}

McGonagall caught him at the top of the dungeon steps when he came up for breakfast on Thursday morning.

"Severus, you are not spending the evening with that woman!"

"How did you know?"

"It is true, then? She's talking about it at the high table, how you've been planning a romantic evening for ages, but it had to be postponed because of the `unpleasantness.' And whenever she says `unpleasantness,' she turns and smirks at me. How could you?"

"It hasn't been for ages, and I have no intention of it's being romantic, but please don't make a row. She wants this, and I don't need any more enemies."

McGonagall regarded him with just the tiniest touch of sympathy. "All right, I won't say anything. But you keep away from that woman's claws. She's a shark."

"Sharks don't have claws."

"Jaws, then. Or think of her as a tigress, ready to tear you to shreds."

"Nice image. I'll keep it in mind," said Snape, wondering if McGonagall had talked to her own son this way. When Carmichael was younger, talk like that might have been counterproductive.

Dumbledore was informed that neither Snape nor Dr. Carmichael would be at supper, and when Snape left his dungeon rooms late that afternoon dressed neatly in Victorian black frock coat and trousers with an overcoat against the cold, Carmichael was waiting for him in a rather elegant green cocktail-length dress, a thick, warm, hooded cape over her arm. Students on their way to the Great Hall paused to watch.

Taking the cape, Snape said politely, "You look very nice tonight," and held the cape open for her to step into, which she did leaning back a little so that her face turned toward him as she replied, "So do you." Around them students were whispering and stifling giggles, but Snape did his best to ignore both them and the amused teachers as Carmichael slipped her arm through his and they walked out into the November evening.

Neither spoke much on the way down the hill. Snape didn't look back either, but Carmichael did, and raised her free hand to wave. "We have an audience," she told him, which didn't make Snape feel any more comfortable about the whole business.

Once outside the Hogsmeade gate, they apparated to a little alley near the Leaky Cauldron. Snape led the way first to a cash machine in Leicester Square to get the necessary pounds while Carmichael watched, fascinated.

"You're a very strange wizard, even for a half-blood," she said. "Half the muggle-borns forget how to do things like this the moment they pick up a wand. It's as if they had a frontal lobotomy."

The restaurant was exactly what Snape had hoped, elegant and formal, where they could order perfectly normal soup, salad, and roast beef. Both oysters and escargot were on the menu, but neither of the two felt adventuresome.

"So," said Carmichael as they waited for their dinner. "Tell me about yourself."

"There isn't much to tell. My mother was a witch, my father a muggle..."

"Was? Are they dead?"

"Car accident. It was a long time ago. I went to Hogwarts, and now I teach there. That's all there is."

"Except for the excursion into the Death Eater business."

"I'd rather not talk about that. Besides, no one could top Moody's fertile imagination. What about you?"

"Muggle-born in a community with a large wizard population, so the Hogwarts letter came as no shock. It was harder for muggle-borns then, but Professor Dippet recognized my potential and had a job arranged for me in the Ministry with Muggle Artifacts. Unfortunately there was a change of teaching staff in my seventh year, and one of the teachers wasn't very experienced. As a result, we weren't adequately prepared for the NEWTs and I couldn't have the job. I got a job with a muggle company instead and met my first husband, Mr. Wolfstone. He was a muggle."

"So it turned out all right, then?"

"It did not. Jacob and I fought like cats and dogs. By the age of twenty-three I was in the middle of my first divorce. That was when I met Ignatius Carmichael. We traveled the world together, he and I, and I started writing. That's why I kept his name, even after the divorce."

Snape's face and mouth remained closed, but Carmichael leaned across the table and patted his hand. "You don't have to ask. Five husbands, five divorces. I'm beginning to think maybe it's me. There was Mulberry, then Pendennis, and the one I'm currently dumping is Buntley. Those are the ones I actually married. Variety is the spice of life."

"I don't think I could do that," Snape said.

"Dear, as tight as you've locked yourself up, I'll bet there's a lot you can't do. I wouldn't wager money you could even talk to a girl, much less go to... Oh, dear," she began to laugh, "I've got you blushing again!"

The food came then, and for a few minutes neither of them spoke.

"I'll bet," said Carmichael, pouring herself a second glass of wine, "that you've been wondering why Professor McGonagall doesn't like me."

"It isn't really my business."

"But you're stuck right in the middle of it, dear. I'd say that makes it at least a little bit your business."

When Snape didn't respond, Carmichael went on. "When I was in seventh year, Professor McGonagall's son started in first year. He was a cute little boy, and because she was one of my favorite teachers, I kind of adopted him - helped him with his homework, things like that. Then I left, got married, and six years later, in the middle of the divorce, I came back to Hogsmeade to have some quiet time and reflect on the direction my life was going. It happened to be an excursion weekend, and Marcellus recognized me and came over to say hello.

"Well, he wasn't eleven any more, he was seventeen and an adult. Smart and good-looking. And worried about his future. His mother was trying to push him into a career he hated. He needed someone to talk to, I needed someone to talk to, and so we talked. I met him the next excursion weekend, and then over the Christmas break. By then we'd both realized that we wanted more than talking.

"McGonagall raised the roof. She threatened him, and she threatened me. She went so far as to tell him she would disown him, and even hired spies to follow him. He was going crazy. During the Easter break he told me he had to get out of the situation and asked me to help. What was I to do? He dropped out of Hogwarts, and we were together for six months. Then he went back to her and patched things up while I took up with and married Ignatius. We've never seen each other since - I understand he's married and has children and is very happy - but McGonagall's never forgiven me."

"That's it?"

"That's it. It was more than twenty years ago."

They talked a little more about other things, then Snape said they had to leave to catch the movie. The cinema was just on the other side of Leicester Square, and they had plenty of time to stroll leisurely over to it. Snape had been in a cinema once or twice as a boy, and to the theater more recently, and so managed to figure out the box office and the tickets with a minimum of trouble.

Just before they went in, Carmichael looked over the posters with some distaste. "Are we sure we want to see this?" She asked.

"I thought it might be interesting. It's the cutting edge of computer graphics technology."

"If you say so, dear, though I never thought I would ever watch something called Tron. Sounds like a cartoon monster to me."

"You can't tell me you enjoyed that!" exclaimed Carmichael as they left the cinema. "It made no sense at all."

"But it did," Snape replied. "It was fascinating. The master program was taking over all the peripheral programs and incorporating them into itself. If it couldn't take over all of a program, it diverted its functions into non-vital areas, then eventually erased them." He was thinking of the NASA computers that ran his beloved Apollo and Voyager spacecraft. "Governments and large companies are all using computers like that. I think they're even marketing a couple that you can use in the home."

"How would you know that?"

"I read a newspaper."

"Okay, smart guy, in the movie, how did the person get into the computer?"

"That was because of the orange."

"What orange?"

"The one they were experimenting on. The one they separated into its molecules then put back together. The computer used the machine on him."

"Sounds like `Star Trek' to me."

Snape stopped dead in the middle of Leicester Square. "What's `Star Trek'?" he asked. "Because there's another movie out that has `Star Trek' as part of its title."

"It's a television show I used to watch with my third husband, Peter Mulberry. He was a muggle, too. That would have been about fifteen years ago. They were always sending things to different places by mixing up their molecules."

"Maybe I'll go to that one next time. Science and the stars. Sounds interesting."

"Science fiction and the stars. There's a difference."

"Didn't you like anything about the movie?"

"Tron was cute. I wouldn't mind crawling into a computer with him. Flynn wasn't too bad either. What do you say to a little nightcap?"

"Do you mean a drink? I can't. It's time for rounds at Hogwarts, and I have to get back."

"Pooh!" snorted Carmichael. "Back it is, but I accept the date for next week."

"What date?" Snape was taken aback.

"To see the `Star Trek' movie, of course. I wish I'd known earlier that you were into this science thing. We could have had a wonderful September and October."

Snape stopped again. They were almost at Charing Cross Road and their apparation point, but the conversation had taken a dangerous turn. "I can't do this with you again next week."

"Why not?" Carmichael's eyes narrowed dangerously.

It was an embarrassing moment, and Snape knew he was blushing again, though the darkness of the night mercifully hid it. "I don't have the money. I mean, I can't afford to spend that much money every week. I haven't got a lot, and I need to be careful."

Carmichael threw back her head and laughed. "If that's all, dear, don't you fret for a moment. I'll take you out. After four divorces and into my fifth, and two of them well-to-do muggles, not to mention all my book sales, I have more than enough to support your taste in movies. I've supported tastes in much more expensive things before now. You just support my public reputation, and you can write your own ticket."

They apparated back to Hogsmeade and walked up the hill to the castle. As they neared the staff room, Carmichael began to tell a joke, so that Snape was smiling when they walked in to greet McGonagall, Flitwick, and Sprout. McGonagall glowered, but offered sherry. Carmichael accepted, but insisted on pouring it herself. The tension was hidden behind masks of civility, and for the most part Snape paid no attention, instead explaining the movie to the others.

"What's a computer?" Sprout asked when he was done.

"Never mind," Snape said, shaking his head. "It's a muggle thing."

That answer apparently gave Dr. Carmichael great satisfaction, for she raised her glass in a mock toast to Professor McGonagall, a look of triumph on her face. "Yes," she said as she set down the goblet and wished the others good night. "There are certain things that only we muggles can understand."

"Well, that's over with," said McGonagall when Carmichael had left. "You don't have to do that again."

"Yes, I do," replied Snape. "We're seeing another movie next week." He ignored McGonagall's angry glare.

The next day, Snape was introduced to another aspect of his `date' with Carmichael - the students. The first time a couple of girls stopped to watch him as he walked past them, he thought it was his imagination. By the short time it took him to cross the entrance hall and go in to breakfast, he knew it wasn't. He felt like the focal point for every pair of eyes in the school. Boys smirked, girls giggled, and it took every ounce of control Snape had not to start blushing again.

Kettleburn didn't help. "Well," he asked as soon as Snape sat down, "how was she?"

"I beg your pardon?"

"How was she?"

Toby's demon was back with full force, and Snape wanted to smash a fist into Kettleburn's nose. Rigid with anger and a growing sense of humiliation, he replied icily, "We went to dinner and a movie."

"And then?"

"We came back here to join the others in the staff room."

"And then?"

Before Snape could answer, Val Carmichael walked into the Hall, clearly rejoicing in the fact that she was the center of attention. And she had changed. Her hair was a gentler honey color, her makeup subtle and flattering. She wore robes of a demure heather green, and her soft, low-heeled shoes were soundless on the stone floor. In making no attempt to hide her age, she'd somehow managed to appear younger.

Both men rose as she approached, and Kettleburn relinquished his seat to her automatically, as if the place next to Snape was now hers by right. She settled in, naturally and comfortably, as little groups of students at the different tables began whispering. In the center of the high table, McGonagall glared, her nose pinched with anger.

"I slept very well," Carmichael began sweetly. "There's nothing like a romantic evening to relax you and give you a good night's rest."

Snape said nothing, but Kettleburn was grinning from ear to ear.

It got worse as the day progressed. Students paid scant attention to their assignments in his morning classes, and Snape caught tiny snippets of their conversations as he moved from cauldron to cauldron.

"...what she sees in him..."

"...enough to be his mother..."

"...better than Filch, I guess..."

"...both must be desperate..."

"...imagine them kissing..."

Toby's demon seethed inside of Snape, his manner becoming colder and colder as he struggled to keep his anger in check. He hardly dared say anything to the students for fear that it would be the wrong thing, or that he might lose control and scream at them, which would just make matters worse.

Lunch was an ordeal, the whispers and stares more pronounced, the level of mirth in the students' faces rising. When Carmichael came in and sat beside him, there were open giggles. Snape hardly spoke to her at all.

Then, in the afternoon classes, he found out that she had been talking about him. Talking about him to her classes. Peterson of Gryffindor didn't even try to hide the laughter in his voice as he settled into his seat before class began. "It's okay if you don't have our papers corrected from yesterday, Professor. We heard you were... occupied." His quip met with general laughter from the Gryffindor students.

Without even thinking of it, Snape had his wand out of his sleeve and in his hand, striding quickly over to Peterson's desk. Their eyes met for a fraction of a second, then Snape was pushing the demon down with all his strength, forcing it into hiding, strapping it into the most hidden corners of his mind and bolting all the doors.

"I might have occupied myself with your paper, Peterson," Snape said in a voice that was deceptively soft and gentle, "if it had contained even one iota of a useful idea. As it was, the unending stream of senseless drivel drove me to seek intellectual stimulation elsewhere, before the infection of its stupidity turned me into the same kind of mindless dolt as its writer. Now, are you going to attempt the assignment, or am I going to take points from Gryffindor?"

Standing there, icy calm, wand in hand and eyes black as jet, Snape radiated more menace than could be accounted for by the mere threat of deducting points. The room was silent, and Peterson backed down. Cold, palpable anger kept Snape's classes under control for the rest of the afternoon.

Snape paced his office in the time between the end of his last class and the beginning of supper. \emph{How dare they! How dare they think my private life is nothing more than fodder for their jokes and gossip! They have no right! It isn't bad enough that I'm jailed up here in the middle of nowhere, with a mad auror thirsting for my blood and an aging nymphomaniac blackmailing me to escort her - not to mention being the focus of McGonagall's guilt and misguided maternal instincts, or Kettleburn's desire for vicarious titillation... Gad! What might Sprout and Flitwick be saying about me behind my back!}

There was a knock on the door, and Snape snapped, "Come in!"

It was Hagrid. "Begging you pardon, Professor, but I come to see how ya was doing. Seems there's some concern about yer, uh, mood."

"So they've gotten to you, have they? That didn't take long. I presume the owlery has been busy all day, too. Maybe they should take out notices in \emph{The Daily Prophet}."

"Yes, I can see where they might get the idea ya was bitter."

"I feel like a bug stuck on a display tray with a pin through him. Well, what I do in my free time is none of anyone's business including yours. Is that what you came for?"

"If it's any help, ya 've always been the subject of some speculation."

"What!"

"Well, for that matter, so have I. Where I come from, and why I'm so big. Flitwick - they try to figure out why he's so small. Trelawney - they've made up a whole history 'bout her `tragic' life. They got so they almost believe it, too. She's the kind it's easy to imagine things about. You, y're a natural, being so close t' their ages and all. And the whole Death Eater thing."

"Why do they have to gossip at all?"

"Boredom. Stuck out here away from family, friends. Ya got t' get yer entertainment where ya can find it. Happens you and Professor Carmichael, well you're hot news at the moment. It'll die down."

"And until it does?"

"Don't let it get t' ya. It ain't the first time, it won't be the last. And you ain't the first nor the last neither. It's just that with you..."

Snape spun around at the hint of mirth in Hagrid's voice. "What about me?"

"Y're such an easy target, lad. So stiff, so straight-laced. They get a chance t' pull you down to earth, make you seem human and fallible, well they're going t' take it."

"Not if I can help it they won't. There's nothing human or fallible about me. No chinks in the armor. They start making me the butt of their jokes, especially those monsters from Gryffindor..."

"Now y're playing right into their hands. Y're letting them rile ya. Got t' stop that, lad. Teenagers are like sharks. They smell blood, and they'll tear ya t' pieces. Don't let 'em smell blood. Cool, that's the ticket."

"Right. Cool. Cold as ice. They aren't important and they don't exist. What do I do about Carmichael?"

"Rumor is she's got a ring through yer nose and she's leading ya t' slaughter. You got your own reasons for keeping company with her, but ya could show a mite of independence in front of the students. Keep 'em guessing. Show 'em she might be exaggerating when she says she's calling the shots."

"Is that what she's telling them?"

"Don't know if she is. That's what they're saying. Might just be 'cause she's older and you're... well... not in her league when it comes t' experience."

"They know that, too!"

"Don't nobody know nothing. It's just talk. Ya handle it right, and ya can squelch it. Just don't hide from them."

"Meaning you want to be sure I go to supper and eat something."

"I always knew ya could see right through me, Professor."

There was something in what Hagrid said, so at first Snape behaved quite normally at supper, acting pleased to see Dr. Carmichael and chatting with her about unimportant things. Then he excused himself to go and speak for a moment to Professor Flitwick. The moment stretched to nearly ten minutes, until Carmichael began to look annoyed. Snape returned then to resume their conversation, but made a point of including Professor Kettleburn, making it clear that Carmichael was not controlling whatever relationship might exist between them.

At the end of supper, as they were rising to leave the Hall, Flitwick came over with the cribbage board. "Fancy a game or two?" he asked, and Snape agreed. They stayed in the Great Hall for the game, and Carmichael left in a huff. Snape glanced once at McGonagall, and she seemed pleased.

A couple of hours later, when Snape and Flitwick went into the staff room, Carmichael was there. She came over, and Snape didn't try to avoid her. "You are still taking me to London next Thursday," she said, making it sound like a question, though both of them knew better.

"Of course," Snape replied.

Dinner the following Thursday was at a restaurant that specialized in seafood. "Now," Snape told Carmichael, "you can have something safe like fish and chips, or you can be as adventuresome in your food as you are in your life and order the platter."

"What's on it?"

"Shrimp, clams, mussels, squid, and abalone."

"Sounds terrible."

"Fish and chips, then?"

"No. I'll take the platter."

The dinner turned out to be amusing. Dr. Carmichael kept her voice at a suitably low level as she explained in excruciating detail why clams and mussels (crabs and lobsters, too, for that matter) had to be alive at the moment of their cooking. This didn't upset Snape at all, since he already knew the sordid details, and they didn't bother him. Then she pronounced the squid the equivalent of tire tread in texture and insisted that Snape take hers - a bonus for him since he liked them very much. An exposition on mussels followed, comparing the black shells to the souls of Death Eaters which, on opening and exposing, revealed a much more attractive soul inside.

"Has anyone ever cracked your soul open, dear?" Carmichael asked.

"I don't crack easily," Snape replied.

The movie, however, was entirely for Snape. He completely forgot that Carmichael was there. It took time to adjust for the fact that most of the audience seemed already to know the characters, and that there was a preexisting situation that he was supposed to be aware of. Then Snape noted the close-up of the book spine that said \emph{Moby Dick}, wondering for future reference if it was a real book, and entered into the story.

The most fascinating aspect was the idea that the molecules of a sterile world could be rearranged to form a new world that contained the seeds of life. This, combined with the characters' habit of disarranging and rearranging their molecules to travel long distances had Snape thinking of the whole question of transfiguration. \emph{I wonder if Professor McGonagall knows she's temporarily rearranging molecular structure?}

Obsessive revenge and heroic sacrifice combined on screen with magical present and scientific future to form a `reality' in which muggles and wizards might work together. It was a revelation. The film makers were not, of course, thinking of magic when they made the movie but Snape, watching it, was.

Afterwards, walking back towards the Leaky Cauldron, Snape tried to explain it to Dr. Carmichael. "Everything we do has its counterpart in the scientific world, even if for the muggles it's still speculation. Potions is chemistry. Transfiguration is molecular physics. There's a meeting place, a point of mutual understanding. In the muggle world that point still inhabits the realm of science fiction, but that could change any day..."

They reached the little alley behind the Leaky Cauldron, Snape still rapt and enchanted by the movie and, as they prepared to apparate, Carmichael twined her arms around his neck and started to kiss him.

Snape instinctively jerked backward, eyes wide as a startled deer, and instantly shut down, barely registering the blow as his head struck the stone wall behind him.

Carmichael rubbed her knuckles where they, too, had abraded against the stone. "Well, that was hardly subtle. I guess I don't need to ask you your opinion of my feeble attempts at seduction. I don't think I've ever been rejected quite so emphatically before."

"You took me by surprise."

"And you interpreted it as an attack. You've got to lighten up a little."

"I'm sorry. I didn't mean to offend you."

"The man repels any closeness from me and has the gall to hope I'm not offended. Am I so repulsive?"

"No, no, don't think that. You're not repulsive. It's me. I'm not ready for this."

"You're twenty-two. When are you going to be ready?"

They apparated back to Hogsmeade in silence, and in silence went to the staff room for a nightcap. Carmichael once again poured her own drink and brought one to Snape. The heads of houses did their rounds a half hour later, and then they all went to bed.

Two days later, on Saturday, Snape asked to speak to Dumbledore.

"I take it you do not reciprocate her affections," said Dumbledore calmly as he handed Snape a cup of tea.

"Not in the slightest. It's like being with your mother, or one of your teachers. It just isn't right."

"Hardly something that Valeria would understand, I fear. I trust she has not yet found out that Eileen was at Hogwarts at the same time she was."

Snape paused in the stirring of his tea. "She's the same age as my mother?"

"I believe there was a five year gap. The older one gets, of course, the less important these minor age differences become, but if you feel awkward being with someone of your mother's generation, there is nothing that is going to swing you to the opposite viewpoint."

Snape considered this for a moment. "No," he said finally. "Nothing is."

Dumbledore sat at his desk, hands steepled in front of him. "Then - excuse me for asking - why do you take her to London every Thursday?"

"She insists on going. I wish I could find someone else for her to be interested in. I don't even know why she picked me. I'm not exactly the best looking, and even she's hinted that it isn't my personality."

"It is youth, I fear. And a narrow field. It was she, alas, who decided it would be best to reside in the castle instead of commuting. I could hardly insist, as she has been in America all these years and did not really have a British home to commute to. And it has been wonderful for the students. This year's NEWT classes are already a little ahead of where they should be, the OWL candidates are performing well at level, and the lower years are catching up rapidly. And I have interviewed the students. The progress is real. So in that sense, her being here is a great boon to Hogwarts, but it keeps her caged. Kettleburn and Futhark are married. Binns is a ghost. Single live males, in descending order of age, are me, Flitwick, Hagrid, Filch, and you. You are the only one younger than she is."

"My rotten luck."

"Pardon me, Severus, but why do you allow her to insist? You do not have to go."

Snape looked down at his hands. Very softly he said, "She's pointed out that by showing an interest in older women I'm relieving parental concerns that I might have an interest in... younger women."

"None of our parents has expressed any such concern. I have seen no evidence of a problem." When Snape didn't reply, Dumbledore continued. "Has she threatened to make it a problem?"

It was a question Snape didn't want to answer, not with Carmichael talking to Moody. There was a saying about Furies and a woman scorned. Carmichael, \emph{The Daily Prophet}, and Moody were a combination that could get him arrested again. "It's nothing I can't handle," was what he said to Dumbledore. Then he looked up. "I know it's none of my business, but she told me about the problem with Professor McGonagall's son. If it's true, I think the antagonism between the two of them is making my problem worse."

"That the two ladies are battling each other over your soul? It is a possibility. Minerva may see history repeating itself in this situation."

"Professor, is it true that Professor McGonagall's son was trying to escape from his mother's control?"

"Is that what she told you?" Dumbledore thought for a moment, then spoke carefully. "Marcellus did feel at the time that he was being pressured into a career he did not want. And he was quite taken with Valeria. Minerva, for some reason, considered it a personal attack against herself."

"Professor, when did the... incident... occur?"

"Let me see... Marcellus started first year just before Minerva began teaching, which was December 1956... It began in the autumn of 1962, I believe."

Snape left Dumbledore's office still uncertain how to handle the Carmichael problem, but with a disturbing piece of information. Carmichael had spoken of a change in teaching staff in her seventh year and a new teacher whose inexperience had left Carmichael unprepared to pass her NEWTs. Could Professor McGonagall have been that inexperienced teacher? And if so, was it possible that Carmichael had influenced McGonagall's son as revenge for having had her own career plans destroyed? He could think of no way to uncover the truth of the matter, but the situation was looking far more complex than it had an hour earlier.

The next several weeks saw Snape and Carmichael at a variety of restaurants and movies. Carmichael seemed to have decided that experimenting with new types of food was a way to show her trust in Snape, and to bring them somehow closer together. By the end of November she'd sampled escargot and pronounced it `better than I expected,' and had developed quite a taste for Moroccan food.

McGonagall, on the other hand, was becoming more and more agitated. On several occasions she pulled Snape aside to warn him that Carmichael was dangerous. She never allowed an opportunity to pass without making some kind of snide comment about Carmichael's relative age, and the two women were constantly sparring.

December progressed, and the Christmas break began. With the student population down to nearly nothing, the resident teachers, reduced to a skeleton staff, would have a lot more time together.

On the morning of Sunday, December 19, the students left. The staff was having an afternoon Christmas party, and then the nonresident teachers would leave to be with their families for the break. Before noon, Hagrid and Flitwick had finished the decorating - hauling in and trimming the trees, and hanging the mistletoe.

\emph{Saturday, December 25, 1982}

The Christmas party started with lunch at noon. When the meal was over, the few students who'd remained on the grounds went to their common rooms, giving the teachers a chance to fraternize. It was a pleasant few hours. Normally during term they hardly ever spoke. At meals they were generally in their assigned places keeping an eye on the students, and so Snape, at one end of the table, never had a chance to talk to Sinistra or Dawson at the other end. Both of them commuted, so he never saw them in the staff room either.

Professor Dawson in particular wanted to pick Snape's brain about banks. She taught Muggle Studies, and had somehow heard about the cash machine and the long-term deposit, and the two of them spent nearly an hour going over Snape's whole experience at Barclay's. Snape didn't pay much attention to the others until Dawson said, "I think I'm appropriating too much of your time."

Snape looked around. Dr. Carmichael was watching them and seemed irritated. "I enjoyed our conversation," Snape told Professor Dawson, rose and went to talk with Futhark and Pince. As long as he kept moving, Carmichael didn't seem to care, but as soon as he devoted too much attention to one teacher, especially if that teacher was female, she started circling.

McGonagall was circling, too. Twice she pulled him away to ask a question just as Carmichael approached to claim him. It would actually have been amusing if Snape hadn't been the focal point of their rivalry. He wondered how many of the other teachers noticed and were hiding laughter. He tried to pretend that he didn't notice and to act as if everything was perfectly normal.

Around three o'clock, Carmichael finally cornered him. She began talking about a Christmas she'd spent in the Himalayas, and as she talked she edged closer. Snape, uncomfortable with her nearness, shifted his own position several times to maintain a more neutral distance. He was totally unaware of what she was doing until Carmichael suddenly exclaimed, "Oh, look! We're under the mistletoe!" leaned forward, and kissed him in front of all the others.

There was general laughter and applause. Snape, who'd been at a loss as to what to do, had simply stood there and let Carmichael take the lead. The rest of the teachers seemed to think the incident was cute and harmless. Carmichael even smiled as she said, "See. It doesn't hurt. You didn't melt or burst into flames."

Only McGonagall refused to accept that the moment was harmless fun. Storming across the room to confront Carmichael, she hissed, "Leave him alone! Can't you see he doesn't want your attentions? Have you no shame? A woman of your age! It's humiliating for him."

"Dear Minerva," replied Carmichael in a calmer tone, "there is nothing I could do that would humiliate anyone as much as the scene you're creating right now. As for my age, I have always respected your skills in Transfiguration, but I think you lack something in math."

"Lack something, do I? You weren't Hogwarts' only student. I looked up Eileen Prince, and she was still in school when you entered Hogwarts."

Carmichael turned to look at Snape. "She was being serious? Your mother was Eileen Prince? Hufflepuff's gobstone queen?" Snape nodded, fearing the explosion.

"Eileen Prince," McGonagall continued. "Not only are you old enough to be his mother, you went to school with her. Now back off and leave him alone!"

Fire glinted in Carmichael's eyes. She had a rival to defeat and an audience to play to. None of the other teachers went to find Dumbledore for fear they would miss some of the action. Teachers are not so very different from their students, after all.

"You envious, dried-up old prune," Carmichael said quietly. "There's more to age than years. From the moment you were born, you were old enough to be his mother."

"Cradle robber!" McGonagall cried. "I know what you're after. You string them along as tools for your own purposes, your own vindictiveness, and then when they're not useful to you any more..."

"I don't need to string them along. They run to me to get away from you!"

Snape was moving slowly toward the door, not wanting to have anything to do with the battle.

"You're a heartless vampire, sucking the life..."

"You smothered one 'til he had to run in order to breathe, don't smother..."

McGonagall jumped forward, fingernails reaching for Carmichael's face as the staff moved into the fray to restrain her and keep the two apart.

"Ladies, please control yourselves." Dumbledore stood in the doorway, some sixth sense having told him there was trouble in the Hall. "Minerva, please sit down over by the Gryffindor table. You seem overwrought. Valeria, I think you need to relax a bit, too. Pomona, would you..."

"That's all right, Albus. Severus and I were going to London today anyway," Carmichael said with considerable satisfaction.

"Is that true, Severus?" When Snape nodded, Dumbledore said, "Then maybe it would be a good idea if you continued with those plans. It would keep the two of them separate for a while."

"Since there are no students right now," Carmichael added, "there's no need to be back for curfew. So if you don't see us until tomorrow morning..."

"Valeria!" Dumbledore admonished as McGonagall rose from her bench. "Severus, get her out of here."

"Yes, sir," said Snape, steering Carmichael out of the Hall. Behind him he could hear McGonagall saying, "Albus, you can't let her..." and Dumbledore's reply, "Minerva, it is not up to you to control his life..."

They got to London far earlier than they'd intended, too early for dinner, so Carmichael insisted they go to a pub where, starting with a pint of beer and advancing to gin, she regaled Snape with her colorful opinion of anyone who allowed a grudge to last for that long. Snape merely sat and listened.

By dinner time, Carmichael had calmed down to the point where they were able to have a more normal conversation. Deciding this was the wrong time to experiment with new foods, Snape suggested they go to the seafood place they'd eaten at before. There he got her to talk more about the Himalayas, and then they went to the movie.

The movie was a comedy about a man who had to pretend he was a woman in order to get a job. There were moments, many moments, when the actor's portrayal of the female part was so good that Snape had trouble believing it was a man playing the role. \emph{I would love to be able to act like that, to fool people into thinking what I want them to think of me, to step out of my own life.} For the first time he began to think of acting as a skill, a talent, an art.

The movie wasn't a long one, and by nine o'clock Snape and Dr. Carmichael were back out in Leicester Square. "Let's stay in London," she said. "Let's find a place and not go back to that horrid school until morning, and shock them all!"

"I really don't think that's a wise idea."

"How did you get to be so old so young? Where's your sense of adventure? Of fun?"

"I really should get back to Hogwarts. You could stay, though."

"By myself? No, dear, when you go I go. But not this early. Come on. Just into the pub for a couple more drinks."

Snape agreed, if only because he didn't want Carmichael to run into McGonagall when they got back to the school. The conversation over drinks, however, quickly became difficult for him.

"Stay the night with me here in London. The older you get, the harder it'll be. We have the opportunity, we have the time. I could teach you so much..."

"Dr. Carmichael, I don't think this is the proper..."

"Who cares about proper? We're two adults who enjoy each other's company. It doesn't have to get any more complicated than that. There's so much in life that you're missing, good things, enjoyable things."

"Look," said Snape, trying to chose his words carefully, "I like your company. I like to talk to you. I enjoy these evenings here in London, with the restaurants and the movies. And you have taught me a lot because you've done so much with your life, but..."

"Ah, that famous word. But what, dear?"

"I'm really not physically attracted to you. It's probably because I'm too young to appre..."

It was too late. Carmichael's face had closed, set into lines of cold, offended rage. When she spoke, her voice was low, the menace clear. "And for how long have we known that we had no physical attraction for each other? From the beginning? Have you been allowing me to pay for your food and entertainment all these weeks with no intention of reciprocating? Have you been misleading me?"

"No," Snape protested, "I've never said I wanted to be closer to you. The opposite. I've made it very clear that I wasn't ready for any kind of relationship with you."

"You asked me out to dinner and a movie because you didn't want to have a relationship?"

"I didn't ask you - you asked..."

"To see a movie about computers? Why would I invite any man to watch a movie about computers? The movie was your idea. Trying all those strange foods was your idea. Don't blame that on me."

"Look, can't we just forget we had this conversation and continue on..."

"Forcing you to go out with someone you find repulsive?"

"I didn't say that!"

"What's not attractive about me? My age? You like them younger? Say seventeen?"

"No, that's not it at all."

"Or maybe it's the seventeen-year-old boys... that's a fine hobby for a school teacher."

"Dr. Carmichael..."

Carmichael stood and finished her drink in one gulp. "We're going back to Hogwarts," she said. "Now."

Snape followed her out into the dark street, trying to calm her down, but every word he spoke seemed to make her angrier and angrier. It was well past nine-thirty when they apparated into Hogsmeade. By the time they got to the entrance hall it was nearly ten. There were quite a few teachers in the staffroom since some of the commuting staff were still there, and Snape tried to get Carmichael to go to her rooms so that she wouldn't run into McGonagall.

Carmichael, however, insisted on the staff room, and there was no stopping her. McGonagall was in the far corner where the drinks were prepared. Carmichael immediately struck up a conversation with Kettleburn as someone handed Snape two cups of punch. He gave one to Dr. Carmichael, who barely acknowledged his presence.

"You're back earlier than expected," Sprout said at Snape's elbow. "Kettleburn was sure it would be at least midnight, just for appearance's sake."

"Change of plan," said Snape, and then the sound of breaking glass diverted his attention.

Carmichael had dropped her cup on the floor and was holding her stomach. "They've done it again," she gasped. "Gad, they've done it again." She reeled from the room, heading for the girls bathroom, Pomfrey and Sprout with her, the rest of the teachers remaining in the staffroom in stunned silence.

This time both St. Mungo's hospital and the Ministry of Magic sent teams immediately. The staff room was left exactly as it was, and all who had been in the castle or on the grounds were asked to notify their families that they would be delayed in returning home. Snape found himself in the guest quarters on the sixth floor, his rooms once again sealed against him. \emph{At least now there are no students. They'd have a wonderful time with this one.}

Once again Snape waited until nearly the end for his own interrogation. It took a turn that was not totally unexpected.

"You gave her the glass of punch."

"Yes."

"Where did you get it?"

"I was handed two. I gave her one."

"Who gave them to you?"

"I didn't notice."

"How did you know which one to give her?"

"I don't think I understand the question. I was given two. I gave her one. It didn't matter which one."

"Why were you given two?"

"I don't know. I presume because they saw us come in."

"Are you and Dr. Carmichael considered a `couple' by the staff here?"

"I don't think so. We're not a `couple.'"

"You been dating for several weeks."

"`Dating' is a little strong."

"Do you know any other teachers on staff who are keeping company with each other?"

"No."

"So there might be grounds for other teachers to see you as a couple."

"I suppose so."

"Professor, do you brew poisons?"

"Of course."

"Please elaborate."

"It's part of the curriculum. My sixth year NEWT class will be analyzing poisons next term."

"Are those poisons already brewed?"

"No."

"What were your duties as a Death Eater?"

The question took Snape by surprise. "I brewed potions, invented spells, and taught rudimentary self-defense. This is all on record."

"What other Death Eater made potions?"

"None that I know of. I didn't know much about other departments. None of us did."

"Were you ever asked to brew poisons?"

"No."

"Do you know anyone who was asked to brew poisons?"

"No."

"Are you sure you wish that to be your answer?"

Nervous now, Snape replied, "Yes. That's my answer."

"We'd like to ask you now about your most recent date with Dr. Carmichael."

Snape's interrogation had, by this time, lasted two hours and gone into minute detail.

"Where did you go first?"

"To a pub."

"At your suggestion?"

"No, at hers."

"Did you eat or drink anything?"

"She had beer and a couple of glasses of gin. I had a glass of wine."

"And then?"

"We went to a restaurant."

"What did you eat and drink?"

"We had the same thing. Oysters Rockefeller, Caesar salad, and shrimp Alfredo. With white wine. Chocolate mousse and coffee."

"Did Dr. Carmichael seem ill at dinner?"

"No."

"Did she seem ill during the movie?"

"No."

"Did she eat or drink anything during or after the movie."

"Not during, but she had another glass of gin at the pub afterwards."

"Did she seem ill when you arrived at Hogwarts?"

"No."

McGonagall was last, not until Tuesday in fact, and it was noted by the staff that after five hours of questioning she seemed tired and drained. A rumor started that her son Marcellus was to be questioned as well. Then the team was gone, and the staff was notified that everyone was to remain available for further questioning after the Ministry analyzed the evidence.

Then, on the morning of Thursday the twenty-third of December, two representatives of the Ministry of Magic arrived at Hogwarts to escort Professor Minerva McGonagall to London for questioning in the Department of Magical Law Enforcement concerning the poisoning of Dr. Valeria Messalina Carmichael, aka Aurifossor, aka Wolfstone, aka Mulberry, aka Pendennis, aka Buntley. Headmaster Dumbledore was informed that the Ministry was not at that time able to tell him when Professor McGonagall would be able to return to Hogwarts.

Snape was distraught. He'd reached the point where he regarded everything that was happening at Hogwarts as his fault. He hated himself, his naiveté, his bluntness, even the lack of critical thinking that allowed him to hand an untested cup of punch to a colleague. He couldn't face the others with the knowledge of his own guilt, and he had to help McGonagall. And he had to do it without implicating anyone else.

That evening, without telling anyone, Snape apparated to London and went to the entrance of the Ministry of Magic. Granted entrance after some argument, he went to the desk in the atrium and asked to be allowed to speak to Judge Bones if she was still there. They took information about him and directed him to sit in a little waiting room while they checked with the judge.

It must have been a half hour later when a voice, a familiar voice, said, "Merlin preserve us, the Death Eaters are turning themselves in." It was Moody, leaning against the door post at the entrance to the room.

"I didn't come to talk to you," said Snape.

"I know, but she's tied up in chambers, and someone had to come up here and give you the greeting that's your due. You checking in at Azkaban this evening?"

"I'm not talking to you," Snape repeated.

Mercifully, Judge Bones arrived then, accompanied by Gawain Robards. "What did you wish to see me about?" she asked.

"It's about Professor McGonagall. She's been arrested for something she couldn't have done."

Judge Bones looked at him shrewdly. "Let me get this straight," she said. "You came here to try to prove that Professor McGonagall is innocent?"

Snape was puzzled. "Of course," he replied. "Why else would I be here?"

The judge stepped to the door. "Alastor," she called, "I want you in here, too. Now, if you please."

Hating the fact that Moody was present, Snape nonetheless sat where he was told and waited for their questions. Robards was taking notes.

"Who sent you here, Professor Snape?" Judge Bones asked.

"No one sent me. They don't even know I'm here."

"Albus doesn't know? So if we arrest you and send you down to a holding cell, no one will come to get you out?"

"You won't arrest me. I haven't done anything wrong. He might," Snape nodded at Moody, "but you wouldn't."

"Tell me about Professor McGonagall."

"She couldn't have planned to poison Dr. Carmichael because no one knew we were going to be there. Dr. Carmichael stated quite publicly that we'd be back later than usual, maybe not until the next morning. Even if we got in at eleven, the others would have gone home or to bed. The staff room would be empty and there wouldn't be any punch left."

"Are you telling me that you're the only one who knew what Dr. Carmichael's movements were that evening?"

"That's right."

"Why are you protecting Professor McGonagall?"

"I'm not protecting her. I'm telling you what the truth is. McGonagall couldn't have done it because poisoning requires preparation, and she didn't know the opportunity would be there."

"Do you have any idea who did do it?"

"I'm sorry, no, I don't."

Judge Bones looked over at the other two. Robards was thoughtful, and Moody had an unreadable expression on his face. "I'm going to tell him," the judge said, and neither man objected.

Turning to Snape, the judge said, "We're a bit surprised that you came here tonight to speak in Professor McGonagall's defense because she isn't our chief suspect. You are."

"I don't understand," said Snape, too bewildered to be wary.

For answer, Robards handed the judge a small book that she opened to a certain page. "Val Carmichael wrote this more than three years ago," the judge said, handing the book to Snape and pointing to a particular paragraph.

\emph{In the spring of 1979, the forces of the self-styled Lord Voldemort changed their tactics. Since all of their direct physical attacks had failed, they started to use more subtle and clandestine ways to get rid of me. Chief among these was poison. Twice I actually consumed a brew intended to kill me, and on both occasions only the quick thinking of my husband and my publishing agent saved my life. It was this new, cowardly bid to exterminate me that finally led to my most difficult decision - leave Britain and emigrate to the United States. It was the only way to stay alive.}

Snape closed the book. Its title was \emph{Battling the Darkness}, published in the summer of 1979.

"When did you become a Death Eater," Judge Bones asked.

"July 1978."

"And when did you become Voldemort's potions brewer?"

"January 1979, but..."

"Yes?"

"I wasn't asked to make poison. No one ever mentioned attacking the Dark Lord's enemies with poison. I never even heard of Dr. Carmichael when I was a Death Eater."

Judge Bones sighed. "Do I have to remind you, Professor Snape, that you already have a well-established reputation for concealing vital information from this Ministry. And may I also remind you that in an earlier interview you stated that you had been asked to brew poisons four times. Why should we believe you when you can't even keep your story straight?"

Snape looked down at his hands. His voice became very small. "Are you going to arrest me?" he asked, and it was clear he expected the answer to be yes.

The other three exchanged glances, and then Moody got up and left the room.

Robards leaned forward. "No, we're not going to arrest you. And we haven't arrested Professor McGonagall either. Dr. Carmichael tells us you both have a motive for trying to harm her, but that by itself isn't enough."

"What's `enough?'"

"I can't tell you that, but believe me, when we have it, you'll be one of the first to know."

The same words coming from Moody would have been a threat, but somehow Snape knew that from Robards they weren't. Robards had done his job, and done it devastatingly well, at Snape's trial, but he'd never gone beyond that, never harassed or threatened. "May I go back to Hogwarts now?" Snape asked.

"Certainly," Robards said. "And thank you for your information. If you wait about fifteen minutes, though, you can go back with Professor McGonagall."

"I think I'll do that," said Snape.

It was closer to twenty-five minutes before a clerk escorted McGonagall into the little waiting room. "They said you were here," she snapped, "but I thought it was a trick. What ever put the foolish idea into your head to come here now?"

"I wanted to tell them you couldn't possibly be guilty."

"I could tell them that myself, and did." McGonagall retorted, fussily straightening his collar, "so there was no reason for you to go getting yourself into more trouble. Now, take me back to Hogwarts."

"Yes, ma'am," Snape replied, and together they left the Ministry.

Carmichael left the school the next day to spend Christmas week with her publishing agent's family, so the atmosphere was much more relaxed for the holiday. It was just Dumbledore, the four heads of houses, Hagrid, and Filch, with a handful of students. They spent the day relaxing, chatting, playing cribbage, and never mentioned Carmichael once.

That changed on Monday, the twenty-seventh. Snape came out of the dungeons on his way to brunch to encounter an enraged McGonagall who waved a copy of \emph{The Daily Prophet} under his nose. "Look at this!" she ordered him. "That woman deserves to be in jail!"

Snape took the paper with a feeling of dread. On the third page was an interview with Val Carmichael in which she talked about her career, her return to Britain, and her teaching at Hogwarts.

"There!" said McGonagall, pointing to a section two-thirds of the way down. "Read that."

\emph{Reporter: It must be pleasant for you to return to your native land without the threat of death hanging over you. Do you find your life more relaxed now that You-Know-Who is gone?}

\emph{Carmichael: I did at first, but I've come to realize that his fear and hatred of my opposition to him stretch beyond the grave. There are still Death Eaters roaming free who are bound to fulfill their master's will, and twice now I've been targeted for execution. Luckily the attempts were unsuccessful.}

\emph{Reporter: It's amazing under the circumstances that you can remain so calm.}

\emph{Carmichael: The Ministry of Magic is investigating the incidents. I have great faith in them, and expect an arrest very soon.}

"So that's her motive for me, that I'm still obeying the Dark Lord. At least she didn't mention my name or that it happened at Hogwarts."

"With all the scandal last year about your past, do you think it's going to be hard for some people to guess? Why is she switching her attack from me to you, anyway?"

"I turned down an offer. I told her I wasn't attracted to her."

McGonagall snorted. "And this is her revenge. Well, something good is coming of this, at least."

"What's that?" Snape asked.

"She's not going to be wanting to go on any more dates with you."

The next day \emph{The Daily Prophet} printed a retraction, apologizing for having interfered with an ongoing Ministry investigation. Snape would have to wait until the start of the next term to find out what, if any, damage had been done.

\end{document}
