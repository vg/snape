% A Difference in the Family: The Snape Chronicles (Rannaro)

\chapter{School Days}

Nineteen-sixty-four was also the year Russ's family began talking about school.

"He'll be going to Hogwarts." Nana was saying. "Eileen was the first of the family to go, but her son should be admitted as well. He's shown he's magical. They can't refuse him." Eileen was Russ's mum, and Nana was her mum, and Nana almost always got her way because she was the most powerful witch around. Russ had not yet come to grips with the fact that Nana and Mum were the only witches around. He thought being the most powerful was pretty good.

"Right," said Dad, who almost never agreed with Nana. "Can y' see me telling m' mates that my son goes to some school called Hogwarts? Y' know what they'll all say."

"You would impress me more, Tobias, if you were worried more about your son's education and less about what your `mates' would say."

"I don't see 's it makes much difference right now anyway," continued Dad. "He can't start that la-di-da school 'til he's eleven. We're talking about right next year, when he's five."

"He should," Nana said, "be home-schooled."

"She's right, Toby," said Wenny. "The boy needs to be prepared for Hogwarts. A primary school isn't going to give him that." Wenny was Dad's dad's dad. Dad didn't have a dad because he was dead—killed in a mine cave-in. Russ knew a mine was where his dad worked, a cave-in was a bad thing, killed meant making someone dead, and dead was when you had to go away and you could never come back even though you wanted to. Mum's dad was dead before Russ was born. He had no memory of either of his grandfathers.

Mum spoke next, and she agreed with Dad. This made Russ happy because when Mum agreed with Dad, Dad was usually in a good mood about it. "No, Wensley," she said, "Toby's right. Who's going to home-school him? I can't. I have to look for jobs. We ain't got the money to pay for it. Are either of you going to pay for it? If you aren't I don't see as you've got any say in the matter."

They were always talking about money. Russ didn't understand a lot about money. He had a collection of five coins: a farthing that used to be able to buy things but couldn't anymore, a ha'penny, a penny, a tuppence, and a thruppence. The other coins were too important for him to have, they needed to be spent. He figured home-schooling must cost a lot of sixpences if none of the adults had the money for it.

The school talk ended. Wenny bent down where Russ was drawing with a piece of charcoal on a scrap of butcher's paper. "You want to come spend the afternoon with me, Severus? I think your mum and dad want to be alone." Wenny was leaning on a cane. He used a cane to walk with, which Russ thought was neat since Wenny's canes always had interesting figures on them. This cane had a dragon's head. It came from a place called Wales.

Russ nodded and got at once to his feet. Whenever he visited Wenny because his parents wanted to be alone, his Dad was always in a good mood when he got back. He also loved going to Wenny's house because it had so many interesting things. He scampered to get his jacket and cap, then ran out the door with a quick "Bye, Mum. Bye, Dad."

Walking down the street with Wenny was fun, too. Men Russ didn't know would tip their caps and say things like, `Afternoon, Cap'n.' Sometimes they noticed Russ. `Is this the young 'un?' they'd ask, and Wenny would say, `Aye, m' great-grandson. Greet the gentleman, Severus.' Then Russ would hold out his hand and say solemnly, `How do you do?' and the men would tip their caps to him, too.

Wenny lived a short ways outside the town, in an old, old cottage with an overgrown, rambling garden. Nana's garden was neat and orderly, and she told him the names of everything and what it was good for, but Wenny's garden was wild and full of things with no names. Russ could pick and examine anything he wanted, as long as he didn't put it into his mouth. He loved both gardens, but Wenny's garden was more fun to play in.

Even more fun was the inside of Wenny's house. It was full of things that nobody else had. There were real human heads so small they'd fit in your hand, and dolls that if you stuck pins in them they could make people sick. There were pig knuckle bones and painted cards that told the future, and blowguns with poison darts that Russ couldn't touch because they could still kill you if you pricked yourself. There were statues with eight arms, and snake skins, and a dinosaur claw. There was the dried-out eye of a creature that lived on the highest mountain in the world, a vial of dirt from a vampire's grave, and the tooth of a man-eating shark. There were drawers and chests full of these things, and Russ loved them all.

The best were the books because they all had pictures. Russ couldn't read yet, but he knew which pictures were the vampires, and which the zombies. He could recognize werewolves and ghouls, harpies and gorgons, banshees and dragons, gremlins and basilisks, the Cyclops and the minotaur, centaurs and satyrs. He could lie for hours on his stomach poring over one of the musty old volumes that smelled of salt and mildew.

Russ was, in fact, not yet five years old, yet he already knew more about herbs and potions, dark arts and magical creatures, than any other student who at the age of eleven had crossed the lake to Hogwarts on his way to being sorted.

\subsection{Wednesday, September 1, 1965 (day before the first quarter)}

Russ stood in front of his mother and father on the first day of school dressed in his brand new school uniform. It was too big for him. This was partly because there were no premade uniforms his size, but partly also because his parents expected him to grow into it.

The uniform was short gray pants and white shirt, a dark blue tie and blazer, a gray cap and socks, and black shoes. All these garments hung loosely on Russ, giving him a scarecrowish appearance. He didn't realize this. His mum and dad were proud.

"There he is," said Dad, almost teary. "M' son's going off to school."

"Now," Mum asked, "What are you supposed to remember? What's your name?"

"Richard Severus Snape."

"Where do you live?"

"End of Spinner's End, down by the mill."

"Parents?"

"Tobias and Eileen Snape."

"Dad's work?"

"Collier."

"Your numbers."

"One, two, three{\el}" he rattled them off all the way to a hundred.

"Alphabet."

"A, B, C{\el}" that one was easy.

"All right, Russ, let's go."

Dad went off to the mine while Mum took Russ's hand and went with him to the school building at the center of the town, a half mile away. Just as they got to the bridge over the river, Mum bent down to adjust Russ's tie.

"And what's the most important to remember?" she asked quietly.

"Don't make nothing happen," Russ replied. It was a rule he'd learned to follow so long ago he couldn't remember. Never make things happen where people could see.

They crossed the bridge and went up the hill toward the school. Other students were going, too, some of them with their mums. Russ was excited and a little bit scared.

Mum left Russ in a big room with desks and tables and chairs. The teacher, a tired young woman with curly blonde hair who introduced herself as Miss Donnelly, showed him where to sit, and soon the room was full with nervous five-year-olds. Russ hunched down in his seat because he didn't want them to look at him. Most of them had uniforms that fit.

The teacher began to call names and to ask the children questions. Some of the questions were easy, like `What do you call the big light in the sky?' or `Count from eleven to twenty.' Some were harder, like `Tell me the names of three animals with four legs.' Russ knew lots of names of plants, but his acquaintance with nonmagical animals was almost nonexistent. Then the teacher called the name of a student who didn't answer.

"Richard?"

The children looked around. Russ looked around. Richard wasn't there.

"Richard?" the teacher said again, then she stood up and walked over to Russ. "Richard Snape? That's you, right?" The class giggled.

Russ looked down at his hands, mortified. "Y's 'm," he muttered, hating himself for not remembering his full name.

"Good, Richard. Now tell me, what country do you live in?"

Russ thought. His parents hadn't given him the answer to this one. "Pendle," he said after a moment. The class giggled again.

"It's England, Richard," the teacher said. "Do you know what the capital of England is? Capital means a large, important city."

Russ thought hard about cities, trying to remember a name. One came. "Blackpool," he answered. More giggles.

"Don't worry, Richard," the teacher said. "We'll have time to learn about London."

It was a terrible day. Russ was confronted time after time with things he did not know. He didn't blame or resent the teacher, he blamed himself. He blamed himself for being too stupid to know these things that everybody else knew. There was something wrong with him. Nana and Wenny were right. He should have been kept at home because he was not good enough to go to school.

When Mum came to pick him up, Russ was silent and miserable.

Russ remained silent after they got home and Eileen got him a glass of water and a piece of bread for tea. "How did it go today?" she asked him finally, sitting beside him at the kitchen table with her own cuppa.

"Okay," Russ answered glumly. He didn't know the name yet for the feeling of shame inside him, but it was new and unpleasant. He didn't want his mother to know about it. She'd been so sure he would do well.

"It doesn't sound so good," said Eileen. "Let's have a look-see." She leaned forward to bring her eyes closer to his.

Russ panicked. For the first time in his life, he did not want to show his mum what he was remembering, what he was thinking. He wanted her to be proud of him, and if she saw, she wouldn't be. Suddenly, not by his own effort, but by a kind of reflex, like pulling your hand away from something hot, he was remembering a half hour in the late morning when he was drawing a picture. That had been nice.

"Oh, drawing," Eileen said. "What did you draw?"

"Nana's garden with the flowers," Russ replied. He'd wanted to draw Wenny's shrunken heads and voodoo dolls, but he didn't know how. The teacher 'd liked the flowers.

"Show me something else."

Russ found that if he left out the bad parts, there were things he could let his mum see. The teacher reading a story, for example, and the song she wanted them to learn. Children running around the play yard. He pushed his own failure down into a place where she would never look for it and{\el} he didn't really understand, but it was like the kitchen door out to the area yard that if you didn't latch it, the wind could blow it open. Russ latched it.

By the time his dad got home, Russ had the story ready. He told all about the drawing, and the music, and the class learning ABC together, and Toby was satisfied.

"Mum, what's a country? Is it like Nana's house?" Russ asked later while Eileen did the washing up.

"What? No, Russ. That's a different word. When we say Nana lives in the country, we mean the countryside. Out of town where there's no other house but Nana's, and all's moor and open land. A country is a big place with lots of villages, towns, farms, and cities inside it."

"Is England a country?"

"That's right. We live inside England. Remember last month when we went to Blackpool with Gra? All the time we were driving, and in Blackpool, we were still in England. We could drive for hours, and we'd still be in England."

The next day in school, the teacher didn't ask about England or London, even though Russ now had the right answers to give her.

As the days and weeks passed, school became at least predictable, even though it continued to be a torment. Russ soon discovered that almost anything he did made the other children giggle, and he hated being called on for any thing. Even if he knew the answer, he couldn't get it to come out of his mouth properly, and he would say things like `L{\el} London,' or `I don't{\el} know.' That was really funny for the others.

Play time was good because the other children ran off to play games, and he could find a quiet place to sit and think. Wild flowers poked their way up through the cracks and around the edges of the play yard, and Russ found old friends—pimpernel, heart's ease, and toadflax.

Books were another good thing, and Russ learned to form the letters into words, and to add and subtract, and more about England and Lancashire. He learned to look at a globe and find his own country, and how the sun made day and night while the globe turned, and about temperature, and that plants make their own food.

On the bad side, he learned that he was poor, and that the part of the town he lived in wasn't a nice place for the other children to go. Most of the families with small children had left his neighborhood when the mill closed, so there weren't too many others from his area, and everyone could tell where he came from just by looking at him.

And he knew that the one thing he must never, never talk about to anybody was about witches, wizards, magic, or Hogwarts. They were all muggles, and they wouldn't understand.

After the first week, Eileen stopped taking Russ to school or bringing him home. He knew the way, and she needed to work or they wouldn't have enough money for food on the table. Russ understood that money was important, so this didn't bother him. Besides, this was when Russ started exploring.

The most important thing about exploring was not to stay on the same side of the river as the school. If you stayed there, people stared at you and warned you off because they could tell you were from the other side. On the school side there were flowers in the yards, and the mothers didn't have to work, so they had time to keep their curtains clean from the dirty air. On the mill side, it was different.

The river had a stone bridge wide enough for a car to cross on it. The river went past the old, closed mill, and the water smelled bad. Russ wasn't supposed to go in it, or drink from it, or even touch it. People threw things there like it was a long, wet rubbish bin.

There were places along the bank, though—mostly on the school side—where scraggly trees grew. It was nice to sit under a tree on a quiet afternoon as long as no one saw you. There were a few children at school who lived on the mill side of the river, all older than Russ. Russ didn't like them to see him because they laughed at his badly-fitting clothes, and they all knew his dad spent too much time at the local. Besides, Russ had learned that you didn't want people talking and looking at you too much because if they looked in your eyes, they could steal your thoughts. Russ's mum could, and Russ now assumed other people could, too. It was okay if it was your mum.

At first it was easy to get lost on the mill side because all the streets were the same. All the houses were grayish brown brick covered with black soot. All the cobblestones were cracked and broken. All the streets and sidewalks were narrow, with gutters running down the center, and there were no trees or flowers anywhere. When Russ started noticing which houses had boarded up windows, it got easier to remember which street he was on.

Soon Russ knew all the important places. There was a shop where his mum bought tea and sugar, and a bakery for bread, and the butcher's shop. They were all small and didn't sell many things. Russ knew from walking through the school side that the shops there were bigger and had more things for sale. This didn't bother Russ because he knew they were different, and he didn't question that this was the natural order. It might have been otherwise if he'd suffered real want, but for all his dad's complaint about putting food on the table, Russ had never been truly hungry. He was small, and didn't eat much. There were even rare occasions when his dad would bring home fish and chips wrapped in newspaper on an Friday evening. Life had its pleasures.

Russ also knew where the pub was. He had to be careful his dad never saw him there because Russ wasn't supposed to be mucking around in the street after school, so if his dad stepped out of the pub, Russ had to run home like the dickens to get there before his dad did. His mum would look up from cooking, tired after a day charring or laundering, and say as he raced into the house, "He's on his way, is he? Good thing supper's nigh ready."

After a month of exploring, Russ discovered the old mill. He had a vague memory that a long time before, maybe a year ago, his dad 'd worked at a place called the mill. That was before the mill closed and everyone had to work at another place called the mine. The mill was all boarded up and surrounded by a fence, but Russ found a place in the fence where it was broken and he could squeeze through. He started prowling around every day after school trying to find a way into the building. He didn't find it because something else happened.

Suddenly, in the third week in October, all the mothers were shepherding the children more closely than usual. Children who'd walked to school on their own now came with a parent. Women talked to Russ's mum in the evening, and she walked him to school as well. The teachers patrolled the play yard at play time, and when Russ wanted to go off in a corner by himself, he was told to stay close to the others. Gossip among the second and third year students was frightening.

"They did bad things to her, and then they killed her and buried her on the moor," was the general story, and some of the boys demonstrated how you could be strangled. On Thursday the body of a boy was found on the moor, too, and Eileen lectured Russ about not talking to or taking rides from strangers. The deaths were in Manchester, not in Pendle, but one never knew. The world was a dangerous place.

One aspect of having to stay closer to the other children during play time was that several of the older ones had little battery-run radios, and Russ could overhear some of the songs and listen to the students talk about the singers. He never did it in school, but at home he started singing some of the words. Those he knew, at any rate.

He wasn't a good singer, so the first time Russ did "Help! I need somebody! Help! Not just anybody!" Eileen came running thinking he really wanted assistance. "Hey! You've got to hide your love away!" was another favorite. Over and over again.

Russ wanted his hair longer, too. "No son o' mine is going t' have hair like a girl!" Toby insisted, but eventually he had to give in because so many of the young people had been growing their hair long for more than a year now, and he wanted his son to be `normal.'

Shortly after Christmas, in spite of frantic hand waving and stop signals from Toby, Eileen asked if Russ wanted to invite some friends over for a birthday party. "Nah," Russ answered, "bunch o' stuffed shirts," a remark that made Toby tousle his hair and say, "That's m' boy!"

In fact, no one was much concerned that Russ didn't make friends at school. He was the son of a poor miner who had to travel to another town for work, and he came from the side of town where boarded up windows and lifeless streets were signs of the decay that had set in after the closing of the mill. It would have been more surprising if the boy had shown signs of wanting to form friendships outside his social class.

The summer of 1966, Russ went to spend several weeks with Nana. He was six and a half now, and his fingers were much more controllable than they'd been when he was five, so Nana set him to weeding and pinching off flowers in her herb garden. At first he tried witching them out of the ground, but Nana told him not to use magic around her potions herbs, so he stopped. Then he also had to pick off caterpillars and aphids, and anything else that liked to eat leaves or suck juices. The pollinators he was supposed to leave alone. Nana didn't believe in using nasty sprays.

"Don't be afraid of that bee, child!" Nana would call to him across the garden. "It won't sting you unless you force it to. A sting 'll just hurt you for fifteen minutes, but it'll kill the bee. Move slowly and give it time to clear out."

That was the time of Russ's first encounter with stinging nettles, too. At first he thought it was a bee, until Nana checked, found no sting, then saw the plant he'd touched. "We can have nettle soup tonight," she said, and laughed at Russ's expression. "Don't worry. Nettle soup is good and healthy, and cooking takes away the sting. In fact, if you grab it instead of just brushing against it, it hurts but not as much. Sometimes if you want something, you have to be willing to let it hurt you for a bit. It depends on how bad it hurts, and how much you want it. You'd better wear gloves, though, if you're going to be pulling nettles." The nettle soup was delicious.

Nana had an tawny owl named Nelson. Nelson was older than Russ by a year, and Nana warned Russ that if he wasn't careful Nelson would live longer than Russ would. Russ thought this was funny until Nana explained that a tawny owl residing with a wizard family could easily live to be twenty or thirty years old. "I know of wizards dead before thirty because they weren't careful with their spells."

It was Nana who now started teaching Russ about magic in earnest. Russ's mum couldn't do it because first his dad was uncomfortable about magic, and second because they lived in a place where there were too many muggles. You weren't supposed to do magic where muggles could see, unless like Gra they were members of the family, and so Mum just never used it. She told Russ she'd never been that good at it anyway.

Nana was good at it. She got out her husband's old wand and showed Russ how to hold it. "Mum says I'm not supposed to use wand magic," Russ told her. "I'm too young."

"Your mother picked up some strange ideas in that school of hers," Nana retorted. "And the Ministry's a bunch of officious busybodies. You're in my house, and nobody can tell if it's you or me doing it. A wizard should start his magic young. How else is he going to be good at it? Now you hold this like I showed you, and you're going to learn how to fix something that's broken. It only puts pieces together, so if you don't have all the pieces, it won't be properly fixed, but if you use it the moment you break something, then it's fine. Just move the wand downward 'til it points at the thing and say the name of the thing you want fixed and \emph{Reparo!}"

That was Russ's first introduction to magical language because you couldn't just speak English. If you wanted to fix a bottle (he practiced on a bottle that he could break and repair over and over again), you had to say \emph{Ampullam reparo!} because magic for bottle was \emph{ampulla}. And you had to know that you couldn't use it on living things because they didn't `fix' the same way.

"You're better at this than your mother was," Nana said thoughtfully after he managed to fix the bottle several times in a row. "I'm not surprised. I never knew a Rossendale or a Prince who wasn't good at magic until your mother came along. Maybe it just skipped a generation."

"What's a Rossendale?" Russ asked.

"I am. My name was Constantina Rossendale before I married Richard Prince, and then I became a Prince by marriage."

"Am I a Rossendale, too?" It was an interesting concept.

"You certainly are. A Rossendale and a Prince, just like your mother." Nana fingered the wand. "This works pretty well for you. Maybe when you're older, you can have it."

\subsection{Thursday, September 1, 1966 (one day after the full moon)}

Life became more tense that fall. The most important factor was Russ's dad. Toby suddenly hated Americans. It wasn't really about the war in that country south of China that Russ had trouble finding on the globe, even though his dad was always saying how the Americans shouldn't be there. No, the real problem was coal. Russ had the idea that if it weren't for the coal, his dad wouldn't care about the war.

The Americans were selling their coal cheaper than the English were selling their coal. That meant everybody wanted to buy American coal instead of English coal. If nobody bought English coal, then Dad would lose his job. If the English sold their coal cheaper, then the mines wouldn't have as much money, and Dad would still lose his job. To make matters worse, somebody in London was talking about joining Europe. If England did that, then the Germans would sell their coal in England, and Dad would lose his job.

Toby Snape was getting drunk more often now, and it wasn't the jolly kind of drunk. He was coming home roaring with rage against the world, furious and frightened, and striking out at fate. The first time he hit Eileen was the night after the first day of school. He staggered late into the house calling for Russ. "Where's m' boy! The whol' worl's 'gainst a man, but 'e's go' 'is son t' comfort 'im. Bleedin' 'ope f'r th' future! Russ! Come sit wit' yer dad!"

Eileen tried to stop him from going up the stairs, where Russ had already gone to bed in his room. "Get yer 'ands off me, woman!" Toby yelled, and punched her in the shoulder, sending her back against the wall. He advanced up the stairs bellowing "Russ! Get out 'ere!"

Russ had been startled awake, and came out of the room, his father's drunkenness by now a matter of common occurrence. One look at Toby's wrathful face, however, and he shrieked in terror and darted back in, grabbing the door and trying to shut it. This only infuriated Toby more, and he lunged for the door, thrusting it open and seizing the boy by the upper arm. "Shut me out, will ya, ya witch's brat! Where's th' magic when a man needs it? Laughing at me, both o' ya, but y 'd never lift a finger t' do a bit t' help! You don't run from me!"

He'd loosened and removed the belt from his waist and now brought the strap down on Russ. The boy didn't wear pajamas—they couldn't afford unnecessary things—and was dressed in underpants and undershirt. The strap caught the flesh at the back of his thighs, and he screamed, more in fear than in pain for Toby was too drunk to do a proper job of it. The strap went up and came down again as Russ shrieked bloody murder, and then Eileen was behind them yelling \emph{"Expelliarmus!"} and the belt flew out of Toby's hand. Toby's grip relaxed in the surprise of finding himself beltless, and Russ was out of the room and down the stairs as fast as he could run.

The wand disappeared, and Eileen was soothing the astounded Toby. "Nothing happened, Toby. Y're dreamin' or something. It's all right, come to bed. Y're tired." Russ didn't hear any more. He ran through the kitchen and crouched in the area yard, shivering in the cold.

Twenty minutes later, Eileen came looking for Russ. "It's okay, dear, you can come inside. He's asleep." Russ padded into the kitchen, his face wet with tears. "Let's look at you," his mum said, examining the still red skin on the backs of his legs. "Does it sting?"

Russ shook his head, but she put cold compresses on the marks anyway and then held him until he stopped whimpering and relaxed in her arms. "It's a cold world for a working man, Russ," Eileen tried to explain. "You work hard for every little thing and then the world takes it away. Sometimes a man just explodes from all the pressure."

Beginning to remember, Russ asked, "Did you use magic on him?"

Eileen stiffened. "No. I did not use magic on him. I used magic on the belt and made it go away. I did not use magic on your father." She sat him up in her lap to lock eyes. "Russ, a witch must never, never use magic on a muggle. It isn't fair. It isn't right. We have all the weapons muggles have. We have words, and fists, and everything else. There's no reason why we can't fight them fairly. Magic in the nonmagical world isn't fair."

"But you used magic upstairs."

"On the belt, not on the person. And only because I didn't want to punch him in the nose."

Russ giggled. Eileen put him to bed then, down in the sitting room on the sofa. She lay down in the boy's bed upstairs while Toby sprawled in their bed in the large bedroom. By the time Toby woke up the next morning, Russ was already in school.

School was no better. They had drawing just before lunch, and Russ took the last full packet of crayons. "Hey," said Neil Philips behind him. "I wanted those."

"Well, I{\el} got them{\el} first." said Russ, and took the crayons to his desk.

At lunch time, Russ found a bench off to one side where he sat to eat the sandwich his mum made for him. Three older boys, about nine years old, came up to him, Neil right behind.

"Hey, funny-looking," the first boy said, "I want to talk to you."

Russ got up and tried to leave, but they blocked his path.

"I said I wanted to talk to you. Is it true you didn't know your own name 'til you were six?" The boys all laughed. "I'm Brian. Neil's my brother. I want you to stay out of his way."

Russ didn't answer. He tried to move sideways, but there wasn't enough space to get away.

"Do you understand me, funny-looking?" Brian looked around at the others. "Not too bright, is he? What's your name?"

"R{\el} Richard," Russ said quietly.

"Well, Ra-Ra-Richard, people like you are supposed to wait and let people like us go first. That's why we live in nice houses and you live in pig sties. Got it?"

"Y{\el} es," Russ answered. There was nothing else he could do. They were bigger, and there were more of them. Probably no one even saw he was in the middle of them since he was so much shorter.

"Good," said Brian. He reached out and fingered Russ's blazer lapel. "And tell your parents to get you some clothes that fit. You're an eyesore." The boys left, laughing.

Russ sat back down to finish his sandwich. He was seething. \emph{Wait,} he thought, \emph{just wait 'til I'm old enough to do magic. I'll show you. I'm better than you are, and you'll have to wait for me.} A new thought came. Muggles. \emph{That's all you are—muggles. Just muggles. I'm a wizard! And I don't have to care what you think because when I'm eleven, I'm going to a different school that wouldn't even look at people like you. And I'll learn to do great magic, and I'll be just like everybody else, and we'll all laugh at you. Who needs muggle friends? When I'm eleven, I'll have wizard friends.}

The thought carried Russ through the rest of the day and gave him an inner dignity that he could see, even if no one else could. He paid no attention to Neil making faces at him behind the teacher's back. He did his exercises and turned in his papers, and when school was over he walked calmly out of the building. He was once again allowed to go home by himself, and now he noticed even more how the houses changed, got older, more uncared for as he crossed the stone bridge over the river. And when he saw his mother wave from the front area yard, she really was dressed more shabbily than the other mothers. It didn't matter. She was a witch, and that made her better.

His dad was shamed-faced and apologetic. Toby didn't remember what he'd done the night before, but he'd been sick enough to know that he was likely out of control. Eileen told him he'd tried to beat his son, and got a couple of good licks in before she could stop him, so Toby was all over trying to make it up to Russ in any way he could.

After dinner, Toby said, "They teach you figuring in that school, right?"

"Yeah," Russ answered.

"What's eight and seven?"

"Fifteen."

"What's six and nine?"

"Fifteen."

"You want me to show you that game your great-granddad and I play all the time?"

"Isn't he a bit young?" Eileen asked.

"That'll just give him more time to get good at it. How about it, son?"

"Okay," Russ said, and watched carefully as his father showed him how to deal the cards, count the hands, and peg. It was a complicated game, and Russ couldn't learn it all in one night. Toby didn't go to the local all weekend, but stayed sober and taught his boy cribbage. By Bonfire Night, Russ was good enough that from time to time he could even skunk his father. They didn't go to the bonfire that night because there was nothing in the way of junk to put in the yard for the lads to scavenge. Even junk was worth too much. Without that, there was no point. Toby stayed home with his family and played cribbage with his son.

Wenny died on Christmas Eve. He was in a shop buying a gift for Gra when he had a massive stroke and was dead in minutes. The funeral was well-attended by older men with a nautical air who'd shipped out on one of Wensley Snape's voyages when they were young, for he really had been a sea captain. Two men even came from as far away as Liverpool.

It turned out, though, that Wenny had lived somewhat above his pension, and that when all was settled, there wasn't much to leave to his daughter-in-law, grandson and great-grandson. Toby got a bit of money, much of which he spent celebrating the fact that he'd gotten it. Russ got a few boxes.

When they opened the boxes, they found the voodoo dolls and the poison dart blowers, and all the other wonderful, dark things that Russ loved so much. Another box had the books with the fascinating pictures, books that Russ still couldn't read because they had words in a language he wasn't learning in school. There were some books in English, though, and Russ was particularly charmed by one on different things witches could do to curse someone. He sat up several nights going over the hexes and the jinxes, wishing he was at Nana's with his grandfather's wand so he could try them.

It wasn't until after Russ's seventh birthday that the full impact of Wenny's death hit. Russ's effective routine human contact had narrowed by a third. He saw Nana and Gra only rarely, but Wenny had lived in the same town, and while Russ hadn't visited every week, he'd generally seen Wenny two or three times a month at least. Now, that was impossible, and the only people Russ talked to outside of school were his parents in the evenings, if his father was sober.

Wensley's death affected Toby, too, in subtle ways. Of the three, Toby had the largest circle of acquaintance, for he had his mates at the mine, and the lads at the local pub. Some of the men were in both circles, for several of them made the same trip each day from home to mine and back, stopping to unwind over a couple of pints at the pub on the way home. But Wensley had been something more. He was Toby's port in a storm. Toby had always known that if worst came to worst and he had to strike out into parts unknown looking for work, Eileen and Russ could stay the while with old Wensley. Now that security was gone, and Toby had no one to look to for help but himself.

Even more subtle, and something Russ was far too young to understand, was that with Wensley gone Toby and Eileen had far less chance to be alone together. Toby's temper became shorter and shorter, and he tended more and more to take it out on Russ, who was now a serious, if subliminal, rival for Eileen's attention. Toby was far more apt now to lash out with palm or fist or belt when drunk, and Russ was far more apt to be the target. Russ was beginning to sport bruises on his wrists, arms, back, and legs, and it was a good thing his sleeves were long.

Then there was the whole battle over baths.

The house was an old one, and while a toilet had been put in, a bathroom had not. Toby did most of his ablutions at the mine, where there was a big washroom for the men to clean off the dust and grime at the end of the day after they came up out of the `hole.'

Eileen washed at the sink, and had the daily privacy of an empty house when she got home to take care of her needs. Russ washed his face, hands, and neck daily, but a couple of times a week he stood naked in a washtub on the kitchen floor, water halfway to his knees, while his mum soaped him down and poured water over his head to rinse him. At the age of seven, this ritual became deeply humiliating to him, repugnant, shameful.

"Mum, please," he begged as she unbuttoned his shirt and began to undo his pants. "Let me do it myself. I'm not a baby!" He didn't know how to tell her that he couldn't bear the thought of undressing in front of her, even though he'd been doing it all his life.

"You're not old enough yet to do it proper. You'd give a lick and a promise and be off."

"Please, Mum, don't. I can do it," She whisked off the clothes despite his efforts to impede her, and he tried to cover himself with his hands. Eileen shook her head and tsked with her tongue, but she did the washing as quickly as possible so he could wrap the towel around him and be decent again. This sudden newfound modesty puzzled her; she could understand neither its origins nor the suddenness of her son's intense embarrassment.

\subsection{Tuesday, April 4, 1967 (three days after the last quarter)}

On the first Tuesday in April, when Russ arrived home from school, his dressing gowned mum met him at the door with a bag that had his play clothes and a change of underwear. "Go over to Mrs.~Hanson's, Russ. You're going to spend the night there. She's expecting you. She'll explain." Russ could see past her that his dad was already home, something very unusual as it was at least four hours earlier than normal. Russ didn't argue. He took the bag and left.

Mrs.~Hanson was, indeed, expecting him. "We're going t' have so much fun, Russy," she gushed. "It's been so long since I could take care of you. My, you're quite the little man now. Come on in and have some milk and a scone."

"What's wrong with mum and dad?" Russ demanded as soon as he was inside. "She said you knew."

"La, child, there's nothing wrong and everything right. Did you know that on this day eight years ago your mum and dad got married? So it's like a birthday for them. Except on your birthday you want your friends over for a party, but on the anniversary of your wedding, you want t' be alone."

Russ let pass the whole reference to birthday parties. "You mean they're not angry with me? Or with each other?"

"Heaven's no, boy. They're happy as clams, and the more we leave them alone, the happier they'll be. Your dad even went t' the foreman this morning and begged the afternoon off without pay so 's he could be alone with his missus, and Eileen thought t' make it the whole night. She come running over here so excited, and I'll guess Toby's pleased as punch. He was hoping for just the afternoon."

Russ still didn't understand, except that he already knew being alone was good. Mrs.~Hanson at least was a vaguely remembered familiarity from his babyhood, so he decided to make the best of it. "Where can I put these?" he asked, showing her the bag.

Mrs.~Hanson was, as she described it, a double-pensioned widow. Her first husband was a sergeant who married just before D-Day and died fighting in France. Her second husband was killed in a mill fire less than a year after they were married. She lived in her late parents' house and got both pensions. The house was quite nice.

To begin with, it had more rooms than Russ's house had—though two of them had paying boarders living in them—and Russ found himself in an honest-to-goodness spare bedroom on the ground floor. Second, it had a real bathroom with a claw-footed tub. Third, it had a television. Supper was early, and Russ kept very quiet because the two boarders were eating at the same table. He kept his eyes down on his plate because he was afraid they would want to look inside him. The door behind his eyes was latched. After supper, thankfully, the two men went to their rooms.

That evening, Russ did two new things. First, he took a bath in a real bathtub in a real bathroom where he shut the door and was all by himself behind a curtain. Second, he watched the first two television shows of his life, both from America. Mrs.~Hanson turned on the television five minutes ahead of time because it had to warm up first, and then the shows came on. The first show was about a married couple. The wife was a little crazy, and she had a job in a factory where she was supposed to put chocolates into boxes, except the belt kept going faster and faster, so she was eating the chocolates to keep up. Russ laughed at that one.

The second show was harder to understand because of the strange accents, but in a way it was better because it was about poor people who suddenly got rich. They left their poor house and moved into a big mansion, but they still acted the same, not stuck up. They were smarter than the rich people, too. It was something nice to dream about—getting rich suddenly.

The next morning Russ left for school from Mrs.~Hanson's house, and when he got home, his mum had made a cake. Dad got home at a good time after only two pints at the local, and he and Russ played cribbage. Everybody was happy. Russ thought about the television a lot, and went around the house occasionally singing, "Oil, that is. Black gold. Texas tea." He thought that was funny, too.

After that Russ started spending occasional Saturday nights at Mrs.~Hanson's whenever his parents got a bit of money saved up. Saturday night was a good night because after his bath he could watch television shows like `Dixon of Dock Green,' and Russ finally saw what London looked like. He also watched the Avengers, and thought Mrs.~Peel was a great fighter. His favorite, though, was Doctor Who and the TARDIS time travel ship. He wanted to travel in time, too, but Mrs.~Hanson explained that it wasn't possible.

The last weekend in June, when summer break was starting, Russ went to spend two nights with Mrs.~Hanson. The first was the normal Saturday night, but Sunday night was really special. On Sunday night, Russ was going to watch `history in the making,' for it was the first time in the whole world when there would be a live television broadcast from every country to every country at the same time, and Russ was going to watch it. So were the two boarders.

"It's because of those satellites they have up in space," Mrs.~Hanson explained. She didn't understand them completely, but one of the boarders knew that they floated up there in space so far away that you couldn't see them, and they could now beam radio and television shows to each other and then back down to earth so that everybody could watch the same show at once, and Russ got to watch the very first one.

All Sunday afternoon, while he played in Mrs.~Hanson's little area yard, Russ kept looking up at the sky, hoping he could see the satellites, and wondering how they could stay up there and not fall down.

That night, with everybody helping him understand what he was looking at, Russ watched the rest of the world. There was a shopping district in north Africa and the traffic speeding by in Paris. They got to see the house where the Presidents of the United States and Russia were meeting, and there were a bunch of people talking that Russ didn't understand. Then he watched a real cowboy in Canada.

It was tomorrow in Japan and Australia, nearly five o'clock in the morning, and men were working on the Tokyo subway, and the trams were taking people to work in Melbourne.

Then they showed the great outdoor disc of the telescope, and Russ was entranced. He couldn't believe that he was not only watching tomorrow morning, but that he was also looking at something that could see millions of miles away. Satellites and telescopes, and he was in love.

The last thing on the broadcast was the Beatles. Russ had heard their songs, but not seen them. Their names appeared on the screen, so Russ knew which was Paul, and which was Ringo. They were recording a new song with a lot of people and an orchestra. Russ especially noticed that John's nose looked just like his dad's nose, and when it was all over went to bed humming "All you need is love{\el}"

The next day, Russ didn't describe the whole show to his parents because there was a lot he didn't understand, especially when the serious men were talking about serious things—that had been boring—but he managed to give them an idea of Paris streets, and the beach in Canada, and he even tried to draw what the great Australian telescope looked like. Toby wasn't sure all this exposure to the outside world was good for a working class boy; it made you discontented with life. Eileen was better pleased.

That summer, while Toby was at work and Russ wasn't in school, Eileen started to talk to Russ about Hogwarts.

"It's a great castle on a hill, with a big lake—there's a squid in the lake, so nobody swims there—and the students can fly on broomsticks and play a game called Quidditch. Everybody lives in four different parts of the castle called houses. You'll probably be in the same one I was in—Hufflepuff. You'll like it there. There's all different kinds of people there, and they all work together."

"Did you play Quidditch, Mum?"

"La, no, child. I was never good on a broomstick. I was captain of the gobstones team, though."

Then, in August, Eileen and Russ went to visit Nana for three days, and everything changed.

Eileen and Russ worked together in Nana's garden, weeding, picking off bugs, and pinching back some of the flowers to prevent plants from going to seed. Eileen was talking about Hogwarts.

"The head of Hufflepuff house is Professor Mullein. He's the Herbology teacher, too, so you should get on well with him. He even knows about Nana by reputation, though she never went to Hogwarts. Hufflepuff house is in the lower levels, and you enter through a wall near the kitchens. That's where the house elves work. I never saw or heard of house elves before I went to Hogwarts{\el}"

"'Leen," Nana called from the edge of the garden. "Can I talk to you for a moment." The two women stood near the kitchen door, but Russ could still hear them in the quiet summer air.

"You'd best not get the boy too excited about Hufflepuff," Nana said in a warning tone of voice. "He might not be sorted there."

"Of course he will," Eileen said. "Children 're always sorted into the same house as their parents. Since I was in Hufflepuff{\el}"

"I went to see Tabitha Pollard yesterday. She did a chart for 1971."

"I checked that already, Mum," Eileen said. "Mercury's in his fourth house and it's in Virgo from July twenty-seventh to October first. He'll be fine."

"No, 'Leen. Mercury is retrograde beginning August thirteenth. It goes back into Leo on August thirtieth, and doesn't reenter Virgo until September eleventh. It will be in Leo on the day he's sorted."

There was silence, and Russ could tell from the quality of the silence that his mum was trying to cope with sudden, bitter disappointment. "They can't put him into Slytherin," she said. "A little half-blood boy like him{\el} They'd eat him alive."

"Don't tell him that. There must be other half-bloods in Slytherin. I don't think we have that many pureblood families left. If he's at least prepared to accept Slytherin, he may be fine there. Just don't get his heart too set on Hufflepuff, and make sure he's ready for Slytherin."

From that day, things changed. Eileen began talking to Russ about the other three houses at least as much as about Hufflepuff, and about Slytherin most of all. She said that in Slytherin everybody was ambitious and eager to get ahead in life. Slytherin students stuck together more than the other houses, and if Slytherins were your friends, they'd watch your back and stick up for you. The Head of Slytherin house was Professor Slughorn, who taught potions, so Russ should do well there since he would go to Hogwarts already knowing so much about potions.

It was also from that day that Eileen began to teach Russ how to defend himself. They brought grandfather Prince's wand back home with them, and she and Russ would go out onto the moors to practice, out where the magic they did wouldn't register with the place Russ's mum called `The Ministry.' She began to show him how to read other people.

"Look in their eyes," she told him. "You'll see the attack in their eyes before they move, before they say anything." She also told him he had to close his mind to the person he was fighting. "Don't let them read you," she said, "or they'll know what you're going to do."

That part turned out to be easy. Russ 'd always known how to close his mind. It was what he did to everyone but his mother, and sometimes he even hid things from her. Nobody knew it, of course, because they didn't know how to read him. Now, on her orders, he closed his mother out completely when they practiced dueling.

"Great!" she told him. "You catch on fast. Let's work on reflexes."

That fall, Russ started getting into fights at school. The first time, the school couldn't call his mum because she didn't have a telephone. Instead, they gave him a note to take home to her.

"What happened to you?" Eileen exclaimed when Russ walked in with a bruised jaw and a cut on the side of his mouth.

"Neil Philips wanted to fight with me," Russ replied, and handed her the note.

The note said that Russ had started the fight, that he'd attacked the Philips boy without provocation, and that naturally the Philips boy had been forced to defend himself. The note asked Eileen to come to school the next day with Russ to discuss the matter.

The meeting was highly unsatisfactory. Mrs.~Philips was there and not only said that boys like Russ shouldn't be allowed in school, she insinuated that Russ's mum was a slovenly housewife, which as she used to come over when Russ and Neil were babies, she knew to be untrue, so Eileen called her a liar and the mother of a bully. Russ insisted that Neil had been about to attack him, Neil denied it, and several of Neil's friends came forward to testify that Neil was the victim. Since Russ had no friends to testify for him, the case was decided then and there. Russ was to stay home for three days. Neil was triumphant.

When Toby learned what had happened, he took off his belt and gave Russ six sharp licks with it for starting a fight and giving the family a bad name. Then he began teaching Russ how to box.

"Nella Tarleton," Toby told Russ gravely. "Y' got to think like him. Featherweight champion of Britain, and th' whole world. I saw him once, his last fight as it turned out, in Manchester when I was fourteen. M' dad borrowed and begged for th' tickets so I could see a master just once. Nel was thirty-nine, and Al Philips was twenty-five. Nel, he'd lead with his left, get several jabs in, and be back out of range 'fore Philips knew what hit him. So quick he was, no one ever could lay a glove on him. Didn't have a mark on him from a gross of fights except from th' ropes. He could sure use th' ropes! A right scientist he was. Always thinking, always planning. After he retired, come t' find he only had one good lung! Imagine going th' distance in all them fights with only one lung. M' own dad used t' tell me the only way they could get Al Foreman t' fight him was t' limit th' fight t' twelve rounds, 'cause if Nella beat Foreman in fifteen, he'd be lightweight and featherweight champion at th' same time!"

Russ was learning how to punch—jabs and hooks—but he wasn't very good at it. What he was good at was dodging and feinting. If you could dodge and feint enough, then land one or two good jabs, you had a chance of winning. It was fun sparring with his dad, and it gave him more confidence about facing larger boys in the play yard.

"Now you remember," his dad cautioned, "don't never start a fight, but if they mess with you, you give 'em what for!"

It was to his mum that Russ expressed most of his animosity. "They're always pushing me around and calling me names. I could see in Neil's eyes that Geoff was behind me and he was going t' push me so Geoff could grab me. But I got him first! And I'm ready. He comes for me again and I'm going t' tie his legs together. I'm going t' glue his tongue t' the roof of his mouth."

Eileen looked at her son with concern. "How are you going t' do that?" she asked.

\emph{"Locomotor Mortis!"} Russ told her. "And there's a tongue-tying curse{\el}"

Eileen seized his wrist in a painful grip. "Where did you learn those curses?" she demanded. "Where! I told you, you don't ever use magic against a muggle! I catch you using magic against a muggle, and I'll skin you alive! Where did you learn those? You tell me now!"

"There's a book that Wenny had. It's in the book." Russ was scared now. His mother seldom got this angry.

"I'm going to lock those up. Imagine you learning things like that! You promise me you'll never use magic on a muggle."

"Promise," Russ said, but his heart was with the curses that could prove to people like Neil and Geoff that he was stronger than they were. He concentrated on the boxing.

Russ was now exploring farther and farther afield. Instead of sticking to the mill side of the river where everything was familiar, he started going out onto the moors. He circumnavigated the town and found there were other ways to get to different areas besides walking openly down the street. He discovered where Neil Philips lived, but he didn't do anything about it because Neil was just a dirty old muggle and not worth the effort.

Sometimes, when his mum was busy putting his dad to bed after a bad night at the local, Russ would go out onto the moors for the half hour she needed instead of staying in the kitchen or area yard. If the night was clear, you could see for a billion miles. He started to read books about stars in the little school library and was captivated by the pictures taken by the Russians the year before he was born of the `other' side of the moon. Both the Americans and the Russians were trying to get there, and Russ tried to find out as much as he could about their space programs. Given who he was and where he was living, that wasn't much. Russ longed for a telescope, and treasured with something close to hunger the thought that at Hogwarts he'd be able to study astronomy.

Over the next couple of years things continued on a downward spiral. Britain devalued the pound, which helped coal, but also made things harder in other ways. Petrol became more expensive, and Mum could no longer afford anything that wasn't made or grown in England, and not much of that. Toby worked just as hard for less reward, and drank even more. His belt became more active because Russ was getting into more fights, though now he made sure he never threw the first punch. Mrs.~Hanson's house was a refuge, and Russ looked forward to those rare Saturdays he was able to spend with her, but it didn't quite make up for everything else.

To make matters worse, Russ was finally growing. At nearly nine, he finally looked like he was six or seven. If he tried hard to act grown up, people believed he was eight. That was the good part. The bad part was that his clothes went overnight from being too big to being too small. It wasn't that he was fatter, it was that he was taller. He was now more than three and a half feet tall, still the smallest in his class by far, and nearly the smallest in the whole school, but bigger than his worn old uniform. His knees stuck out, his ankles stuck out, his wrists and neck stuck out. His play clothes were better, but even his jeans and jacket were rapidly getting too small.

By this time, Russ actively hated the other children at school and had long realized that he was strange because nobody would be his friend. The others were always talking about things he couldn't share, like `best friends,' and birthday parties, and even little things like passing notes when the teacher was writing on the blackboard. Things Russ wasn't permitted to join in on. He resented their friendships and hated the air of superiority they used whenever they couldn't avoid contact with him, and he told himself again and again that it didn't matter. He wasn't like them. He was a wizard, and they were just muggles. And one day he'd show them all.

On his ninth birthday, Russ's mum gave him a special present. She let him have the books she'd used at Hogwarts. There were books on the History of Magic, and on Charms, books on Transfiguration and Arithmancy and Muggle Studies. But the books that Russ loved most were the books on Potions, Defense Against the Dark Arts, and most especially, Astronomy. Russ remembered that although half of him was a Snape, the other half of him was a Prince—a half-blood wizard. On the back of each volume he carefully inscribed in his tight, cramped handwriting: \emph{This Book is the Property of the Half-Blood Prince.}

Russ also had a passion now, the first true passion in his life, and the passion was named Apollo. The satellites and telescopes of his seventh year of life, the stars of his eighth, were making way for a love of Saturn rockets and the capsules they carried into space. One good thing about school was that he now understood why the satellites didn't fall down, and what an orbit was, and he knew not only that the Americans were going to the moon, but when. Russ scavenged used newspapers out of rubbish bins looking for articles on the American space program. He treasured in his heart the possibility that he himself might go to the moon one day.

And then, in May of Russ's fourth year in school, a miracle happened. It was such a miraculous miracle that at first Russ didn't believe it, even though he'd seen it happen. The miracle happened at school.

Russ was doing his maths work when a small click caught his attention, and he looked up to see that three rows ahead of him a girl had dropped her pen on the floor. She bent down to retrieve the pen, and the miracle occurred. Her hand still inches away from it, the pen suddenly leapt upwards and into her grasp. Russ couldn't believe his eyes. A muggle had just performed a wandless summoning spell. A red-haired muggle girl of no account whose name, as Russ knew, was Lily Evans.
