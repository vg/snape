% A Difference in the Family: The Snape Chronicles (Rannaro)
% The Middle Years

\chapter{Snares}

The ending of the play came so quickly that it took Snape by surprise. \emph{All dead? They're all dead?} Snape closed the book and stared at it, forgetting his lunch completely. \emph{How many people died in this play? We're told about old king Hamlet, Ophelia, Rosencrantz, and Guildenstern. We watch Polonius, Queen Gertrude, Laertes, King Claudius, and Hamlet die. And in the end, we never really know if Hamlet was crazy or not. But the whole play is about death, about murder, and maggots, and skulls dug out of graves, and ghosts, and committing suicide...}

Snape flipped back through the pages looking for one speech. He found it in scene two of the first act. \emph{King Claudius tells Hamlet that his father's death is part of the natural order of things, that his father's father also died, and his father before that. It's just natural, so why grieve beyond the normal amount of time? And Hamlet says death is like sleeping. And the gravedigger jokes and sings while he's digging a grave because this is so normal to him. The soldiers go off to die for a worthless piece of land, and they consider it their duty. The only one who is really troubled by it is Claudius because he killed his brother for personal gain, and he still has the power, wealth, and woman that he killed for. Professor Dumbledore is fascinated by muggle attitudes toward death, and this play is full of it.}

His afternoon classes were quiet, maybe because the students had heard about the morning session. It gave Snape time to think, and what he was thinking most about was Lily.

There were, Snape reasoned, two basic categories of death. Timely death of natural causes after a full life, and untimely death where life was cut short either by early natural causes or by artificial causes. Snape knew three people who'd died the first kind of death - both his grandfathers, and his muggle grandmother, Gra. Grieving for them had been gentle, with a sense of the circle of life turning as it should.

All the others, and there were so many others, had gone by violence. \emph{How do you reconcile yourself to loss through malice?} Hamlet and Laertes had sought revenge, and Snape could see how vengeance might close the wound and allow grief to heal. But what if all chance of vengeance were snatched away?

No vengeance was needed for his parents. His mother's death was tragic misjudgment, and his father had atoned by suicide. Nor did he any longer need vengeance for Nana - he already had it. It was his information that had sent Rosier and Wilkes to their deaths, a particularly satisfying piece of work.

What about Lily? The Dark Lord had killed her, Sirius Black had betrayed her, and James Potter had bartered her safety for sentiment. All had been punished, two with death and one with Azkaban, but the punishment had nothing to do with Snape and so the wound remained open. At no point had Snape been allowed vengeance, and so there was no justice.

Dumbledore, of course, was absolved of blame. He could not have forced James to make him secret-keeper. That fault rested with James, and James alone. Snape knew, too, that he himself was not at fault because he'd realized that the prophecy he'd heard referred to the pureblood auror's child, not to Lily's part-blood baby, and only the arrogant stupidity of the Dark Lord had led to her tragedy.

What then of Lily, the pain of whose death could never fully heal if vengeance was denied? \emph{It's a pity after all that Black's in Azkaban. I might otherwise have gone after him.} It was a canker, an ulcer, a wound that could never heal.

What of Lily? She was no ghost, of that Snape was sure. Hamlet's father was a ghost because he was doomed to daily punishment and allowed to wander at night. Lily, naturally, would never be punished because she was a saint, an angel...

The second afternoon lesson ended, and Snape locked up quickly. He needed to find a ghost. The Baron spoke in monosyllables, and most of the others were unknowns to him. The only one he could recall being at all talkative was the Gryffindor ghost, Nearly Headless Nick. The question was, where to find him.

Students were heading in different directions, to the common room, to the entrance hall... Snape let them pass and then turned into the more labyrinthine passages of the dungeons, places where classes were no longer held. Out of sight, and hopefully out of hearing, of the stream of Slytherin students, Snape quietly called, "Baron? Baron, I was wondering if I might talk with you."

It was several minutes before the Baron appeared. He hovered before Snape, grim and gaunt, his clothing stained with silver gore. The air grew colder; Snape's breath hung in a misty cloud before him. "Baron, I don't mean to disturb you, but I have some questions about the spectral existence." The Baron nodded, so Snape continued. "May I ask you, or if that isn't convenient, would you know who I could ask?"

The Baron held up a hand to stop Snape from speaking or moving, and dissipated into the cold dungeon gloom. A moment later the Gryffindor ghost appeared. "The Baron said you needed me, what? I don't believe I've ever been summoned by the head of Slytherin house before."

"Summoned?" Snape wasn't sure what to say, but he certainly didn't want to offend a ghost. "I would never summon you - it's not my place. I asked the Baron for advice, and he thought you might be the best..."

"Well, that all right then, isn't it? You're the new one, aren't you? I'll say you're different from old Slughorn. What did you need? "

"I... well, sir, I... Excuse me, but I've only ever heard a nickname for you, and I don't know if it's respectful to use it."

The ghost beamed. "You always did have that aura of being partly in the other world. I am Sir Nicholas de Mimsy-Porpington. You may call me Sir Nicholas."

"Thank you. Sir Nicholas, why do some people become ghosts when they die?"

Sir Nicholas froze, as if he'd been ambushed. "That is a rather personal question," he said. "I'd hardly call it asking for advice."

"I'm sorry. I didn't mean to pry."

The ghost relented. "Most witches and wizards are willing to accept death when it happens. They go on. Others choose not to go, usually because... well, because they're afraid. They cling to a shadow existence that stops short of death."

Snape's voice dropped to a nervous whisper. "Does that mean the Baron was afraid of dying, too?"

Sir Nicholas stared at him in surprise, then slowly began to smile. "I guess it does," he said.

That resolved the question of Lily's ghost. Lily would never be afraid to die. She was gone, and would never come back except in dreams. \emph{She never knew I was thinking of her, never knew I was helping. For all I know, she died thinking I was her enemy. And now I'll never be able to explain it to her.} It was another bitter thought.

\emph{What happens when people die? Is it different for wizards and muggles? Are we all the same when we die, or do we go each somewhere different? Is my muggle father with my witch mother? And what of Lily, who was both muggle and witch together? Or me, who am both and neither at the same time?}

Dumbledore studied death, so perhaps, probably, he would know. Snape made his way to the Great Hall for supper, and headed for Scrimgeour's vacant seat rather than his own. "May I?" he asked, and Dumbledore responded, "Of course."

The food was served, supper started, and Snape plunged in. "I've been reading Hamlet."

"A most excellent play. Not my favorite to watch, nor my favorite to read, but unequaled for its depth and philosophy."

"It's about death. You said you were interested in it because you were interested in muggle ideas about death."

"I did, because I am."

"What happens to us when we die?"

"I do not know. I have never been there myself to check. It is an undiscovered country..."

"From whose bourn no traveler returns. But after all these centuries, surely we know something."

"You did not ask me this question when your parents died."

"I was too young to realize its importance."

"Well, I fear I have no answer for you. There are many who believe this world is all there is, and that when we die, we are simply gone. Others believe that we remain as a spirit world, guiding and protecting those who are left. Still others believe that there is an all-encompassing cosmic force from which we have been separated, and that it is our goal to re-submerge ourselves into that force, our success and failure depending on how well we have lived our lives, so that if we are not successful, we may return in a new body to try again. Did your parents give you no religious upbringing?"

"Dad didn't believe in it. Neither did Mum."

"It is a pity. Many muggles go to great lengths to inculcate these ideas into their young children. In any case, another group believes that we have only one chance to do well in this world, after which we are sent on to the next stage. If we have performed correctly, the next stage is happiness. If we have performed ill, the next stage is punishment."

"Heaven and hell."

"Those names have been used."

"Which of all these views is right?"

"I do not know. I have never made the journey, and none who has, has ever returned to advise me. It is, empirically, an entirely open question. I did once speak to a woman whom the muggle healers believed to have died, but who was then revived."

"What did she say?"

"She described a contraption called, if I remember correctly, a roller-coaster. She said it was exactly the same feeling you have just as the car reaches the highest point and you know you are about to plunge down. She said to herself, `Here we go...' and then there was nothing. She woke up to find the healers reviving her."

"That doesn't tell us anything."

"I believe that it tells us a great deal, but that it does not answer all of the question."

"You mean it doesn't address the question of what happens after death, only the moment of dying."

"Precisely. I find it comforting to think that the moment of death is devoid of fear. It is like riding a children's amusement ride - Here we go!"

"What leads up to it can be pretty terrible, though."

"Very true. Often when we think we are afraid of death, we are really afraid of its preamble, of all that comes before dying. The shorter the preamble is, the easier death is."

"You still haven't answered my question."

"It is because I do not know the answer."

And there it was. No answer to Snape's most pressing question because the answer didn't exist. \emph{Where is Lily? Has she been reabsorbed into some cosmic force, or is she a distinct entity that can still communicate with me, or something else entirely?}

The Library turned out to be no help at all, since it had no works on religious philosophy. It was maddening. \emph{No Shakespeare and no religion. What kind of a library is this?}

The next day was Thursday, and now Snape had a problem. Books. He desperately wanted books. But unlike the Shakespeare, these were books no one else could pick out for him because he himself didn't know what they were. He had to go to that great bookstore on Charing Cross road, climb to the proper floor, and browse.

"Would it be all right," Snape asked Dumbledore at lunch. "If I went to town this evening?"

"You do not need to ask. It is your night off."

Snape hesitated. "I was rather hoping that Hagrid might come with me."

"If you are going where I think you are going, Hagrid may create something of a stir. Still, it is good for people to have to adjust their preconceptions from time to time. I shall speak with Hagrid."

"London? Monday he was shaking like a leaf at the thought of going to London. What'd y' do?"

"I? I did nothing, Hagrid. He is doing what he did the day he left Voldemort and came here to you for help. He is letting an all-consuming priority override everything else, including fear. That boy's ability to focus is stunning. Then, it was concern for a friend. Now it is concern for the same friend from a somewhat different angle. When his parents died, he never opened up enough to be able to resolve these questions. It is good that he can do so now. So you will go with him?"

"Course I will. I ain't never been in a muggle bookstore. Might be interesting."

"I may make a small trip this evening as well," said Dumbledore with a gentle smile. "I have a sudden overwhelming urge to visit Stratford on Avon and lay flowers on the grave of William Shakespeare."

Snape and Hagrid went first to the Leaky Cauldron. Or rather, they passed outside the Leaky Cauldron, since Snape wanted to show Hagrid the route between it and the bookstore, apparating not always being possible where there were crowds of muggles. The two then made their way up Charing Cross Road, across Shaftesbury Avenue and into the bookstore.

Hagrid had some trouble maneuvering between the tables of books on the ground floor, and Snape wouldn't let him get into the lift. "That's all we need, to have the lift get stuck with you in it," so they took the stairs. Snape wanted history as well as religion, and spent a minute or two studying the board that told what was on each floor.

Hagrid wandered through rooms filled with thousands upon thousands of books while Snape browsed, and by the time Snape was ready to go, Hagrid had found a book on animal anatomy - one intended for veterinarians - and held it out to Snape. "D' ya think ya could get this one, too?"

The price of Hagrid's book was more than Snape's put together. Snape checked his money and stifled a sigh. "Yes, of course. Just let me decide which of these I want to buy right now." He settled on a volume entitled \emph{Views of the Afterworld} and a history of medieval England, and reshelved the others. He could get them later, after he'd had a chance to go to the bank.

They queued up to pay. "What's that little card?" Hagrid asked in something more than a whisper.

"It's a credit card. You use that to pay so that you don't have to carry money with you. Then you reimburse the credit card company."

"Why ain't you got one o' those?"

"You have to qualify for them. Have a job and income."

"You got a job."

"A job that pays you in pounds. Banks don't understand our currency."

"Oh," said Hagrid, but Snape could see he still didn't understand.

"Next time I come, I'll have to visit a bank and get more money. You can come, too, if you like."

"That'd be right interesting," said Hagrid.

As they left the bookstore, Snape was considering a completely new problem. His muggle money wasn't going to last forever, not if he kept buying books. Where was he going to get more?

"They said you'd left Hogwarts,"

Snape and Hagrid had gone into the Leaky Cauldron so that Hagrid could visit with some of his friends. Snape reasoned that since Hagrid had waited for him, it was only fair now that he wait for Hagrid, and a little butterbeer while he read at a corner table would be nice. And it would have been nice had not Alastor Moody made his presence known.

"Is it a crime now to leave Hogwarts?" Snape said without turning around.

"Just hit the statute book. Decree four-seven-eight-seven-dash-three. Concerns known felons on parole attempting to defeat justice by trying to live a normal life. You might look at me while we're talking, you know."

Snape turned to Moody and instantly looked away. The auror's face was more ravaged by scars than Snape could have imagined possible, and it was now dominated by a false eye, a rotating blue orb that darted its gaze haphazardly around the pub, seemingly independent of both its normal mate and the will of its owner. It was hideous.

"It's even worse from my side," Moody laughed hoarsely. "Come sit with me, Death Eater."

"I would prefer not," Snape replied, but Moody sprang to his feet and took Snape's arm before the words were finished.

"Can't look at your own handiwork? I thought Death Eaters had stronger stomachs than that. Sit with me."

Steered to an out-of-the-way table, Snape sat while Moody ordered two firewhiskies, but he would not look at Moody's face. Even the glimpse he'd had was too much.

"They just let me out of the hospital this morning, and what do I get as a welcome back gift?" Moody continued as the firewhiskies were placed in front of them. "The chance to express my appreciation to the little dungball that gave me my new eye. I hurried right over, hoping you'd drop in before slithering back north to hide behind Dumbledore." He paused. "That whisky's from me to you. Are you going to insult me by not drinking it?"

Reluctantly, Snape took the glass and raised it to his lips. Just as he took the first sip, Moody raised his own and said, "Death to Death Eaters!" tipping the glass back and taking the firewhisky in two gulps. "Two more," he signaled to the barkeep.

"I don't want any more," said Snape.

"You drink my whisky, or I'm going to describe in loving detail what it feels like to have a blasting spell hit you in the face, and to hold your own eye in your palm..."

Snape choked the rest of the whisky down and allowed Moody to put the second glass in his hand.

"Good boy. Now we get to discuss the vital matter of your breaking parole and the direness of the consequences."

"I'm not breaking parole. I have to stay under Dumbledore's authority. I still am. I still work for him. It didn't say I had to spend every minute at Hogwarts."

"That's a matter of opinion. Normally when they weigh a rat's opinion against a human being's, the rat loses. My opinion is you owe me an eye, several chunks of flesh, and a load of skin and blood."

"I didn't do it."

"Accessory before the fact. That's a crime, and it occurred after your sentencing. Pure justice would allow me to feed you to the dementors personally." He tapped Snape's glass. You'd better drink this quick, you've got a lot more coming, and we wouldn't want this to take all night."

"Why are you trying to get me drunk?"

Moody laughed again. "I want to watch you try to apparate when you're snockered. I've got a bet I can push you to a three-way splinch."

Snape set his glass down at once, but Moody covered his hand with a great paw. "It's not considered polite to refuse to drink with someone you nearly got killed."

A great bulk obscured the lantern light as Hagrid's shadow darkened the table. "Hullo there, Alastor," Hagrid said jovially. "Sorry t' interrupt yer fun, but I promised t' get this youngster back in time t' do his rounds. Ya put that drink down now, lad, and come with me."

Snape obeyed quickly, not glancing at Moody to see his reaction. Together he and Hagrid hurried out of the pub and into a little side street where, unobserved, they apparated back to Hogwarts.

"Do me a favor," Snape said to Hagrid after they were inside the gates and about to separate. "Don't tell Professor Dumbledore about Mr. Moody. There's already enough trouble."

"Suit yourself," said Hagrid, and went to his hut.

Friday was normal. Snape closed off all memory of the evening before and didn't think about Moody, the Leaky Cauldron, books, or bookstores all day. As he cleared up his classroom just before supper, however, he wondered which of his two books he should take to the Great Hall to read. He went to his office, took the books from the bag that he'd left on the desk and stood, one in each hand, contemplating the titles on the spines.

"Neither," said Dumbledore from the doorway.

"Why not, sir?" Snape asked, looking up and across the room towards him.

"One of the great benefits of a community meal is the opportunity to treat each other as a community. The staff spends its days isolated in little cubicles, cut off from one another, deprived of contact with their peers - supper is time to reconnect and socialize. Not to seek more isolation."

"I don't feel comfortable with them all, and I don't think they feel comfortable with me."

"You are relatively new. You had a different relationship with them a few years ago and are finding it hard to adjust. There have been distractions and unpleasantnesses that have interfered with the progress of forming collegial bonds, but that is not sufficient reason to abandon the attempt. It has been a stressful week. Come, join us at supper and leave these for later."

Snape put the books back on the desk and followed Dumbledore to the Great Hall. "How was London?" Dumbledore asked as they walked.

"It was all right. I didn't see much more than a bookstore, a street, and a tavern."

"No problems?"

After weighing the pros and cons of his options, Snape asked, "Have you been talking to Hagrid?"

"Hagrid was singularly reluctant to talk. This is usually not a good sign. I thought I might hear about it from you. Good evening, Minerva."

They'd reached the high table, and Dumbledore took his place, motioning Snape to the chair on his right while he greeted McGonagall on the left. Snape murmured, "Good evening, Professor McGonagall," but didn't really want to talk about what had happened in London where she would overhear, or Flitwick on his other side either.

The food appeared, and they filled their plates, then Dumbledore asked, "Did you find any books you wanted? Hagrid showed me the one you bought him. I was quite impressed."

"Hagrid got a book?" chimed in McGonagall. "Whatever for?"

"It is a medical book, Minerva, meant for animal healers. The language is highly technical, but the anatomical drawings and discussion of ailments are of great value. I will not describe them in detail as we are eating, but perhaps later."

"I got a history book and a book on the philosophy of death," said Snape. "I thought the history might give me some background on more of Shakespeare's plays. He writes about things I'm not familiar with."

"An excellent idea. It is always valuable to prepare oneself to understand as much as possible."

"There were other books I was interested in, but I need to go to the bank first."

"I trust Hagrid was not too much trouble in the bookstore."

"He did attract a lot of attention, but since he was looking for things in other parts of the store, it didn't bother me much. Then we went back to the Cauldron so he could see some of his friends." Snape had by now resolved how to handle this part, so he continued. "Alastor Moody was there."

"He is out of the hospital! That is wonderful news. I trust he is well on the way to complete recovery?"

"He seemed fit. He said he'd hoped to meet me there, and bought me a couple of drinks. I only had one of them, though, because we had to come back to Hogwarts."

"It sounds like he was expecting you."

"Well, it was my night off, and it was the area of town where I normally go."

Dumbledore then turned his attention to McGonagall while Snape discussed a levitation charm with Flitwick, but the important information had been passed in the guise of ordinary conversation - the aurors were watching Snape and knew when he left Hogwarts and where he was going.

I don't have to attract attention by going up to Dumbledore's office to tell him things. We could talk right in front of everyone else, and if we do it right, no one will even suspect. It was a useful tidbit of information, and Severus stored it in the back of his mind for later retrieval.

That evening after supper, mindful of Dumbledore's order (for such he took it to be) to socialize, Snape went to the staff room to play cribbage with Flitwick. It was McGonagall who referred back to the supper conversation.

"Muggle banks? You have money in a muggle bank? Wouldn't Gringotts be safer? You can keep anything safe there."

"Thirty-one, that's two for me. Yes, Professor, but in a muggle bank, the amount of money I have keeps getting larger. Fifteen-two, fifteen-four, and a pair is six."

"Larger? You mean they pay you to keep your money there?"

Flitwick was counting quickly. "Fifteen-two and three is five. And a pair in the crib for another two. And... I'll take the two you didn't count because you were paying too much attention to Minerva."

"What? Blast! If I get skunked this game, Professor, it'll be your fault."

"It's only a game. Tell me about this `they pay you' business."

"It's called interest. They don't put your money in a vault. They loan it to other people or invest it, which is a kind of loan. They collect interest on the loans, and they pay part of it to you because the loan was made with your money. I hadn't been using the money for a while, so the total got bigger. Now I've been withdrawing some for books and the theater, so the total is going down. I either have to stop buying so many things, or find a way to make more muggle money."

"Where did you get that money?" asked Sprout.

"They sold my muggle grandmother's house when she died, and I was her only heir. Then I made some muggle money tutoring. I used to charge less if they paid me in pounds."

"Why would you want muggle money?"

"For things you can't buy with wizard money, like a subscription to the \emph{Guardian} or a ticket to the theater. Books."

"You can buy books at Flourish and Blotts."

"Not these books."

Sprout wasn't really interested in the kinds of books Severus bought, so she let the subject drop. Later, having lost the game but not having been skunked, Snape went to his rooms to decide which book to look at first. He chose the history.

It was a peculiar thing, but although Snape always started a nonfiction book at page one and tried dutifully to read to page two, page three, and so on, he always succumbed to the desire to skip around and jump back and forth. This evening was no exception. There seemed to be little profit in beaker people and Celtic migrations, so he started to browse. That was when he came across the Civil War.

Now Snape actually knew something about the Civil War, for there were some in the countryside around his home who still passed down the old divisions, remembering that mighty Liverpool had once been a Whig island in the Tory sea of Lancaster and lamenting the tragic death of King Charles, and who might have continued drinking to the king `over the water' if any such were left alive.

To his surprise, however, this was quite a different civil war. This one was fought in the twelfth century between a king's daughter, Matilda, who claimed to be queen, and her cousin Stephen, who had himself crowned king. It was a time of lawlessness and wildly shifting fortunes, when for a time both armies were generaled by women, Stephen having been captured, but his wife refusing to surrender and finally forcing a prisoner exchange.

\emph{I have those books!} Snape thought, and dug into his belongings for things purchased the previous August, including the four volumes of murder mysteries he'd bought but never had a chance to really read except for the first. Now he started the second.

That first volume had not mentioned the civil war, but this one revolved around the siege of Shrewsbury and a murderer who tried to hide his victim among the scores of executed traitors after the town and castle fell.

There was from the beginning a character with whom Snape identified. In his early twenties, short, slender, dark haired and dark eyed, quick with his tongue but slow to reveal his true self, this character quickly became one of the murder suspects, and throughout the rest of the book the question of his guilt or innocence hung in the balance.

It was past midnight and halfway to dawn when Snape finished the book, marveling first that he'd read the whole thing in a few hours, and second that the medieval mind saw no problem in judging guilt and innocence by combat, where the better fighter was also deemed to be right.

Now there were two questions to be resolved, that of death and that of justice. Snape dragged himself to bed, but dozed only fitfully, his head filled with too many ideas for sleep.

February eased its way toward March, though the cold remained bitter. In the middle of the month, Maggie Pulcifer left Hogwarts. She was the Hufflepuff student whose uncle, as it turned out, had been a Death Eater. Sprout tried everything she could to shield the girl, but it was the steady drip of small things, like drops of water on a stone, that wore her down and drove her away. The two Ravenclaw students were still holding out, though they were visibly weakening, growing more quiet and isolated as the days passed.

Only Slytherin remained strong, protecting its own with fierce jealousy. Slytherin students who'd never been Death Eaters, even those whose families had been harmed by Death Eaters, found themselves targeted for teasing by the other houses because they defended their housemates. The house as a whole became stronger with adversity, its unity now a matter of pride and honor, its opinion of the other houses reduced to scorn and expressed in defiance.

Snape visited the common room on an almost daily basis to be sure everything there was going smoothly and to ensure no retaliation was being planned. Behind the safety of the Wall and guarded by the lake above them, the Slytherin students were able to relax, devising ways to amuse themselves or help each other despite being confined more than usual by the weather and their own fortress mentality. Many were helping their housemates with their studies, and Slytherin's academic record for the year was high.

There were no more trips to London for the rest of February. Snape just couldn't force himself to brave the outside world again, so while his house strengthened, it seemed he weakened. There were more books he wanted to get, but that priority had not risen to the point where it could override his nervousness.

\emph{Views of the Afterworld} proved to be a valuable resource in some ways, but not in others. It gave very deep, detailed explanations of the different religions' and philosophies' beliefs and attitudes about dying and what happened thereafter, but no instruction on which belief was most likely to be correct. Snape was left with the impression that each person was allowed to imagine the afterworld that was most comforting or useful at the moment, and that there was a presumption that simple belief caused the image to become real.

\emph{That's silly. It's like belief in God. If God exists, will the fact that I don't believe in God change God at all? And if God doesn't exist, will the fact that I believe create God? Of course not. My personal belief or disbelief has no effect on God whatsoever, for the existence or nonexistence of God is outside of me.}

This wasn't an easy thought, and Snape wished his parents had raised him with some religious training so that he could have a little background on the matter, but there was no help for it now. He had to find out on his own. The only way to do that was to try practicing some of the things in the book and see if he got a reaction from the universe.

He decided to start at the beginning with animism and ancestor worship. Animism was, in fact, rather easy, since it was the belief that everything had a spirit and was capable of conscious thought, including trees and rocks, and that somehow these spirits formed a vast network so that disruption in one part could cause turbulence in another part. Most magic was based on this interconnectedness as was, apparently, the whole concept of prayer, and Snape had certainly had enough experience with the magical aspects.

Ancestor worship was based on the idea that the dead didn't leave you. Their animating spirits continued to watch over and protect the living, and they deserved to be honored. \emph{I'm not sure how much protection Dad could give, but Mum and Nana and Gra might be worth something.}

The next thing was what to do about it. The ancient Romans had a lararium in their home for simple daily observance, and it seemed easy enough to do, so Snape looked around his small domain for a place. \emph{Near the hearth. I don't cook here, though, so it isn't a proper hearth. On the other hand, I do have that cookbook, so if I started cooking it would be.} He set a small table next to the fireplace in his office and thought about what to put on it.

\emph{Spirits of the ancestors. I might have pictures at home, but that would mean going home and hunting for them. Leaving Hogwarts... Spirits of the place. That would mean the Founders, I suppose, though it might mean unnamed spirits of the cliff, the lake, and the forest.} The small dishes for burning incense and food offerings were easy - his potions stores were full of them - but he needed to set them aside for special use, dedicate them as it were, for the lararium.

The biggest obstacle was the picture of his parents, and as February waned, Snape tried to work up the courage to leave again, to go to the bank, to buy books, and to go to his childhood home in Lancashire for photographs.

With the beginning of March, however, a new priority took over, a priority that gripped the whole school and electrified Slytherin house with determination. The first weekend in March was the Quidditch game against Ravenclaw, and Slytherin's chance to show the rest of the school their true mettle.

The first signs of trouble appeared a week before the Slytherin-Ravenclaw match, though at first no one realized that there was anything ominous about it. Notices appeared on the school bulletin boards asking the school to join Gryffindor house in dedicating the first Saturday of every month to the celebration of a different world culture. The first one, coincidentally on the day of the game, was to be a tribute to Mexico.

"Where did they get that idea?" Snape asked McGonagall at Saturday supper.

"They came up with it themselves. I was quite thrilled. I didn't think they paid any attention to history, geography, or culture. To tell the truth, I'd have to think a bit to find Mexico on a map, but they're tremendously wrapped up in it. They've even checked with the kitchens to see if the house-elves can serve Mexican food that day. The whole thing is delightful."

For the next couple of days, Gryffindor students lobbied the other houses for their support in the event, calling it a chance for them all just to get together and have fun. It was soon apparent that Ravenclaw had entered wholeheartedly into the spirit of the occasion, for by midweek many students had already begun sporting sombreros and serapes over their birettas and robes. Hufflepuff was for some reason a bit more reluctant to go along.

Wednesday evening there was a major meeting in the Slytherin common room where the whole question of supporting the world culture event was debated. The diehard Gryffindor haters insisted that nothing Gryffindor did could ever do less than harm Slytherin, while the more conciliatory saw the occasion as an opportunity to build bridges and relieve some of the tension. It was finally decided that each student could do as he or she saw fit. Several of the Slytherin students sent for costumes.

\subsection{Saturday, March 6, 1982 (3 days before the full moon)}

On the day of the Slytherin-Ravenclaw game the school was a riot of red, white, and green. The school was planning to march en masse to the Quidditch field flying the Mexican flag. It was, Snape thought as he returned to his rooms from breakfast, rather enjoyable to see the blaze of color and hear the magically produced strains of \emph{`Cielito Lindo'} played in the entrance hall. He was looking forward to the celebratory lunch, and went to the Great Hall at noon with pleasurable anticipation.

The Slytherin table was empty.

"Where are they, Severus?" Dumbledore asked as the rest of the Hall filled while Slytherin remained conspicuous by its absence.

"I don't know, sir," Snape replied. "Let me check in the common room. I know many of them were planning to take part."

He met Chris Tobin coming across the entrance hall. "Thank goodness, Professor," Chris said. "I was coming to get you. We need you right away."

The common room was packed and, as Snape came through the Wall, silent. Algie stood defiant in the speaker's area by the great fireplace, and rage radiated through the room.

"What's wrong?" Snape asked. "They're expecting you in the Hall."

Algie spoke up. "They're expecting us all right, so they can laugh at us. And they're expecting us to parade to the Quidditch field under the Mexican flag. Well they won't get us. Have you ever seen the Mexican flag, sir?"

Numbly Snape shook his head. From under his arm, Algie pulled a vibrant piece of red, white, and green bunting and spread it out for the head of Slytherin house to see. There, on the central white vertical stripe, was the picture of an eagle, sitting on a cactus, killing a snake.

The Ravenclaw eagle devouring the Slytherin snake.

Snape felt sick. "Where did you get this?" he asked.

"They're setting them up on the seventh floor. They're not planning to unfurl them until they're halfway down the hill, when it'll be too late for the teachers to stop it. They were hoping to have lots of Slytherins in the procession to make our house look divided and fragmented. I spied on them, and I grabbed this one, and I ran. I had six Gryffindors after me all the way to the Wall."

"Of course we can't join them," said Snape. "If you'll give me that, I have to tell Dumbledore."

Taking the colorful piece of cloth, Snape left the common room, but it was already too late. Gryffindor, knowing its moments numbered, had jumped the gun and started the procession. They were leaving the Hall and heading across the lawn to the hill by the time Snape reached the entrance hall.

"Well," said McGonagall, slipping in beside him. "Where are they all?"

For answer, Snape handed her the flag.

"Oh," said McGonagall. "My. We must find Albus at once."

"You find him, if you don't mind," said Snape. "I have a house to look after."

Slytherin came down the hill after all the other houses, flying the green serpent banner. As they marched, they chanted: "Sly-ther-IN means to WIN! Sly-ther-IN means to WIN!" They could not match Gryffindor and Ravenclaw together in numbers, but they managed to match them in volume. Many Hufflepuffs deserted the school flags to join Slytherin, since they were to play Ravenclaw in a month's time.

As Snape made his way to the staff stands, to sit with McGonagall and Sprout between him and Flitwick, Kettleburn appeared at his side. "Do they have a chance?" Kettleburn whispered anxiously.

"A chance? Can't you hear them? They want blood. I didn't do that, Professor Kettleburn, they did it. With a little help from Gryffindor and Ravenclaw. You bet the way you feel, but my house wants blood."

"That's all I need to know," said Kettleburn, and left to place his bets.

Flitwick leaned forward and spoke across the others. "I'm sorry, Severus. I didn't know. If it's any help, I think my students see it as purely a Quidditch prank. Not directed against you as a house but as a match opponent."

Dumbledore appeared then. "It is your call, Severus. We can remove the flags if you want, though your house did catch on to the plot before they were made the butt of the joke..."

Snape looked at the stands, where Slytherin green was now accented by Hufflepuff yellow, the roars of defiance growing louder by the second. "No, sir," he said. "Let the flags stay. They'll only help us now."

With a nod and a quiet smile, Dumbledore took his seat.

Madam Hooch started the game, and the Keepers sped to their hoops. A Ravenclaw Chaser got the Quaffle and headed across the pitch, two Slytherin Chasers behind her on either side. As she entered the scoring area in front of the left hoop, the Ravenclaw suddenly dove down, sideways, and up to the right hoop, the Quaffle scoring before the Slytherin Keeper could adjust his position. The Ravenclaw and Gryffindor stands erupted in cheers.

Slytherin took the Quaffle, and the Ravenclaw Chasers united in a Parkin's Pincer to force him away from the hoops. Instead he broke away from them to soar high into the air, and as the three Ravenclaw Chasers went after him, he flung the Quaffle left-handed down to a waiting team mate who was already in the scoring area. Slytherin went wild as their team scored ten points.

The next Ravenclaw who tried to score was distracted by a Bludger that flew across his path, directly in front of his nose, and Slytherin was able to take the Quaffle again. Algie slammed another Bludger in the direction of the Ravenclaw Keeper and it immediately turned on him in attack, Ravenclaw's own Beaters being halfway down the pitch at the time. With the Keeper unable to protect his hoops, Slytherin scored again.

Meanwhile, high above the pitch, the two Seekers hovered unmoving. Both teams needed a high-scoring game, and for the time being the Snitch was unimportant.

There was no doubt from the moment the game began that the Ravenclaw Chasers had the advantage. They'd been playing together for four years, and their coordination was spectacular. Slytherin had the advantage in Keeper and Beaters. Ravenclaw's ability to hold on to the Quaffle in attempt after attempt meant that Slytherin, while blocking throw after throw, began to fall behind. The score stood at 60-20 when the Slytherin Seeker suddenly dove downward, hurtling toward the ground in a death-defying plunge, the Ravenclaw Seeker right behind.

At the last moment, the Seeker pulled away, skimming the edge of the pitch and climbing back up into the air. It was a feint, there had been no Snitch, but while the other players were distracted, Slytherin had scored a goal.

Fifteen minutes later, a Bludger took out Ravenclaw's Keeper, and for the short while he needed to recover, Slytherin was able to rack up points, but Ravenclaw's greater team experience was taking its toll. Twice more, the Slytherin Seeker tried a Wronski Feint, and the third time the Ravenclaw Seeker scarcely paid him any attention. After two hours of intense play, with the stands on either side screaming "Sly-ther-IN!" and "Ra-ven-CLAW!" the score stood at 150-90, and Ravenclaw was looking for the Snitch.

Algie and Chris turned the Bludgers on the Ravenclaw Seeker now, while the Ravenclaw scoring attack turned into a fight to protect their Seeker. As Slytherin managed one more goal, the Slytherin Seeker streaked downwards in yet another Wronski Feint.

Only this time it wasn't a feint. This time the Snitch darted and zigzagged near the ground. By the time the Ravenclaw Seeker realized this and joined the hunt, it was too late. Slytherin had the Snitch and the game, 250-150, and Ravenclaw's chances for the Cup were destroyed.

Slytherin and Hufflepuff went wild, green banners and yellow streamers waving high over their heads. Students poured onto the pitch to lift the players into the air on their shoulders and carry them back up the hill for a victory celebration. Ravenclaw and Gryffindor hung back, disappointed and quiet.

In the teachers' box, Flitwick reached over to shake Snape's hand. "Well played! Well played!" he exclaimed. "I haven't seen such an exciting game in years! I won't say it was worth losing just to see it, because it wasn't, but it was a great game!"

The most heartfelt congratulations came from Kettleburn. "Great game, lad! Well played! You had me worried for a bit, but it ended up fine!"

"I take it you decided not to bet on Ravenclaw," Snape commented.

"I got better odds betting on Slytherin. And a two-hour game with that much action would've been worth losing a bet on. To get both was icing on the cake."

They were following the students up to the castle. "I'd like to be able to promise you an equally long and exciting game next time, but the truth is that we need to win, and the quicker the better."

"How do you figure that?"

"If we win, we'll be the only house with three victories, and the score totals won't matter. But if Gryffindor wins, we'll have to be at least forty points ahead of them in Quaffle scoring in order to beat their point total. Gryffindor plays like Ravenclaw, and you saw how hard it was for us to keep up with them. They'll want to win fast, too, so that they can get the Cup on scoring."

"Sounds like you're actually learning how this game is played."

"I can't afford to sound like an idiot when I'm talking to my own team, can I? Besides, this part is mathematics, and a lot easier to understand."

They reached the castle to find the Great Hall had been taken over for a Slytherin-Hufflepuff party. The decorations from the `cultural celebration' remained, but new flags had been added. These were yellow, white, and green, and the central picture was of a snake killing an eagle.

"Are you going in?" Kettleburn asked.

"No. I'd just have to tell them to break up the party. I don't think they're supposed to be doing this in the Great Hall. As long as I don't go in, I can pretend I don't know about it."

McGonagall stopped next to them. "You know they shouldn't be carrying on like that in the Hall. What do you plan to do about it?"

"Not a thing. I haven't looked in, and I haven't seen it."

"Young man, you are shirking your duties!"

"Oh, come on, Professor. They've had so little to celebrate recently. Give them an hour."

"Professor Dumbledore," McGonagall called as Dumbledore came into the castle. "Professor Snape doesn't think it necessary to restrain his house's over-exuberant enthusiasm."

"Are they being over-enthusiastically exuberant, Minerva? That may be a matter of opinion. Perhaps we should discuss the parameters in my office."

"Albus! You know what you would do if this was Gryffindor!"

"Yes, Minerva. I should ignore it for a decent interval of time, then tell the students to relocate their festivities to another area. Their common room, for example. Surely you expect me to give Slytherin the same treatment."

McGonagall opened her mouth to respond, then closed it again. Around them the entrance hall was beginning to clear as Gryffindor and Ravenclaw students made their way upstairs to their houses. Since both Slytherin and Hufflepuff had their common rooms and dormitories in the lower areas, there would be no confrontations.

"It is now about forty-five minutes since the end of the game," Dumbledore went on, "most of which time was spent walking up the hill. If I disperse them now, and if Gryffindor wins the cup at the end of the year, Gryffindor will be granted no more time for public celebration than Slytherin is given now. Deal, Minerva?"

"You have a deal, Albus."

Dumbledore swept into the Great Hall in majesty, and those nearest the door quieted at once. "Slytherin house!" he called, and students shushed one another. "I have come to offer my congratulations, to the green and silver, for today Slytherin stands at the top, with two victories and no defeats..." The roar of cheering and banging on tables drowned him out for a moment, then they calmed down again. "It was a good game, well-played and hard-fought - I must say that I enjoyed watching every exciting minute of it." More cheers and pounding. "Now, however, there is another priority, and that is supper. The house-elves must have the hall cleared and cleaned and supper laid in an hour, and so I ask that you avail yourselves of those decorations you wish, and join your head of house in the Slytherin common room to continue your celebrations..."

Algie was up on a bench at once, still dressed in his Quidditch uniform. "Will you join us, too, Headmaster? Will you come to the common room and raise a glass to the Slytherin Quidditch team?"

"Master Colfax, I shall. And with the greatest pleasure."

That settled it. As the students poured from the hall back to their houses, Dumbledore took Snape's arm and walked with him, cheerfully waving to McGonagall as he disappeared down the steps into the dungeons.

Dumbledore didn't stay long in the Slytherin common room, but then no one expected him to. They showed him around with some pride, applauded when he toasted the team with pumpkin juice, and bade him good afternoon politely when he made his excuses and left, but the house as a whole was immensely pleased that he'd been there.

"Thank you for coming," Snape said as he walked with the headmaster back to the entrance hall. "It means a lot to them. Sometimes they feel ostracized."

"You must thank Master Colfax for inviting me. I could not have come otherwise. Remember that if Professor McGonagall challenges you."

"I will, sir. Thank you, sir."

By supper time the house was reasonably calm again, and made their way in to supper in a fairly normal manner, without unnecessary grandstanding. It appeared Gryffindor and Ravenclaw may have been prepared for a confrontation, but as it never reached that point, supper was peaceful.

Nothing else of moment happened all of March except for a brief flurry of excitement four days after the Quidditch match when a rumor spread through the school that the world was about to end. Someone received word from outside that on March 10 all of the planets would line up on the same side of the sun, and the resulting gravitational pull would tear the earth apart. Snape, with his interest in astronomy, patiently explained to his nervous students gathered in the common room that this had occurred many times in the past with no ill effects, and they were mollified. It was fun, however, to watch the rest of the school run around in panic.

Then the Easter break was upon them and, as at Christmas, the great majority of the students were leaving to spend a couple of weeks at home with their families. "Where will you go for the break, Severus?" Dumbledore asked at breakfast on the first Sunday of the break when the students were all departing for the train.

"Here. I'm not setting foot outside of Hogwarts."

"Do you wish my opinion?" Since Snape didn't reply, Dumbledore went on. "I think you should take advantage of the free time to get some things accomplished. You wanted to go to the bank, for example, or buy more books."

"And let them throw me in Azkaban for violating parole? No, I think not."

"You know, I have always been curious as to the interior of a muggle bank. I would take it as a great favor if you would allow me to see one."

"I suppose if I'm going to have a nursemaid, I may as well get top value for my money."

"My thoughts exactly."

Dumbledore did not pass muster the next day as they prepared to go to London.

"No. Absolutely not. You're not wearing a striped jacket with plaid trousers. Don't you have anything that's just black, or brown, or dark blue?"

The trousers became white, and the jacket navy blue, with brown shoes. Snape held his head in his hands. "You look like you just walked off a yacht on the valet's day off."

"I really do not understand why you are so particular."

"I'm going to a bank to do business. You already look odd enough with the long white hair and beard. In conservative clothing you are merely eccentric. In anything else, you look like a nut case."

The entire suit became navy blue with black shoes and a modest tam o'shanter. "I'll take it," Snape said before Dumbledore could change into anything else.

They apparated into the west end near Snape's bank. Instead of just making a withdrawal, Snape asked to speak to a bank officer about his options. They looked at the relative advantages of long-term time deposits while Dumbledore wandered around examining everything.

"Do you think you might get your grandfather to sit down?" the bank officer said after a while, as she helped Snape fill out forms for having the interest from a long-term deposit transferred to his normal savings account. "I think he's making the other customers nervous."

Dumbledore was corralled and seated next to Snape as the paperwork was completed. Snape was then offered a little plastic card. "What's this?" he asked.

"It's for the cash machine outside. You put the card in, enter your personal secret number, and retrieve the cash you want. It operates twenty-four hours a day, and you don't have to come into the bank to get money. The amount is automatically deducted from your account."

Snape tried the machine instead of withdrawing the money inside. It worked beautifully and could be used wherever his bank had a cash machine. He took his money, thanked the bank officer, and he and Dumbledore headed towards Charing Cross Road. They ran right into Moody before they reached the end of Leicester Square.

"Fancy seeing you here!" Moody exclaimed. "Can't hardly turn around but what I run into you in London. Morning, Albus. What brings you to town?"

"Business," replied Dumbledore. "I see you knew exactly where to find us."

"Me? Chance encounter, purely chance encounter. You don't think I lie in wait stalking people, do you?"

"The thought had crossed my mind. We are now on our way to a bookstore. Are you planning to accompany us?"

"Books? I love books! Lead me to a book, and I'll follow like a thirsty horse to water. I'd be honored to accompany you."

It wasn't comfortable for Snape to walk up Charing Cross Road with Alastor Moody trailing him, but Dumbledore and Moody chatted along the way like the old friends that they were, and left Snape alone. When they got to the bookstore, Dumbledore demurred. "I know I promised to stay with you, but if Alastor will be my companion I think I should enjoy myself more in the Leaky Cauldron." It was impossible for Moody to refuse, and so Snape was left unmolested in the bookstore.

He spent two hours there, haunting the sections on religion, philosophy, history, and literature, trying to decide. He ended up with a book of European myths and legends, another of myths from around the world, a classic work on the first emperors of Rome, and the most recent of the series of murder mysteries set in the twelfth century. It was a most satisfactory day's work.

No one bothered Snape on his way to the Leaky Cauldron, where he found Dumbledore and Moody deep in a debate about the Salem witchcraft trials. Snape slipped into a chair at the table with them.

"Did you find anything?" Dumbledore asked.

"Several things. It was a very profitable day. Isn't it lunchtime? Maybe we could eat here?"

"That is an excellent idea, Severus. We shall make a whole day of it," Dumbledore replied, but Moody rose to say good-bye. Though both were polite, neither Snape nor Dumbledore was truly sorry to see him leave.

"I did," said Dumbledore as they ordered a bite to eat, "get Alastor to promise that neither he nor any other auror would hound you if you came to London. I think you should be reasonably safe from here on in."

"Thank you, sir," said Snape

On Thursday, Snape got a message by owl. It was a rather tattered looking owl, something like those used by a public messenger service, and it didn't wait for a response. Snape watched it flap its way out of the Hall, then opened his letter. It was short and succinct.

\emph{Don't get angry. I tried everyone. Some can't help - some won't. I got to find a job. I have a drink at the Pig's Snout when I got a sickle. Some days I don't got a sickle. Please help. M. Bodkin.}

It took a moment for Snape to remember Marcellus Bodkin, a quiet, untalented wizard who'd always had trouble getting jobs, and who never rose above supply clerk in the clinic. Snape didn't think he'd even attended Hogwarts; he certainly never seemed well educated. A harmless mouse of a man that no one ever noticed.

The Pig's Snout was a lower class pub on the south side of the Thames, frequented by working class wizards. Snape knew where it was, though he'd never been inside. \emph{I guess it can't hurt to talk, and buy him a drink. I don't know how he thinks I can help.}

The problem was Moody. He couldn't lead the aurors to someone like Bodkin. And yet Moody 'd promised Dumbledore not to trail him anymore. Snape thought maybe this was a time to test Moody's promise.

It was the break, so there were no students or classes. That evening it was an easy matter to go into Hogsmeade as if for some shopping or relaxation, and from there to apparate to his usual London haunts. Snape wandered around Leicester Square for a while, but saw no one who looked like a wizard, much less an auror. Just to be on the safe side, he popped over to a secluded part of Hyde Park, checked that he wasn't followed, then apparated across the river.

The Pig's Snout was tucked into the end of an alley. Snape entered a smoke-filled room, looked around and saw Bodkin, and walked over to his table, seating himself where he could watch the door.

"It was good of you to come, sir," Bodkin said, profuse in his thanks when Snape bought drinks for them both. "Truth to tell, sir, I been down on my luck. Them as has jobs open, they want to know what a man's done with himself the last few years, and I can't tell 'em. I heard you got a position up there in that school, and I thought it's a big place and might need someone who can do a day's hard work."

"It's possible. Why don't you come up and ask?"

"Me? Go way up there where I'd stick out like a sore thumb and no guarantee they'd even talk to me? Might just as well walk into the Ministry and ask for a ticket to Azkaban. You, now, you could pave the way, like."

"I really don't have that kind of influence up there. I'm very much the junior teacher and in need of help myself. The only thing I could do would be to tell Dumbledore about you and see what he says."

"That's a sight more 'n most would do, sir. I'd be in your debt for that much."

They talked a few minutes longer, and it turned out Bodkin had a family, a wife and two children right there in Southwark, then Snape felt he should go. No one had entered the pub after him.

On the street everything was quiet in the fading evening light. No one noticed as Snape ducked into another alley to apparate back to Leicester Square, and no one noticed as he disapparated from there to Hogsmeade.

They came out of the shadows as he walked past the Hog's Head on his way to the gate, an auror squad with wands drawn, no telltale sounds of apparation to warn Snape of their approach. Snape held his hands up immediately, to show he held no wand and didn't intend to fight. They removed his wand from his sleeve and bound his wrists.

"Severus Snape," one of them said, "you're under arrest for violating your parole by continuing to have contact with Death Eaters without permission of the Ministry."

"Please let me inform..."

"Shut up, Death Eater," and with that they apparated back to London and the Ministry.

Once again Snape was pulled through the Ministry atrium and down into the lower levels while witches and wizards turned to stare. This time he was taken directly into a cell block and locked in a cage, a box of metal bars with only one solid wall and no chance for privacy, its only furniture a chair and a cot.

After half an hour's wait, Moody arrived with three other men and a witch stenographer. He threw a black and white striped shirt and trousers onto the chair, and a pair of slippers next to them - a convict's garb. "Take off your clothes and put those on. We'll record identifying marks and then you're going on a little trip."

"No," Snape said flatly, trying not to look at the smug witch with her parchment and quill.

"No? Suit yourself." The blue eye began to whirl in Moody's socket. "About five foot seven, and I'd say well under nine stone. Black hair, dark eyes, Dark Mark branded into left arm and extensive scarring on the back, looks like a whip. Who did that, Death Eater, Voldemort when he was feeling frisky? Mole on the left shoulder..."

Snape stood rigidly still as Moody described him for the others, his face burning with shame. Then the stenographer left. "Now," Moody continued, "take off your clothes and put those on. Unless you'd rather have me put you in a full body bind and we get to do it for you." Defeated, Snape slowly began to unbutton his jacket.

When he was dressed in the prison clothing, they took his own clothes away. Snape lay down on the cot and stared blankly at the wall through the bars on his cage. He could hear the voices in the cell block and knew they were looking for a judge to sign some papers. From time to time a new voice would intrude.

"Alastor, I need you to release this to my department. It's that muggle tooth-healer's chair that Quimbly had set up in his cellar. The one he was using to... Oh, got another one, did you?"

"You're working late tonight, Arthur. Yeah, that's one we're shipping north as soon as we can find a judge. What's the hurry about this chair?"

"Just need to clear things off my desk. I've got the papers right here to sign. And there, too, if you don't mind. Thank you, Alastor. Try not to work too hard."

Then Snape's cell door opened and Moody and his men came in. "You've got a meeting with a judge," Moody said. "On your feet. He's late for supper and wants to get this over with as soon as possible."

The slippers were a little too big, forcing Snape to shuffle as he followed Moody out of the cell block, down a little corridor, and into an office. He felt like he was in Azkaban already.

The judge looked up as they entered. He was old and gray-haired, and looked bored. "I don't see why this can't wait until morning, Alastor. Let him stew a while in a cell. These Death Eaters don't deserve any better. Why they keep thinking that `lord' of theirs is going to come back, I don't know. Well, as long as I'm here, give me the papers."

"I wouldn't have bothered you, your Honor, but there's a shipment going up to Azkaban just after midnight, and we wanted to get this over with."

"All right, all right. Let's see... trial... verdict... sentencing... probation... conspiracy... Not too bright, this one. Had a free ride at Hogwarts, then goes off consorting... Well, where's the pen?" He was dipping the quill into the ink when the office door opened.

"Good evening, Carter, Alastor, gentlemen," said a calm, dignified voice. "This is a bit late for you, isn't it, Carter?"

"Judge Wigglestaff was just helping us clear out some last minute business, your Honor," said Moody quickly.

"Good evening to you, Amelia," said the old judge. "Looks like you're working late as well."

"I've got a heavy docket. Who's this?" Judge Bones didn't look at Snape, but took the unsigned papers from the desk in front of Judge Wigglestaff. "Case looks familiar. I think I had it on arraignment. You'd think they'd have brains enough to stay out of trouble. I'll take this, Carter. My unfinished business anyway. You go home. I'll wager Ermentrude is waiting supper for you."

Judge Wigglestaff was only too happy to wash his hands of the whole business as Judge Bones took his place at the desk. She looked sharply at Snape. "Did you know the terms of your parole forbade you to have contact with any former Death Eaters?"

"Yes, ma'am."

"Why did you?"

"He asked for help. He needs a job. He was just a clerk at the clinic, and he has to support his family."

"That's not what he says."

"Ma'am?"

"I have here a statement that you were discussing the possible return of Lord Voldemort and the..."

"No, ma'am, that's not true." Realization was beginning to dawn, and Snape knew the trap was already sprung.

"We have here," said the judge, "a case of he said, he said. I can understand why you might lie about the conversation to protect yourself, but can you think of a reason why Mr. Bodkin would incriminate himself so deeply just because of you? It seems much more logical that he's telling the truth. You understand that unless you have another witness to back you up, I shall have to sign these papers? Do you have anything to say?"

Snape wracked his brain trying to think of something. "No, ma'am," he said at last, feeling as if the world had come to an end. "Could you let Professor Dumbledore know? I don't want him to think I ran away or anything."

"I am sure, since you were arrested at Hogwarts, that Professor Dumbledore already knows."

Snape looked up. "I was arrested in Hogsmeade, so I'm not sure if he knows or not."

"This says Hogwarts."

"No, ma'am. I didn't even get close to the gates. They picked me up right after I apparated in."

Judge Bones regarded Moody carefully. "Does this arrest report contain inaccuracies, Alastor? The prisoner says he was in Hogsmeade."

"I wasn't present, your Honor. I can check with the arresting aurors."

"Mr. Snape, how soon after you arrived in Hogsmeade did the arresting aurors get there?"

"I think they were already there, ma'am. I didn't hear them apparate in."

Judge Bones pushed the report away and put the quill down. "Alastor Moody," she said icily, "I have the greatest respect for your work as an auror, and sympathize with the fact that you have been through a very traumatic time, but this is taking things too far. I am going to give you a choice. First, we can investigate this matter, interviewing other patrons in the tavern and talking to witnesses in Hogsmeade who may have seen the arrest. If everything is as you say, he goes to Azkaban, but if I find one hint of entrapment I'll have you up on criminal charges. Or you can drop the matter, and I write up an injunction forbidding you to speak to this young man or come near him for the next ninety days, or even to discuss his case with other aurors. Which shall it be?"

Moody turned positively purple and fizzed like a steam engine, but in the end said, "I'll drop the charge, your Honor."

"Very wise. Bring Mr. Snape's belongings. I believe he should be getting back to Hogwarts."

Snape changed back into his own clothes in the privacy of the office, then went with Judge Bones back up to the atrium area. Professor Dumbledore was there waiting for him.

"You need to keep a leash on this one, Albus," Judge Bones said as they shook hands in greeting. "He's a wanderer."

"I owe you an immense debt, Amelia," Dumbledore replied. "You and Arthur both. We had no idea there was anything wrong until I got Arthur's message he was here pending transport. It was good of you to come back in tonight just for this. I am yours to command henceforth."

Back in Hogwarts, Dumbledore ordered Snape up to his office. "Of all the foolish, irresponsible... Severus, how could you go off like that to meet with another former Death Eater? - Here, have a glass of firewhisky and sit by the fire a bit. - Do you not understand that they will use any excuse...? - Have you eaten? You must be tired and hungry..."

After an hour of being alternately scolded and coddled, Snape went to his own rooms for the night, still shaking when he thought of how close he'd come to losing everything because he'd felt sorry for someone. 


